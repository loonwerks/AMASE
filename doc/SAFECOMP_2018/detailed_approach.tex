\section{Detailed Approach}
\label{sec:detailed_approach}

\subsection{Functionality}
In this section, we describe the main features and functionality of the Safety Annex.
%
An AADL model of the nominal system behavior specifies the hardware and software components of the system and their interconnections. This nominal model is then annotated with assume-guarantee contracts using the AGREE annex~\cite{NFM2012:CoGaMiWhLaLu} for AADL. The nominal model  requirements are verified using compositional verification techniques based on $k$-induction model checking~\cite{2017arXiv171201222G}.

Once the nominal model behavior is defined and verified, the Safety Annex can be used to specify possible faulty behaviors for each component. The faults are defined on each of the relevant components using a customizable library of fault nodes and the faults are assigned a probability of occurrence. A probability threshold is also defined at the system level. This extended model can be analyzed to verify the behavior of the system in the presence of faults. Verification of the nominal model with or without the fault model is controlled through the safety analysis option during AGREE verification.

To illustrate the syntax of the Safety Annex, we use an example based on the Wheel Brake System (WBS) described in ~\cite{AIR6110} and used in our previous work ~\cite{Stewart17:IMBSA}.
The fault library contains commonly used fault node definitions. An example of a fault node is shown below:
\begin{figure}[h!]
	\vspace{-0.19in}
	\begin{center}
		\includegraphics[trim=0 9 0 5,clip,width=1.0\textwidth]{images/faultNode.png}
	\end{center}
	\vspace{-0.4in}
\end{figure}

The \textit{fail\_to} node provides a way to inject a faulty input value. When the \textit{trigger} condition is satisfied, the nominal component output value is overridden by the \textit{fail\_to} failure value. In the WBS, the pump component generates an expected amount of pressure to a hydraulic line.  Declaration of a non-deterministic fault in the pump component is shown below:
\begin{figure}[h!]
	\vspace{-0.17in}
	\begin{center}
		\includegraphics[trim=0 330 150 0,clip,width=1.0\textwidth]{images/annex.png}
	\end{center}
	\vspace{-0.40in}
\end{figure}

The \textit{fault statement} consists of a unique description string, the fault node definition name, and a series of \textit{fault subcomponent} statements. \\
\textbf{Inputs} in a fault statement are the parameters of the fault node definition. In the example above, \textit{val\_in} and \textit{alt\_val} are the two input parameters of the fault node. These are linked to the output from the Pump component (\textit{pressure\_output.val}), and \textit{alt\_value}, a nondeterministic value defined within the Safety Annex. When the analysis is run, these values are passed into the fault node definition.\\
\textbf{Outputs} of the fault definition correspond to the outputs of the fault node. The fault output statement links the component output (\textit{pressure\_output.val}) with the fault node output (\textit{val\_out}). If the fault is triggered, the nominal value of \textit{pressure\_output.val} is overridden by the failure value output by the fault node. Faulty outputs can take deterministic or non-deterministic values. \\
\textbf{Probability} (optional) describes the probability of a fault occurrence.\\
\textbf{Duration} describes the duration of the fault; currently the Safety Annex supports transient and permanent faults.\\
%\textit{Equation Statements}: Equation statements support deterministic or nondeterministic types. For more details on equation statements, see ~\cite{NFM2012:CoGaMiWhLaLu}.

To-add: HW fault and dependent fault:

Fault propagations: it resides outside of fault statements and inside safety annex. This is because fault propagations are typically tied to the way components are connected or bound together; this information may not be available when faults are being specified for individual components. In addition, having fault propagations specified outside of a component’s fault statements makes it easier to reuse the component in different systems. 

HW fault: no behavioral aspects.

Dependent fault modeling: With the introduction of dependent faults, active faults are divided into two categories: independently active (activated by its own triggering event) and dependently active (activated when the faults they depend on become active). Top level fault hypothesis (e.g., maximum number of faults that can be active, or probability threshold for occurrences of faults) now applies to independently active faults. Faulty behaviors augment nominal behaviors whenever their corresponding faults are active (either independently active or dependently active).

\subsection{Architecture and Implementation}

The architecture of the Safety Annex is shown in Figure~\ref{fig:plugin-arch}.  It is written in Java as a plug-in for the OSATE AADL toolset, which is built on Eclipse.  It is not designed as a stand-alone extension of the language, but works with behavioral contracts specified AGREE AADL annex and associated tools~\cite{NFM2012:CoGaMiWhLaLu}.  AGREE allows {\em assume-guarantee} behavioral contracts to be added to AADL components.  The language used for contract specification is based on the Lustre dataflow language~\cite{Halbwachs91:IEEE}. AGREE improves scalability of formal verification to large systems by decomposing the analysis of a complex system architecture into a collection of smaller verification tasks that correspond to the structure of the architecture.

\begin{figure}
	\begin{center}
		%\includegraphics[trim=0 400 430 0,clip,width=0.85\textwidth]{images/arch.png}
		\includegraphics[width=.9\textwidth]{images/arch.png}
	\end{center}
	\vspace{-0.2in}
	\caption{Safety Annex Plug-in Architecture}
	\label{fig:plugin-arch}
\end{figure}

AGREE contracts are used to define the nominal behaviors of system components as {\em guarantees} that hold when {\em assumptions} about the values the component's environment are met.  The Safety Annex extends these contracts to allow faults to modify the behavior of component inputs and outputs.  To support these extensions, AGREE implements an Eclipse extension point interface that allows other plug-ins to modify the generated abstract syntax tree (AST) prior to its submission to the solver.  If the Safety Annex is enabled, these faults are added to the AGREE contract and, when triggered, override the nominal guarantees provided by the component.  An example of a portion of an initial AGREE node and its extended contract is shown in Figure~\ref{fig:comp}.  The \texttt{\_\_fault} variables and declarations are added to allow the contract to override the nominal behavioral constraints (provided by guarantees) on outputs.  In the Lustre language, \texttt{assertion}s are constraints that are assumed to hold in the transition system.

\begin{figure}
	\vspace{-0.1in}
	%\includegraphics[trim=30 150 120 10,clip,width=\textwidth]{images/sample_code.png}
	\includegraphics[width=\textwidth]{images/sample_code.png}
	\vspace{-0.3in}
	\caption{Nominal AGREE node and its extension with faults}
	\label{fig:comp}
\end{figure}

An annotation in the AADL model determines the fault hypothesis.  This may specify either a maximum number of faults that can be active at any point in execution (typically one or two), or that only faults whose probability of simultaneous occurrence is above some probability threshold should be considered.  In the former case, we assert that the sum of the true {\em fault\_\_trigger} variables is below some integer threshold.  In the latter, we determine all  combinations of faults whose probabilities are above the specified probability threshold, and describe this as a proposition over {\em fault\_\_trigger} variables.

Once augmented with fault information, the AGREE model follows the standard AGREE translation path to the model checker JKind~\cite{2017arXiv171201222G}, an infinite-state model checker for safety properties.  The augmentation includes traceability information so that when counterexamples are displayed to users, the active faults for each component are visualized.
