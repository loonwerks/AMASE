\subsection{Modeling Language for System Design}
\label{subsec:aadl-agree}
%Talk about AADL, AGREE, and why safety annex
%Pull AADL/AGREE background from previous papers to support points in the safety process
%Following the motivation/discussion in the process subsection, talk about why we choose to extend AGREE in safety annex, instead of using a separate safety model, or a semi-separate safety model like EMV2.
The Architectural Analysis and Design Language (AADL)~\cite{FeilerModelBasedEngineering2012} is an SAE International standard~\cite{AADL_Standard} that defines a language and provides a unifying framework for describing the system architecture for ``performance-critical, embedded, real-time systems''~\cite{AADL_Standard}. An AADL model describes a system in terms of a hierarchy of components and their interconnections, where each component can either represent a logical entity (e.g., application software functions, data) or a physical entity (e.g., buses, processors). An AADL model can be associated with properties and be extended with language annexes to provide a richer set of modeling elements for various system design and analysis needs (e.g., performance-related characteristics, configuration settings, dynamic behaviors). The language semantics supports formal analysis tools that allow for early phase error/fault detection.

The Assume Guarantee Reasoning
Environment (AGREE)~\cite{NFM2012:CoGaMiWhLaLu} implements as an AADL annex and annotates AADL components with formal behavioral contracts. Each component's contracts can include assumptions and guarantees about the component's inputs and outputs respectively, as well as predicates on how the state of the component can evolve over time.

AGREE translates an AADL model and the behavioral contracts into Lustre~\cite{Halbwachs91:IEEE} and then query a user-selected
model checker to conduct the back-end analysis. The analysis is performed compositionally along the architecture hierarchy such that analysis at a higher level is only using the components and their behavioral contracts from its immediate lower level. This allows the analysis to scale for the design of large and complex systems. 

In the avionics context, the software functions/applications, the hardware equipments, and the system that is composed of their integrations can all be represented as components connected to/composed of/bind to other components in a hierarchical AADL model. AGREE contracts can be used to capture the functional requirements at each level of the hierarchy. Once the model has been reviewed and the requirements captured have been validated, the back-end analysis can be conducted to verify if each level of the model implements its higher level requirements correctly.

AADL with the AGREE extension serves as a good candidate as the modeling language for describing the system design aspects of a shared system design and safety analysis model. Our prior work~\cite{Stewart17:IMBSA} adds the initial failure effect modeling capability to this language and tool set. However, the following goals are yet to be achieved before the combined language and tool set can be used to satisfy system safety objectives of ARP4754A~\cite{SAE:ARP4754A} and ARP4761~\cite{SAE:ARP4761}:

\begin{enumerate}
	\item Being able to provide a comprehensive, qualitative description of the causal relationship between basic events and system level safety requirements.
	\item Being able to provide an accurate, quantitative description of the contribution relationship between failure rates of the fault tree basic events and numerical probability requirements at the system level.
\end{enumerate}

The remainder of the paper describes our approach towards both of the goals.



