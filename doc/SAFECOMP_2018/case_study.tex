 \section{Case Studies}
\label{sec:case_study}
To evaluate the effectiveness of the Safety Annex, we looked at a number of case studies.  

\subsection{Quad-Redundant Flight Control System}
We applied the Safety Annex to the Quad-Redundant Flight Control System (QFCS) model~\cite{QFCS15:backes}.  We introduced faulty behaviors to see the response of the system to several faults, and to evaluate fault mitigation logic in the model.  The QFCS system-level properties failed when unhandled faulty behaviors were introduced.

We also used the Safety Annex to explore more complicated faults at the system level on a simplified QFCS model with cross-channel communication between its Flight Control Computers.

\begin{itemize} 
	\item Byzantine faults~\cite{Driscoll-Byzantine-Fault} were simulated by creating one-to-one connections from the source to multiple observers so that disagreements could be introduced by injecting faults on individual outputs. A system-level property failed due to the fault on the baseline model, but did not fail on the model with Byzantine fault handling protocol added. Using the Safety Annex like this can test a system's vulnerability to Byzantine faults and verify mitigation mechanisms.
	
	\item Dependent faults in hardware were modeled by injecting faults to hardware components (physical layer), and faults to software components (logical layer) that are bound to the hardware components, then specifying fault propagations at the QFCS system level to indicate that the software faults are dependent on the hardware faults.
\end{itemize}

\subsection{Simple Wheel Brake System}
The Wheel Brake System (WBS) described in ARP4761 has been used as a case study for safety analysis, formal verification, and contract based design. In the preliminary work of the Safety Annex, we used a simplified version of the WBS. 

\subsubsection{Simple WBS architecture description}
The highest level model component is the WBS. It consists of the Braking System Control Unit (BSCU), green and blue hydraulic pressure lines (supplied by the green and blue hydraulic pumps respectively), a selector which selects between normal operating mode and alternate operating mode, and the wheel system. 

There are three operating modes of the WBS. In \textit{normal} mode, the system uses the \textit{green} hydraulic circuit. In \textit{alternate} mode, the system uses the \textit{blue} hydraulic circuit.  If the BSCU detects lack of pressure from the green line or one of its command units are invalid, then the system switches into alternate mode. The last mode of operation of the WBS is the \textit{emergency} mode. This is supported by the blue circut but operates if the blue hydraulic pump fails. The accumulator pump has a reserve of pressurized hydraulic fluid and will supply this to the blue circuit in emergency mode.  Antiskid braking commands receive data from the BSCU that will determine if skidding is found at the wheel and handle accordingly. 

In the simplified WBS model, there is one wheel that receives pressure from either the green or blue line. This wheel provides feedback to the BSCU providing information about the pressure supplied. 

To evaluate the effectiveness of the Safety Annex, we updated the simple WBS model~\cite{Stewart17:IMBSA} to specify faulty component behaviors. The components' nominal  and faulty behaviors are modeled separately. At the top-level AADL component, the fault hypothesis was specified as the maximum number of faults that can be active at any time. The AGREE contracts at the top-level component were verified using AGREE, with the ``Perform Safety Analysis'' option selected. This signals the tool to weave the nominal and faulty behaviors into one augmented AGREE model before feeding to the model checker.

In this example, the top level contract ``Pedal pressed and no skid implies brake pressure applied'' was verified in the presence of at most one fault active during execution.  However, it was shown to be invalid when more than one fault was allowed. The counterexample indicated that both Selector's outputs failed to non-deterministic values due to the faults introduced.

\subsection{Extended Wheel Brake System}
In order to show scalability and better compare results with other tools and studies, the AADL model of the WBS was enhanced using as a guide the NuSMV ARCH4 model as described in~\cite{DBLP:conf/cav/BozzanoCPJKPRT15}. This version of the WBS model was chosen for our purposes due to the complexity of the model and the fact that this model addresses required safety concerns (for description of these concerns, see~\cite{DBLP:conf/cav/BozzanoCPJKPRT15}). Due to the added complexity of this WBS system, a short description of the subcomponents and behavior is necessary. 

\subsubsection{Extended WBS architecture description} 
The WBS is composed of two main systems: the control system and the physical system. The control system electronically controls the physical system and contains a redundant Braking System Control Unit (BSCU) in case of failure. The physical system consists of the hydraulic circuits running from hydraulic pumps to wheel brakes. This is what provides braking force to each of the 8 wheels of the aircraft. 

Just as in the simple WBS model, there are three operating modes. In \textit{normal} mode, the system uses the \textit{green} hydraulic circuit. The normal system is composed of the green hydraulic pump and one meter valve per each of the 8 wheels. Each of the 8 meter valves are controlled through electronic commands coming from the BSCU. These signals provide brake commands as well as antiskid commands for each of the wheels. The braking command is determined through a sensor on the pilot pedal position. the antiskid command is calculated based on information regarding ground speed, wheel rolling status, and braking commands. 

In \textit{alternate} mode, the system uses the \textit{blue} hydraulic circuit.  The wheels are all mechanically braked in pairs (one pair per landing gear). The alternate system is composed of the blue hydraulic pump, four meter valves, and four antiskid shutoff valves. The meter valves are mechanically commanded through the pilot pedal corresponding to each landing gear. If the system detects lack of pressure in the green circuit, the selector valve switches to the blue circuit. This can occur if there is a lack of pressure from the green hydraulic pump, if the green hydraulic pump circuit fails, or if pressure is cut off by a shutoff valve. If the BSCU unit becomes invalid, the shutoff valve is closed. 

The last mode of operation of the WBS is the \textit{emergency} mode. This is supported by the blue circut but operates if the blue hydraulic pump fails. The accumulator pump has a reserve of pressurized hydraulic fluid and will supply this to the blue circuit in emergency mode. 

\subsubsection{WBS translation from NuSMV to AADL/AGREE}
The ARCH4 NuSMV model is annotated with LTL formulae for each of the subcomponent behavioral contracts. To assist in the process of translation from LTL into AGREE, a translation tool was developed (MIKE: Not sure how we want to cite this or if we should even include it here....)

Upon compositional analysis of each layer of the model, details of the system behavior contracts needed further clarification. For instance, the control system of the WBS consists of a BSCU which has two channels for redundancy. All of the electrical components are subcomponents of the BSCU and this is what calculates the system validity, electronic pedal commands, and braking commands to the wheels. The contracts specified in the BSCU and all of it's subcomponents do take into account whether power is supplied to the control system. The higher level component Control System does not require power to be supplied in order to have correct behavior. In the AGREE behavioral model specifications, this needed to be added in order to complete compositional verification. 

\textbf{Note from Danielle: I think we should sum up the number of subcomponents, depth of the model in terms of layers, show time of verification for each of the compositional layers, number of fault nodes defined in total, and time it takes for fault analysis verificatiion. I will create a little table that shows some of these values. Does anyone know if there is a way to easily count the subcomponents without manually counting? An option in aadl perhaps...?}

Once the behavioral contracts were specified, both monolithic and compositional verification were run on the WBS model. There is a maximum layer depth of 6 for compositional verification and the time of verification was between 0 and 4 seconds. Monolithic verification was run on the top level of the system which took 24 seconds. 

\subsubsection{Fault Analysis of WBS using Safety Annex}
After the verification was completed, we defined faults equivalent to those described in the xSAP model for the NuSMV WBS system in~\cite{DBLP:conf/cav/BozzanoCPJKPRT15}. A total of 33 fault definitions were given to 18 different components and the top level compositional verification was run with one permanant fault introduced into the system. 

Compositional analysis on the top level WBS system was run with maximum one permanant fault present in the system. This caused the top level property to fail that there will be inadvertant braking at the wheel. This was caused by a fault on the pedal position sensor. This sensor determines if the mechanical brakes are pressed. If so, it sends an electrical command to the BSCU to command braking. The fault was an inverted boolean fault which inverted the mechanical pedal value (false) and created a true electrical pedal value. Thus, we have a situation where no  braking is commanded mechanically, but braking is commanded electronically nonetheless.





