\section{Case Studies}
\label{sec:case_study}

To evaluate the effectiveness of the Safety Annex, we updated the WBS model~\cite{Stewart17:IMBSA} to specify faulty component behaviors. The components' nominal  and faulty behaviors are modeled separately. At the top-level AADL component, the fault hypothesis was specified as the maximum number of faults that can be active at any time. The AGREE contracts at the top-level component were verified using AGREE, with the ``Perform Safety Analysis'' option selected. This signals the tool to weave the nominal and faulty behaviors into one augmented AGREE model before feeding to the model checker.
%The tool automatically weaves the nominal and faulty behaviors before feeding the result to the model checker.

In this example, the top level contract ``Pedal pressed and no skid implies brake pressure applied'' was verified in the presence of at most one fault active during execution.  However, it was shown to be invalid when more than one fault was allowed. The counterexample indicated that both Selector's outputs failed to non-deterministic values due to the faults introduced.

We also applied the Safety Annex to the Quad-Redundant Flight Control System (QFCS) model~\cite{QFCS15:backes}.  We introduced faulty behaviors to see the response of the system to several faults, and to evaluate fault mitigation logic in the model.  The QFCS system-level properties failed when unhandled faulty behaviors were introduced.

We also used the Safety Annex to explore more complicated faults at the system level on a simplified QFCS model with cross-channel communication between its Flight Control Computers.

\begin{itemize} 
	\item Byzantine faults~\cite{Driscoll-Byzantine-Fault} were simulated by creating one-to-one connections from the source to multiple observers so that disagreements could be introduced by injecting faults on individual outputs. A system-level property failed due to the fault on the baseline model, but did not fail on the model with Byzantine fault handling protocol added. Using the Safety Annex like this can test a system's vulnerability to Byzantine faults and verify mitigation mechanisms.
	
	\item Dependent faults in hardware were simulated by injecting faults to hardware components (physical layer) to affect their data outputs (logical layer), and consequently failing the software components bound to the hardware components. The relationship between the hardware components' outputs and the software components' inputs were specified in AGREE as part of the system's nominal behavior.	
\end{itemize}


\begin{comment}
\subsection{Safety Process Related Case Study}
value added to traditional safety assessment

\subsection{Scalability Related Case Study}
Talk about the findings from translating WBS LTL properties into AGREE/Safety Annex, e.g., missing including power in the system validity assessment that caused the compositional verification to fail?
\end{comment}