\section{Shared System and Safety Modeling}
\label{sec:safety_modeling}

In this section, we describe how different features of the Safety Annex can be used to describe the qualitative and quantitative causal relationship between faults and system safety requirements, and to perform safety analysis on a shared system and safety model.

\subsection{Mapping to Safety Concepts}
\janet{Can our approach describe faults, errors, failures? What about non-deterministic faults?
address Mike Peterson's comments}

\begin{enumerate}
	\item \textbf{Error} Definition and how we model it
	\item \textbf{Fault} Definition and how we model it
	\item \textbf{Failure} Definition and how we model it
	\item \textbf{Non-deterministic Faults}
	\item \textbf{Using AGREE verification for requirements validation}
\end{enumerate}

\begin{comment}
Errors/Faults/Failures - to a safety engineer, these terms have very specific meanings.  You will see my specific comments on this topic where I located them by your Section 3.3.  If needed, we can talk about this comment after I send you my mark-ups.
In Section 3.1, you introduce the term "non-deterministic".  I am not sure how your new process can be used by the safety engineering discipline unless things are "deterministic" and therefore "repeatable".
\end{comment}

\subsection{Interacting with System Design}

\begin{comment}
This paper should define how your team of authors see the interaction between "functions" and "system".  
Here is how I see the interaction / relationship:
(a) Aircraft Functions are the highest level
(b) Aircraft Functions are comprised of one or more System
(c) Each System performs one or more System Function
(d) Each System is comprised of one or more LRUs (a.k.a. "items")
\end{comment}

\subsection{Generating Safety Analysis Artifacts}