%\documentclass{sig-alternate-05-2015}
\documentclass{llncs}
\usepackage{makeidx}
\usepackage{tabularx,colortbl}
\usepackage[dvipsnames]{xcolor}
\usepackage{flushend}
\usepackage{cite}
\usepackage{amsmath}
%\usepackage{amsthm}
\usepackage{amssymb}
\usepackage{epsfig}
\usepackage{stmaryrd}
\usepackage{url}
\usepackage{multirow}
\usepackage{latexsym}
\usepackage{graphics}
\usepackage{graphicx}
\usepackage{enumitem}
\usepackage{comment}
\usepackage{longtable}
\usepackage{supertabular}
\usepackage{times}
\usepackage{listings}
\usepackage{subfigure}
\usepackage{color}
\usepackage{balance}
\usepackage{xspace}
\usepackage[ruled, vlined, linesnumbered]{algorithm2e}
\usepackage[autostyle]{csquotes}



%\theoremstyle{Definition}
%\newtheorem{definition}{Definition}
%%%
%\theoremstyle{Theorem}
%\newtheorem{theorem}{Theorem}


%\newcommand{\definition}{\noindent \textbf{Definition} \citation{}}
%\newcommand{\theorem}{\noindent \textbf{Theorem} \citation{}}
%\newcommand{\lemma}{\noindent \textbf{Lemma} \citation{}}

%\newdef{lemma}{Lemma}
%\newdef{definition}{Definition}
%\newdef{theorem}{Theorem}
%\newdef{corollary}{Corollary}
%\newdef{note}{Note}
%\newdef{axiom}{Axiom}
\newcommand{\mkeyword}[1]{\mbox{\texttt{#1}}}
\DeclareMathOperator{\kuop}{uop}
\DeclareMathOperator{\kbop}{bop}
\DeclareMathOperator{\kite}{ite}
\DeclareMathOperator{\kpre}{pre}
\DeclareMathOperator{\dom}{dom}
\DeclareMathOperator{\ktrue}{true}
\DeclareMathOperator{\kfalse}{false}
\DeclareMathOperator{\kselect}{select}
\DeclareMathOperator{\ran}{range}
\newcommand{\lbb}{[\![}
\newcommand{\rbb}{]\!]}
\newcommand{\expr}{\phi}
\newcommand{\exprS}{\Phi}
\newcommand{\mike}[1]{\textcolor{red}{#1}}
\newcommand{\mats}[1]{\textcolor{blue}{#1}}
\newcommand{\darren}[1]{\textcolor{green}{#1}}
\newcommand{\danielle}[1]{\textcolor{orange}{#1}}

\sloppypar



\begin{document}

\definecolor{gold}{rgb}{0.90,.66,0}
\definecolor{dgreen}{rgb}{0,0.6,0}
\newcommand{\stateequiv}{\equiv_{s}}
\newcommand{\traceequiv}{\equiv_{\sigma}}
\newcommand{\ta}{\text{TA}}
\newcommand{\cta}{\text{TA$_{C}$}}
\newcommand{\tta}{\text{TA$_{T}$}}
\newcommand{\ucalg}{\texttt{\small{IVC\_UC}}}
\newcommand{\ucbfalg}{\texttt{\small{IVC\_UCBF}}}


\title{Safety Annex for AADL}
%
\author{Danielle Stewart\inst{1}
\and Janet Liu\inst{2}
\and Michael W. Whalen\inst{1}
\and Darren Cofer\inst{2} }
\institute{University of Minnesota\\Department of Computer
Science and Engineering\\
200 Union Street\\
Minneapolis, MN, 55455, USA\\
\email{dkstewar, whalen@cs.umn.edu}
\and
Rockwell Collins\\
Advanced Technology Center\\400 Collins Rd. NE\\
Cedar Rapids, IA, 52498, USA\\ \email{ Jing.Liu, darren.cofer@rockwellcollins.com}
}
\maketitle

\begin{abstract}
This paper describes the Safety Annex for Architecture Analysis and Design Language (AADL). The safety annex provides model-based safety analysis features for systems which have been annotated with an Assume-Guarantee Reasoning Environment (AGREE). Using a quantitative reasoning approach, the safety annex provides a model-based safety analysis approach in which faults can be formalized and analyzed. The language for describing faults is extensible which allows safety engineers to weave various types of faults into the nominal system model. The safety annex supports the injection of faults into component level outputs and the behavior of the system can be analyzed using model checking support through AGREE. 


\end{abstract}

\keywords{Model-based systems engineering, fault analysis, safety engineering}

\section{Introduction}
\label{sec:intro}

%Mats' intro
System safety analysis is crucial in the development life cycle of critical systems to ensure adequate safety as well as demonstrate compliance with applicable standards. A prerequisite for any safety analysis is a thorough understanding of the system architecture and the behavior of its components; safety engineers use this understanding to explore the system behavior to ensure safe operation, assess the effect of failures on the overall safety objectives, and construct the accompanying safety analysis artifacts. Developing adequate understanding, especially for software components, is a difficult and time consuming endeavor. Given the increase in model-based development in critical systems~\cite{Joshi05:Dasc,CAV2015:BoCiGrMa,info17:HaLuHo,5979344,Gudemann:2010:FQQ:1909626.1909813}, leveraging the resultant models in the safety analysis process holds great promise in terms of analysis accuracy as well as efficiency.

In this paper we describe the \emph{Safety Annex} for the system engineering language AADL (Architecture Analysis and Design Language), a SAE Standard modeling language for Model-Based Systems Engineering (MBSE)~\cite{AADL_Standard}. The Safety Annex allows an analyst to model the failure modes of components and then ``weave'' these failure modes together with the original models developed as part of MBSE. The safety analyst can then leverage the merged behavioral models to propagate %failures
errors through the system to investigate their effect on the safety requirements. %(implicit %failure
%error propagation). 
Determining how %faults
errors propagate through software components is currently a costly and time-consuming element of the safety analysis process. 
\begin{comment} 
The use of behavioral contracts to capture the implicit %fault
error propagation characteristics of software component is a significant benefit for safety analysts.  
In addition, the annex allows modeling of explicit %failure 
error propagation that is not captured through the behavioral models, for example, the effect of a single electrical failure on multiple software components or the effect hardware failure (e.g., an explosion) on multiple behaviorally unrelated components. 
\end{comment}
The use of behavioral contracts to capture the %implicit %fault
error propagation characteristics of software component without the need to add separate propagation specifications (\emph{implicit} error propagation) is a significant benefit for safety analysts.  
In addition, the annex allows modeling of %explicit %failure 
dependent faults that are not captured through the behavioral models (\emph{explicit} error propagation)},
%error propagation that is not captured through the behavioral models, 
for example, the effect of a single electrical failure on multiple software components or the effect hardware failure (e.g., an explosion) on multiple behaviorally unrelated components. 
Furthermore, we will describe the tool support enabling engineers to investigate the correctness of the nominal system behavior (where no failures have occurred) as well as the system's resilience to component failures. We illustrate the work with a substantial example drawn from the civil aviation domain.

Our work can be viewed as a continuation of work conducted by Joshi et al.~where they explored model-based safety analysis techniques defined over Simulink/Stateflow~\cite{MathWorks} models~\cite{Joshi05:SafeComp,Joshi07:Hase,Joshi05:Dasc,DBLP:conf/cav/BozzanoCPJKPRT15}. Our current work extends and generalizes this work and provide new modeling and analysis capabilities not previously available.  For example, the Safety Annex allows modeling explicit %fault 
error propagation, supports compositional verification and exploration of the nominal system behavior as well as the system's behavior under failure conditions. Our work is also closely related to the existing safety analysis approaches, in particular, the AADL Error Annex (EMV2)~\cite{EMV2}, COMPASS~\cite{10.1007/978-3-642-04468-7_15}, and AltaRica~\cite{PROSVIRNOVA2013127,BieberERTS2018}. Our approach is significantly different from previous work in that unlike EVM2 we leverage the behavioral modeling for implicit %failure 
error propagation.  We provide compositional analysis capabilities not available in COMPASS.  In addition, the Safety Annex  is fully integrated in a model-based development process and environment unlike a stand alone language such as AltaRica. 

The contributions of the Safety Annex and this paper are:
\begin{itemize}
\renewcommand{\labelitemi}{\textbullet}
		\item close integration of behavioral fault analysis into the {\em Architecture Analysis and Design Language} AADL, which allows close connection between system and safety analysis and system generation from the model,
		\item support for {\em behavioral specification of faults} and their {\em implicit propagation} through behavioral relationships in the model, in contrast to existing AADL-based annexes (HiP-HOPS, EMV2) and other related toolsets (COMPASS, Cecilia, etc.),
		\item additional support to capture binding relationships between hardware and software and logical and physical communications, and
		\item guidance on integration into a traditional safety analysis process.
\end{itemize}

%The remainder of the paper is organized as follows. INSERT WHATEVER IT ENDED UP LOOKING LIKE. 

\begin{comment}
%Danielle's and Janet's intermediate intro

System safety analysis techniques are crucial in the development life cycle of highly integrated/complex aircraft systems and are used to show compliance with certification requirements. A prerequisite of performing any safety assessment of a system design is to understand how the system is intended to work, primarily focusing on the relationship between component outputs and the overall behavior of the system. The safety engineers then use this information to conduct safety analysis, construct the safety analysis artifacts, and compare the analysis results with established safety objectives and safety-related requirements. Acquiring knowledge about the behavior of the software applications hosted in a system and its impact on the overall system behavior is typically a time consuming and involved process.

To help solve this problem, researchers and industry practitioners have turned to the use of models. Models have been shown to be an effective way to help engineers capture, understand and analyze complex systems. Previous work has been done showing the benefits of leveraging the system model in the safety analysis process~\cite{Joshi05:SafeComp,Joshi07:Hase,Joshi05:Dasc,DBLP:conf/cav/BozzanoCPJKPRT15}.

%In order to effectively assist safety engineers to acquire the knowledge on software application behaviors and assess their effects on the overall behavior of the system, the models should allow system designers to capture the expected behaviors of the software application and the expected behavioral propagations among different components/application in the system; and allow safety engineers to leverage the same model provided by system designers, capture failure modes for individual components, and automatically assess the effects to the overall system through the behavioral propagations built in the existing model.

During this safety analysis process, it is important to reason about faults and how faulty component behaviors can impact the overall system behavior. In order to address the problem of understanding the model and complex system, it is advantageous to provide an automated analysis framework that allows for various types of fault definitions, propagations, and modeling options. 

This paper introduces a tool that provides a solution to these problems: the Safety Annex for the system engineering language called Architecture Analysis and Design Language (AADL), a widely used SAE Standard design language for MBSE applications~\cite{AADL_Standard}. Given a system model in AADL and a behavioral model developed in the Assume Guarantee Reasoning Environment (AGREE)~\cite{NFM2012:CoGaMiWhLaLu}, the Safety Annex is a fault modelling tool that utilizes model checking in order to analyze the behavior of a system in the presence of faults. The Safety Annex allows safety engineers to leverage existing models from system development for conducting assessment. 

Throughout this paper we show that the Safety Annex allows both implicit and explicit failure propagation which gives richer fault modeling capabilities than comparable tools. It is also shown how behavioral information regarding the active faults, the component properties and the overall system behavior when faults are active are provided. We demonstrate that the toolset (AADL, AGREE, and Safety Annex) captures behaviors of both the nominal model (absence of faults) and the faulty model in a cleanly separated and yet analyzable fashion. This serves to preserve the system model for the systems engineering process and simultaneously be able to see their combined effect on the system behavior. 


 %Using a Model-Based Safety Analysis (MBSA) approach allows safety engineers to weave a fault model into the entire MBSE process while preserving the separation of a system model and a fault model.


%To help solve this problem, researchers and industry practitioners have turned to Model-based System Engineering (MBSE). Models have been shown to be an effective way to help engineers capture, understand and analyze complex systems. Previous work has been done showing the benefits of leveraging the system model in the safety analysis process~\cite{Joshi05:SafeComp,Joshi07:Hase,Joshi05:Dasc,DBLP:conf/cav/BozzanoCPJKPRT15}. Using a Model-Based Safety Analysis (MBSA) approach allows an analyst \janet{use the term ``safety engineer'' only or both ``safety engineer'' and ``safety analyst''?} to weave a fault model into the entire MBSE process while preserving the separation of a system model and a fault model.
\end{comment}

\iffalse

Throughout the development life cycle of highly-integrated/complex aircraft systems, safety assessment process is a crucial piece in asserting development assurance, and is used to show compliance with certification requirements and meeting a company's internal safety standards. A prerequisite of performing any safety assessment of a system design is to understand how the system is intended to work, primarily focusing on the relationship between component outputs and the overall behavior of the system. The safety engineers then use the acquired understanding to conduct safety analysis, construct the safety analysis artifacts, and compare the analysis results with established safety objectives and safety-related requirements.  

In practice, prior to performing the safety assessment of a system, the safety engineers are often equipped with the domain knowledge about the system, but do not necessarily have detailed knowledge of how the software functions are designed. Acquiring the required knowledge about the behavior and implementation of each software function in a system is typically the most time consuming and involved step in the process.

Industry practitioners have come to realize the benefits and importance of
using models to assist the safety assessment process, such as to better understand system behaviors, communicate with system designers, capture the failure propagations, and manage and analyze more complex systems. And a revision to the Guidelines and Methods for Conducting Safety Assessment Process on Civil Airborne Systems and Equipment~\cite{SAE:ARP4761} to include {\em model based safety analysis} is under way.

%condensed version
%System safety assessment is a crucial process in the development of complex airborne systems to show that the relevant safety requirements are met. Acquiring the required knowledge about how the software functions are intended to work in such systems has shown in practice a very involved and time consuming task. Existing approaches that annotate the system architecture model with failure modes and fault propagations help safety analysts better communicate with system designers and address the complexity of the system. However, knowledge on how the faults propagate through the components still needs to be acquired by safety analysts before such information can be captured in the model. Acquiring the information on fault propagation is still a manual effort.

We think that the following criteria are important for the models to help safety analysts effectively acquire the knowledge about system/software behaviors and capture/analyze failure propagations:

\begin{itemize}
	\item Allow safety engineers to leverage existing models from system development for conducting assessment. %This captures the current state of the system design as it moves through the system development lifecycle, reducing the gap in comprehending the system behavior and transferring the knowledge between the system designers and the safety analysts.
	\item Support capturing behaviors of nominal and faulty behaviors in the system that are cleanly separated, yet analyzable in an integrated fashion to see their combined effect on the system.
	\item Enable safety engineers to inject failures/faults at component level, and assess the effect of behavioral propagation at system level, without needing to acquire the knowledge on the propagation beforehand. 
	\item Allow safety engineers to add failure propagations to the model that may not be behavioral related such as common cause/hardware dependent faults (e.g., common failures such as pipe burst that can propagate through physical systems).
\end{itemize}

\janet{To add: describe our approach that satisfy the criteria and all other work don't}
%The methodology described in this paper enables safety analysts to specify faults and faulty behaviors at individual components (using the Safety Annex for the Architecture Analysis and Design Language (AADL)). 

%The provided tool support auto weaves the faults into the nominal system model provided by system designers. No additional effort is needed to specify fault propagations as the faulty behavior propagates in the nominal system model the same as the normal behavior. The behavior of the system in the presence of faults are verified using model checking through Assume Guarantee Reasoning Environment (AGREE), and the effects of any triggered fault are manifested in the formal analysis results.

%1. Introduce our approach and it addresses all that
%2. behavioral propagation of failures
%3. We could just describe/split in implicit and explicit fault/failure propagations. E.g., for explicit failure propagation, now we can connect behaviorally unrelated components.
%3. You activate a fault/inject a failure to the system, so it's not fault propagation, but failure propagation

%How we evaluate our work in comparison with others'
%1. There are other approaches support some of them. However, they don't support ...
%What are the related work and why they don't solve the problem:
%Researchers like Anjali have explored ...
%EMV2
%xSAP
%2. What we do is different from XXX because ...
%3. The EMV2 is really talking about failure propagation. They view errors as corrupted states. That lead to certain level of confusion. 
%4. In ARP4754, an error is treated as a source of fault, but a fault can happen without error. An error might lead to a fault. A SW can have errors as software doesn't fail on its own - someone has to put it there

\begin{comment}
%This paper describes a new methodology with tool support for model based safety analysis. It is implemented as a {\em Safety Annex} for the Architecture Analysis and Design Language (AADL). The Safety Annex provides the ability to describe faults and faulty component behaviors in AADL models. In contrast to previous AADL-based approaches, the Safety Annex leverages a formal description of the nominal system behavior to propagate faults in the system. This approach ensures consistency with the rest of the system development process and simplifies the work of safety engineers. The language for describing faults is extensible and allows safety engineers to weave various types of faults into the nominal system model. The Safety Annex supports the injection of faults into component level outputs, and the resulting behavior of the system can be analyzed using model checking through the Assume-Guarantee Reasoning Environment (AGREE).

System safety analysis techniques are well-established and are a required activity in the development of safety-critical systems. Model-based systems engineering (MBSE) methods and tools based on formal methods now permit system-level requirements to be specified and analyzed early in the development process~\cite{NFM2012:CoGaMiWhLaLu,CAV2015:BoCiGrMa}. While model-based development methods are widely used in the aerospace industry, they are only recently being applied to system safety analysis.  

%How can we leverage these model-based methods and tools to perform safety analysis based on models of the system architecture and initial functional decomposition? Can these design models be integrated into the safety analysis process to help guarantee accurate and consistent results?
%Seeking solutions to these questions are especially important as the amount of safety-critical hardware and software in various domains has drastically increased due to the demand for greater autonomy, capability, and connectedness.

In this paper, we describe a {\em Safety Annex} for the Architecture Analysis and Design Language (AADL)~\cite{FeilerModelBasedEngineering2012} that provides the ability to reason about faults and faulty component behaviors in AADL models. In the Safety Annex approach, we use formal assume-guarantee contracts to define the nominal behavior of system components. The nominal model is then verified using the Assume Guarantee Reasoning Environment (AGREE)~\cite{NFM2012:CoGaMiWhLaLu}. The Safety Annex  provides a way to weave faults into the nominal system model and analyze the behavior of the system in the presence of faults. The Safety Annex also provides a library of common fault node definitions that is customizable to the needs of system and safety engineers. Our approach adapts the work of Joshi et. al in
~\cite{Joshi05:Dasc} to the AADL modeling language, and provides a domain specific language for the kinds of analysis performed manually in previous work~\cite{Stewart17:IMBSA}.  %More information on the approach is available in~\cite{Stewart17:IMBSA}, and the tool and relevant documentation can be found at: \small \url{https://github.com/loonwerks/AMASE/}. \normalsize

%There are other tools purpose-built for safety analysis, including AltaRica~\cite{PROSVIRNOVA2013127}, smartIFlow~\cite{info8010007} and xSAP~\cite{DBLP:conf/tacas/BittnerBCCGGMMZ16}. These notations are separate from the system development model. Other tools extend existing system models, such as HiP-HOPS~\cite{CHEN201391} and the AADL Error Model Annex, Version 2 (EMV2)~\cite{EMV2}. EMV2 uses enumeration of faults in each component and explicit propagation of faulty behavior to perform safety analysis. The required propagation relationships must be manually added to the system model and can become complex, leading to potential omissions and inconsistencies.

The Safety Annex supports model checking and quantitative reasoning by attaching behavioral faults to components and then using the normal behavioral propagation and proof mechanisms built into the AGREE AADL annex. This allows users to reason about the evolution of faults over time, and produce counterexamples demonstrating how component faults lead to system failures. It can serve as the shared model to capture system design and safety-relevant information, and produce both qualitative and quantitative description of the causal relationship between faults/failures and system safety requirements.
%
Thus, the contributions of the Safety Annex and this paper are:
\begin{itemize}
\item Close integration of behavioral fault analysis into the {\em architectural design language} AADL, which allows close connection between system and safety analysis and system generation from the model,
\item support for {\em behavioral specification of faults} and their {\em implicit propagation} through behavioral relationships in the model, in contrast to existing AADL-based annexes (HiP-HOPS, EMV2) and other related toolsets (COMPASS, Cecilia, etc.),
\item additional support to capture binding relationships between hardware and software and logical and physical communications, and
\item guidance on integration into a traditional safety analysis process.
\end{itemize}
%\mike{What are our contributions?}
\end{comment}

\fi

%\section{Safety Assessment Process}
\label{sec:process}
\mike{COMPLETELY STOLEN FROM THE SAFECOMP-05 PAPER!  Either note it or modify it}
\danielle{I added a citation. I think this late in the game we might as well just cite it.}

The overall safety assessment process that is followed in practice
in the avionics industry is described in the SAE standard ARP
4761~\cite{SAE:ARP4761}. Our summary in this section is largely
adopted from ARP 4761.

\iffalse

This section describes the overall safety assessment process that is followed in
practice in the avionics industry along the lines of the SAE standard ARP 47-61 \cite{SAE:ARP4761}. The descriptions of the various phases of the safety assessment process
covered in this section are essentially based on the ARP 47-61 document.

\fi


\begin{figure}
\includegraphics[trim=25 375 0 125, clip, scale=.60]{V}
\caption{Traditional ``V'' Safety Assessment Process} \label{fig:V}
\end{figure}

%The safety assessment process is an integral part of the development process.

Figure~\ref{fig:V} from \cite{Joshi05:SafeComp} shows an overview of the safety assessment
process as recommended in ARP 4761. The process includes safety
requirements identification (the left side of the ``V'' diagram)
and verification (the right side of the ``V'' diagram), that
support the aircraft development activities. An aircraft level
Functional Hazard Analysis (FHA) is conducted at the beginning of
the aircraft development cycle, which is then followed by system
level FHA for individual sub-systems. The FHA is followed by
Preliminary System Safety Assessment (PSSA), which derives safety
requirements for the subsystems, primarily using Fault Tree
Analysis (FTA). The PSSA process iterates with the design
evolution, with design changes necessitating changes to the
derived system requirements (and also to the fault trees) and
potential safety problems identified through the PSSA leading to
design changes. Once design and implementation are completed, the
System Safety Assessment (SSA) process verifies whether the safety
requirements are met in the implemented design. The system Failure
Modes and Effects Analysis (FMEA) is performed to compute the
actual failure probabilities on the items. The verification is
then completed through quantitative and qualitative analysis of
the fault trees created for the implemented design, first for the
subsystems and then for the integrated aircraft.

%\medskip

We propose to modify this traditional ``V'' process so that the
lower level PSSA and SSA activities are performed based on a
formal model of the system under consideration.
Figure~\ref{fig:Vmod} shows the modified ``V'' diagram for
model-based safety analysis. The shaded blocks are those
activities that will be modified or added.

\begin{figure}
\includegraphics[trim=15 350 0 125, clip, scale=.60]{Mod_V_Process_FaultModel}
\caption{Modified ``V'' Safety Assessment Process} \label{fig:Vmod}
\end{figure}


As we can observe from Figure~\ref{fig:Vmod}, the parts of the analysis that are
primarily affected are at the bottom of the ``V''. The biggest difference is that
the safety analysis activities at this level are now focused around a formal
model of the system behavior, and that many of the artifacts of the safety
analysis can be derived from this model. The idea is to try to pose the right
verification questions to formal tools (such as model checkers and theorem
provers) so that it is possible to derive the necessary safety analysis
information. We then wish to turn the results of these analyses back into
artifacts that can be easily understood and used by safety engineers.


\section{Functionality}
In this section, we describe the main features and functionality of the Safety Annex.
%
An AADL model of the nominal system behavior includes mechanical and digital components and their interconnections. This nominal model is then annotated with assume-guarantee contracts using the AGREE annex for AADL ~\cite{NFM2012:CoGaMiWhLaLu}. The nominal model safety requirements are verified using compositional verification techniques.

Once the nominal model behavior is defined and verified, the Safety Annex can be used on each of the system components that may be affected by faults. The faults are defined on each of the relevant components using a predefined customizable library of fault nodes. This extended model is then analyzed and the behavior of the system in the presence of faults can be seen. Separation of the nominal model from the fault model is handled through a selection mechanism provided to the user. The nominal model behavior can be defined and verified separate from fault model verification.

To illustrate the grammar and syntax of the Safety Annex, we will use an example from the Wheel Brake System (WBS) described in ~\cite{AIR6110} and used in our previous work ~\cite{Stewart17:IMBSA}.
The fault library contains commonly used fault node definitions. An example of a fault node is shown below.
\begin{figure}[h!]
\vspace{-0.19in}
\begin{center}
\includegraphics[trim=0 9 0 5,clip,width=1.0\textwidth]{images/faultNode.png}
\end{center}
\vspace{-0.4in}
\end{figure}

The \textit{fail\_to} node provides a way to input a failure value. When the fault is triggered, the nominal component output value is overridden by the \textit{fail\_to} failure value. In the WBS, the Selector valve component receives inputs from the hydraulic pressure pumps and a digital system component. The Selector will change modes of the system depending on the inputs it receives. See ~\cite{AIR6110,Stewart17:IMBSA} for more information. There are two main hydraulic lines that feed into the Selector, namely the green line and the blue line; respectively the normal and alternate modes of operation. The output of the Selector component will either be the green or the blue hydraulic fluid. To motivate the description of the syntax, we will go through a fault on the Selector component that describes a nondeterministic fault on the blue hydraulic line as shown in the example below.
\begin{figure}[h!]
\vspace{-0.17in}
\begin{center}
\includegraphics[trim=0 15 0 11,clip,width=1.0\textwidth]{images/annex.png}
\end{center}
\vspace{-0.40in}
\end{figure}

The \textit{fault statement} consists of a unique description string, the fault node definition name, and a series of \textit{fault subcomponent} statements. \\
\textit{Inputs}: The inputs in a fault statement are the parameters into the fault node definition. As shown in the example above, \textit{val\_in} and \textit{alt\_val} are the two parameters into the fault node. These are linked to the values found in \textit{blue\_output.val}, the output from the Selector component, and \textit{alt\_value}, a nondeterministic value defined within the Safety Annex. When the analysis is run, these values are passed into the fault node definition.
\textit{Outputs}: The outputs of the fault definition correspond to the outputs of the fault node. The fault output statement links the component output (\textit{blue\_output.val}) with the fault node output (\textit{val\_out}). If the fault is triggered, the nominal value of \textit{blue\_output.val} is overridden by the failure value output by the fault node.
\textit{Duration}: Currently the Safety Annex supports transient and permanent faults.
\textit{Equation Statements}: Equation statements support deterministic or nondeterministic types. For more details on equation statements, see ~\cite{NFM2012:CoGaMiWhLaLu}.\\







\section{Architecture and Implementation}


The architecture of the Safety Annex is shown in Figure~\ref{fig:plugin-arch}.  It is written in Java as a plug-in for the OSATE AADL toolset, which is built on Eclipse.  It is not designed as a stand-alone extension of the language, but works with behavioral contracts specified AGREE AADL annex and associated tools~\cite{NFM2012:CoGaMiWhLaLu}.  AGREE allows {\em assume-guarantee} behavioral contracts to be added to AADL components.  The language used for contract specification is based on the Lustre dataflow language~\cite{Halbwachs91:IEEE}. AGREE improves scalability of formal verification to large systems by decomposing the analysis of a complex system architecture into a collection of smaller verification tasks that correspond to the structure of the architecture.

\begin{figure}
\begin{center}
%\includegraphics[trim=0 400 430 0,clip,width=0.85\textwidth]{images/arch.png}
\includegraphics[width=.9\textwidth]{images/arch.png}
\end{center}
\vspace{-0.2in}
\caption{Safety Annex Plug-in Architecture}
\label{fig:plugin-arch}
\end{figure}

AGREE contracts are used to define the nominal behaviors of system components as {\em guarantees} that hold when {\em assumptions} about the values the component's environment are met.  The Safety Annex extends these contracts to allow faults to modify the behavior of component inputs and outputs.  To support these extensions, AGREE implements an Eclipse extension point interface that allows other plug-ins to modify the generated abstract syntax tree (AST) prior to its submission to the solver.  If the Safety Annex is enabled, these faults are added to the AGREE contract and, when triggered, override the nominal guarantees provided by the component.  An example of a portion of an initial AGREE node and its extended contract is shown in Figure~\ref{fig:comp}.  The \texttt{\_\_fault} variables and declarations are added to allow the contract to override the nominal behavioral constraints (provided by guarantees) on outputs.  In the Lustre language, \texttt{assertion}s are constraints that are assumed to hold in the transition system.

\begin{figure}
\vspace{-0.1in}
%\includegraphics[trim=30 150 120 10,clip,width=\textwidth]{images/sample_code.png}
\includegraphics[width=\textwidth]{images/sample_code.png}
\vspace{-0.3in}
\caption{Nominal AGREE node and its extension with faults}
\label{fig:comp}
\end{figure}

An annotation in the AADL model determines the fault hypothesis.  This may specify either a maximum number of faults that can be active at any point in execution (typically one or two), or that only faults whose probability of simultaneous occurrence is above some probability threshold should be considered.  In the former case, we assert that the sum of the true {\em fault\_\_trigger} variables is below some integer threshold.  In the latter, we determine all  combinations of faults whose probabilities are above the specified probability threshold, and describe this as a proposition over {\em fault\_\_trigger} variables.

Once augmented with fault information, the AGREE model follows the standard AGREE translation path to the model checker JKind~\cite{2017arXiv171201222G}, an infinite-state model checker for safety properties.  The augmentation includes traceability information so that when counterexamples are displayed to users, the active faults for each component are visualized.





\section{Applications}

To evaluate the effectiveness of the Safety Annex, we updated the WBS model~\cite{Stewart17:IMBSA} to specify faulty component behaviors. The components' nominal  and faulty behaviors are modeled separately. At the top-level AADL component, the fault hypothesis was specified as the maximum number of faults that can be active at any time. The AGREE contracts at the top-level component were verified using AGREE, with the ``Perform Safety Analysis'' option selected. The tool automatically weaves the nominal and faulty behaviors before feeding the result to the model checker.

In this example, the top level contract ``Pedal pressed and no skid implies brake pressure applied'' was verified in the presence of at most one fault active during execution.  However, it was shown to be invalid when more than one fault was allowed. The counterexample indicated that both Selector's outputs failed to non-deterministic values due to the faults introduced.

We also applied the Safety Annex to the Quad-Redundant Flight Control System (QFCS) model~\cite{QFCS15:backes}.  We introduced faulty behaviors to see the response of the system to several faults, and to evaluate fault mitigation logic in the model.  The QFCS system-level properties failed when unhandled faulty behaviors were introduced.

We also used the Safety Annex to explore more complicated faults at the system level on a simplified QFCS model with cross-channel communication between its Flight Control Computers.

\begin{itemize} 
	\item Byzantine faults~\cite{Driscoll-Byzantine-Fault} were simulated by creating one-to-one connections from the source to multiple observers so that disagreements could be introduced by injecting faults on individual outputs. A system-level property failed due to the fault on the baseline model, but did not fail on the model with Byzantine fault handling protocol added. Using the Safety Annex like this can test a system's vulnerability to Byzantine faults and verify mitigation mechanisms.
	
	\item Dependent faults in hardware were simulated by injecting faults to hardware components (physical layer) to affect their data outputs (logical layer), and consequently failing the software components bound to the hardware components. The relationship between the hardware components' outputs and the software components' inputs were specified in AGREE as part of the system's nominal behavior.	
\end{itemize}




















\section{Conclusions \& Future Work}
In this paper, we describe our initial work towards performing MBSA using the AADL architecture description language using a failure effect modeling approach.  Our goal is to be able to perform safety analysis on common models used by systems and safety engineers for functional and non-functional analyses, schedulability, and perhaps system image generation.  To perform this analysis, we use existing capabilities within AADL to describe the structure of the system, and build on the existing AGREE framework for compositional analysis of components.  

As part of our exploration, we are interested in examining the strengths and weaknesses of our FEM and the AADL Error Annex FLM-based approach.  We believe that the FEM approach has advantages both in terms of brevity of specifications and accuracy of results, and can build on existing analyses performed for systems engineering.  However, there are also risks in the FEM approach involving incomplete or mis-specified properties.  

We illustrated the ideas using architecture models based on the Wheel Braking System model in SAE AIR 6110 \cite{AIR6110} and use this in the evaluation of our approach. Using assume-guarantee compositional reasoning techniques, we prove a top level property of the wheel brake system that states when the brake pedals are pressed in the absence of skidding, there will be hydraulic pressure supplied to the brakes.  

Starting from the error model notions of error types, two main faults were defined: \textit{fail\_to} which will describe failures of valves and pressure regulators and \textit{inverted\_fail} which describes the failures occurring to components that output boolean values. Using the AADL behavioral model of the WBS, these permanent faults were tied into the nominal model in order to reason about how this model behaves in the presence of specific kinds of faults.

In order to demonstrate that the system was resilient to single faults, we modified the model to allow feedback from the wheel pressure to the BSCU.   This changed the way the system responded to faults that were further downstream of the BSCU or Selector and created a chance for the system to switch to alternate forms of hydraulic pressure. We also reasoned about the initialization values of the system in regards to which mode is the starting mode. It is crucial for the system to begin in Normal mode in order to function successfully in the presence of faults.  After model modification and a small weakening of our original property to account for feedback delay, the model does fulfill the top level contract even when a permanent fault of one of the high level components is introduced.

The current capabilities of AGREE are well-suited to specifying faults.  Our approach allows for scalar types of unbounded integers and reals, as well as composite types such as tuples and structures.  It is possible to model systems and reason about them in either discrete time or real-time.  However, adding faults to existing components is cumbersome and can obscure the nominal behaviors of the model.  We are currently examining several fault specification languages, giving special consideration to the xSAP modeling language.

Future research work will involve the continuation of development of the methods and tools needed to perform model-based safety analysis at the system architecture level. By introducing a common set of models for both nominal system design and safety analysis, we hope to reduce the cost of development and improve safety. Our hope is to demonstrate the practicality of formal analysis for early detection of safety issues that would be prohibitively expensive to find through testing and inspection. We will base this research on industry standard notations that are being used in airborne and ground-based avionics in order to ensure transition of this technology.

\subsection*{Acknowledgements} This research was funded by NASA AMASE NNL16AB07T and University of Minnesota College of Science and Engineering Graduate Fellowship.



%ACKNOWLEDGMENTS are optional
%TODO: Fill in for final version
\vspace{0.08in}


%\textbf{Acknowledgments:}
%This work was supported by

%We thank XXXX

\bibliographystyle{abbrv}
\bibliography{biblio}

% This ~ seems to fix an odd bibliography alignment issue
~

%\ifdefined\TECHREPORT
%\appendix
%
%\section{Appendix: Proof of Equivalence}
%\input{appendix}
%\fi

%\section{Appendix: GPCA CENTA Model}
%\label{appendix:gpcacenta}
%\begin{figure}[!ht]
%\begin{center}
%\includegraphics[scale=0.6]{images/sampled_pca.PNG} %[trim = 0 2 0 0, clip=true]{Comp}
%\caption{GPCA AGREE Properties modeled as a Timed Automata} \label{fig:samplepca}
%\end{center}
%\end{figure}

%\balancecolumns

\end{document}
