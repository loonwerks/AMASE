\section{Introduction}
System safety analysis techniques are well established and are a required activity in the development of safety-critical systems. While model based development methods are widely used in the aerospace industry, it has been a recent development that these methods are applied to safety analysis. Formal model-based systems engineering (MBSE) methods and tools now permit system-level requirements to be specified and analyzed early in the development process~\cite{NFM2012:CoGaMiWhLaLu,CAV2015:BoCiGrMa}. These tools can also be used to perform safety analysis based on the system architecture and initial functional decomposition. Design models can be integrated into the safety analysis process to help guarantee accurate and consistent results. This integration is especially important as the amount of safety-critical hardware and software in various domains has dramatically increased due to desire for greater autonomy, capability, and connectedness.

The Safety Annex for Architecture Analysis and Design Language (AADL) provides the ability to reason about faults and faulty component behaviors in AADL models. In the Safety Annex approach, we use an assume-guarantee reasoning environment (AGREE)~\cite{NFM2012:CoGaMiWhLaLu} to define the contracts of system components. The nominal model is then verified using compositional reasoning techniques. The Safety Annex then provides a way to inject faults into the nominal system model and analyze the behavior of the system in the presence of faults. The Safety Annex provides a library of common fault node definitions that is customizable to the needs of an analyst. 

There are tools purpose built for safety analysis such as AltaRica~\cite{PROSVIRNOVA2013127}, smartIFlow~\cite{info8010007} and xSAP~\cite{DBLP:conf/tacas/BittnerBCCGGMMZ16}. These notations are separate from the safety analysis system model. Other tools extend existing system models, such as HiP-HOPS~\cite{CHEN201391} and the AADL error annex~\cite{SAEAS}. The Safety Annex provides a way to extend the AADL system model and easily incorporate safety analysis techniques into the model development process. The Error Annex in AADL primarily uses qualitative reasoning regarding faults. The faults must be enumerated and their propagations through system components must be explicitly defined. These propagation relationships become increasingly complex proportional to the complexity of the system in question. Furthermore, complex propagation relationships may be easily overlooked by the analyst and thus may not be explicitly described within the fault model. To this end, the Safety Annex provides a quantitative reasoning environment in which to describe faults and analyze the behavior of the fault model.

To a large extent, our work has been an adaptation of the work of Joshi et. al~\cite {Joshi05:Dasc} to the AADL modeling language and an extension of the work of Stewart et. al~\cite{Stewart17:IMBSA}. The tool and relevant documentation can be found at: \small \url{https://github.com/loonwerks/AMASE/}. \normalsize




