\section{Analysis of the Fault Model}
\label{sec:fault_analysis}

When the Safety Annex is enabled, users can invoke either the monolithic analysis or compositional analysis in AGREE to check if the top level safety properties of the system hold in the presence of faults under the fault hypothesis given for the system. If an active fault causes the violation of a contract, a counterexample is provided by the model checker. The counterexample can be used to further analyze the system design and make necessary updates to the shared model between safety assessment and system development processes. This iterations continues until the system safety property is satisfied with the desired fault tolerance and failure probability achieved.

\subsection{Fault Hypothesis}
As the number of component faults increases, the different fault combinations can grow exponentially, making model checking infeasible. Therefore, a fault hypothesis needs to be specified for the system under verification to limit the simultaneous fault activations that are considered by the model checker.

A Safety Annex annotation in the system implementation of the AADL model determines the fault hypothesis. There are two types of fault hypothesis:

The \textit{max fault hypothesis} specifies a maximum number of faults that can be active at any point in execution. This is analogous to restricting the cutsets to a specified maximum number of terms in the fault tree analysis in
traditional safety analysis. In implementation (i.e., the translated Lustre model feeding into the model checker), we assert that the sum of the true {\em fault\_\_trigger} variables is below some integer threshold. Each layer of the model needs to have a max fault hypothesis statement specified in order to consider fault activation in that layer in the analysis.

The \textit{probabilistic fault hypothesis} specifies that only faults whose probability of simultaneous occurrence is above some probability threshold should be considered (see Figure~\ref{fig:hwFaultProp}). This is analogous to restricting the cutsets to only those whose probability is above some set value. In implementation, we determine all combinations of faults whose probabilities are above the specified probability threshold and describe this as a proposition over {\em fault\_\_trigger} variables. Each subcomponent fault needs to specify a probability of occurrence in order to be considered in the analysis.

With the introduction of dependent faults, active faults are divided into two categories: independently active (activated by its own triggering event) and dependently active (activated when the faults they depend on become active). The top level fault hypothesis applies to independently active faults. Faulty behaviors augment nominal behaviors whenever their corresponding faults are active (either independently active or dependently active).

\subsection{Monolithic Analysis}
When monolithic analysis is performed on the nominal system model, the architectural model is flattened in order to perform the analysis. All of the contracts in the lower levels are used for the analysis.
%A top level threshold is defined within the safety annex located in the top level system implementation. In the lower levels, each defined subcomponent fault is given a probability of occurrence. 

Given a probabilistic fault hypothesis, this corresponds to performing a %prior 
analysis on which combinations of faults have a probability less than the threshold and then inserting assertions into the Lustre code accordingly. If the probability of such combination of faults is in fact less than the designated top level threshold, these faults may be activated and the behavioral effects can be seen through a counterexample.  

To perform this analysis, it is assumed that the non-hardware faults occur independently and possible combinations of faults are computed and passed to the Lustre model to be checked by the model checker. As seen in Algorithm 1, the computation first removes all faults from consideration that are too unlikely given the probability threshold. The remaining faults are arranged in a priority queue $\mathcal{Q}$ from high to low. Assuming independence in the set of faults, we take a fault with highest probability from the queue (step 5) and attempt to combine the remainder of the faults in $\mathcal{R}$ (step 7). If this combination is lower than the threshold (step 8), then we do not take into consideration this set of faults and instead remove the tail of the remaining faults in $\mathcal{R}$. %The reason we can do this is because of the arrangement in priority queue from highest to lowest value. If this combination is below threshold, certainly any other combination of these faults with one of lesser value in the priority queue will also be below threshold. 
 
In this calculation, we assume independence among the faults, but in the Safety Annex it is possible to define dependence between faults using a %\textit{hardware fault} node\
fault propagation statement. After fault combinations are computed using Algorithm 1, the triggered dependent HW faults are added to the combination as appropriate. 

\begin{algorithm}[H]
	% \KwData{this text}
	% \KwResult{how to write algorithm with \LaTeX2e }
	$\mathcal{F} = \{\}$ : fault combinations above threshold \;
	$\mathcal{Q}$ : faults, $q_i$, arranged with probability high to low \;
	$\mathcal{R} = \mathcal{Q}$ , with $r \in \mathcal{R}$\;
	\While{$\mathcal{Q} \neq \{\} \land \mathcal{R} \neq \{\}$ }{
		$q =$ removePriorityElement($\mathcal{Q}$) \;
		\For{$i=0:|\mathcal{R}|$}{
			$prob = q \times r_i$ \;
			\eIf{prob $<$ threshold}{
				removeTail($\mathcal{R}, j=i:|\mathcal{R}|$)\;
			}{
				add($\{q, r_i\}, \mathcal{Q}$)\;
				add($\{q, r_i\}, \mathcal{F}$)\;
			} % end if else
		} % end for
	} % end while
	\caption{Monolithic Probability Analysis}
\end{algorithm}

%After all possible fault combinations are computed from Algorithm 1, we look at the collection of propagation statements used in HW fault definitions and add additional faults into the possible fault combinations if a fault that triggers the fault can become active, as computed from Algorithm 1.

%At the end of Algorithm 1, the possible fault combinations reside in the list $\mathcal{F}$. We then look at the collection of propagation statements used in HW fault definitions. These have a source (HW fault) and destination (faults triggered by HW fault). 

%Let $\mathcal{P}$ be the collection of propagation statements. For all $S \subset \mathcal{F}$, check to see if for $f \in S$, $f \in \mathcal{P}$ as a source. If so, add the corresponding destinations to the set $S$. This set $\mathcal{F}$ of allowed fault combinations is then added as a constraint to the Lustre model and thus they become active. If an active fault causes the violation of a contract, this is seen in a counterexample provided by the model checker.

\subsection{Compositional Analysis}
In compositional analysis, the analysis proceeds in a top down fashion. To prove the top level properties, the properties in the layer directly beneath the top level are used to perform the proof. The analysis proceeds in this manner.

The compositional analysis currently works with the max fault hypothesis. Users can constrain the maximum number of faults within each layer of the model by specifying the maximum fault hypothesis statement to that layer. If any lower level property failed due to activation of faults, the property verification at the higher level can no longer be trusted because the higher level properties were proved based on the assumption that the direct sublevel contracts are valid.

The compositional analysis is helpful to see weaknesses in a given 
layer of the system. In future work, we plan to reflect lower layer
property violations in the verification results of higher layers in the architecture and enable the display or constraint active faults system wide instead of layer wide.


%\subsection{Analysis Results of the WBS Example}
\subsection{Analysis of the WBS Example}
\label{sec:results}

The fault analysis on the top level WBS system was performed on the 11 top-level properties applying the max fault hypothesis and probabilistic fault hypothesis separately. It requires between 2 and 4 minutes to run either compositional analysis with max fault hypothesis or monolithic analysis with probabilistic fault hypothesis on the model. The analysis is computationally inexpensive, allowing quick iterations between systems and safety engineers.

We first applied compositional analysis with max one fault hypothesis at all layers of the model. Most properties were verified except for the \textit{Inadvertent braking at the wheel} properties. The failure of the verification for those properties shows that they are not resilient to a single fault. We then applied monolithic analysis with probabilistic fault hypothesis of $10^{-9}$ fault threshold. The \textit{Inadvertent braking at the wheel} properties also failed. Results from the first round of checks indicate that the WBS design is not fault tolerant to the inadvertent braking properties and is not meeting the Probability of Failure goal of $10^{-9}$ on those properties.

The counterexample returned by the tools allowed us to straightforwardly diagnose the fault conditions that lead to property failure: in this model, there is a single pedal position sensor for the brake pedal. If this sensor fails, it can command braking without a pilot request. The counterexample can be used to further analyze the system design and explore a solution to the problem. There are several ways to proceed (here we note that the architecture of the pedal assembly is not discussed in AIR6110):
	\begin{itemize}
	\renewcommand{\labelitemi}{\textbullet}
	\item Decrease the probability of failures of the brake pedal sensor. The failure probabilities are estimated from the failure rates and exposure times of the events. We may adjust the exposure time to match the phase of flight, rather than normalizing it per-flight-hour, if the phase of flight is sufficiently short. We may also find a brake pedal sensor with a lower failure rate. For example, if the failure probability for the sensor component is lower than the $10^{-9}$ fault threshold, it will not be considered in the possible fault combinations for the analysis. Decreasing the probability of failure of the components could help increase the reliability of the system. However, in this case the system still has a single point of failure.
	
	\item Create redundancy in the sensor component and model a voting procedure.
	In order for braking to occur, both (or all) sensors must agree. This would eliminate a single point of failure and would also cause the model to meet the top level probabilistic threshold. However, by introducing redundancy, it could affect the probability of meeting of other properties in the system that command braking, for example, when braking is commanded, braking pressure is provided at the wheel. Introducing the redundancy can increase the integrity of the system with respect to inadvertent braking, but decrease the availability of the system with respect to braking when needed.
\end{itemize}

Providing a way to quickly and effectively run analysis on the model %so that 
for these different modes of failure can be of great benefit to assist system engineers to make design decisions and safety engineers to assess the effect.

\begin{comment}
The fault analysis on the top level WBS system was performed on the 11 top-level properties \janet{applying the max fault hypothesis and probabilistic fault hypothesis separately. It requires between 2 and 4 minutes to run either compositional analysis with max fault hypothesis or monolithic analysis with probabilistic fault hypothesis on the model. The analysis is computationally inexpensive, allowing quick iterations between systems and safety engineers.}

\janet{We first applied compositional analysis with max one fault hypothesis at all layers of the model.} Most properties were verified except for the \textit{Inadvertent braking at the wheel} properties. \janet{The failure of the verification for those properties shows that they are not resilient to a single fault.} The {\em counterexample} returned by the tools allowed us to straightforwardly diagnose the fault conditions that lead to property failure: In this model, there is a single pedal position sensor for the brake pedal. If this sensor fails, it can command braking without a pilot request.

This counterexample can be used to further analyze the system design.  For our model, there are several possible reasons for failure: it could be that redundant sensors are required on the pedals (here we note that the architecture of the pedal assembly is not discussed in AIR6110) or that the phase of flight is sufficiently short that we need to adjust our pedal failure rate to match this phase of flight, rather than normalizing the failure rate to per-flight-hour.  

\janet{We then applied monolithic analysis with probabilistic fault hypothesis of $10^{-9}$ fault threshold. The \textit{Inadvertent braking at the wheel} properties failed. Updating the probability of occurrence for the sensor components identified above to be above the fault threshold allowed the properties to pass, showing that increasing the reliability of those components help increase the reliability of the system.}

\danielle{Given these results, we were faced with a couple of options regarding the model. In order to meet the probabilistic threshold, one option is to find a brake pedal sensor with better reliability. Given that our top level threshold is $10^{-9}$, the sensor reliability must be less than this value. This is most likely an unrealistic solution, but even if it was possible, the model still has a single point of failure. This is shown thorugh the compositional analysis. The other option is to create redundancy in the sensor component and model a voting procedure. In order for braking to occur, both (or all) sensors must agree. This would eliminate a single point of failure and would also cause the model to meet the top level probabilistic threshold. The problem this raises is the complementary property in the system which states that when braking is commanded, braking pressure is provided at the wheel. By introducing redundancy, this also introduces a tradeoff between false positives and false negatives and directly affects the property regarding commanded braking.}

\danielle{These are realistic scenarios and decisions that are faced by system and safety engineers. Providing a way to quickly and effectively run analysis on the model so that these different modes of failure can be easily seen is of great benefit.} 


%It is computationally inexpensive to run the analysis, allowing quick iterations between systems and safety engineers. In the case of the WBS, it requires between 2 and 4 minutes to run either compositional (restricting the number of faults) or monolithic (probabilistic) fault analysis on the model.

%During our analysis, we discovered that most properties were verified, but the \textit{Inadvertent braking at the wheel} properties are not resilient to a single fault nor do they meet the desired $10^{-9}$ fault threshold for probabilistic analysis. In this model, there is a single pedal position sensor for the brake pedal.  If this sensor fails, it can command braking without a pilot request.  Given the {\em counterexample} returned by the tools, it is straightforward to diagnose the fault conditions that lead to property failure.


%what did we do monolithic, what did we do compositional
%what did we use max one fault
%what did we do with what probability


%The sync and update between the safety analysis artifacts and the system architecture/analysis model continues until the system safety property is satisfied with the desired fault tolerance and failure probability achieved.
\end{comment}

\begin{comment}

Fault analysis on the top level WBS system was performed on the 11 top-level properties using two fault hypotheses.  In the first case, we allow at most one fault, and in the second we allow combinations of faults that exceed the acceptable probability for the top-level hazard defined in AIR6110.

We use this model to demonstrate the benefits of formal fault analysis and to show the scalability of our tools.  We applied both monolithic and compositional analysis.

%\janet{Janet Note: TODO: replace timing data on something about what we learned from the analysis results, and use it to feedback to the system design, and rerun the analysis. As Mats pointed out, "can we add a redundant sensor and get fault tolerance?}
%\danielle{Danielle Note: The last 2 paragraphs of this section addresses this idea a little bit. One thing we can do is add a redundant sensor to the wbs model and run it to see what happens. If we do not want to add more to the model and perform analysis, then perhaps the last paragraphs of this section is good enough.}

%For the fault-free ``nominal'' system model, monolithic analysis requires 21 seconds, whereas compositional analysis requires 1 minute and 53 seconds.  Although the compositional time is longer, each sub-problem completes in less than 5 seconds.  The additional time for compositional analysis is  due to the start-up overhead to invoke the JKind model checker many times for individual layers.  On the other hand, when examining the model under a single-fault hypothesis, compositional analysis requires 2 minutes 6 seconds, while monolithic analysis did not terminate after 60 minutes.

%For probabilistic fault hypotheses, we are currently developing a sound approach for composition with respect to the top-level fault probability, but our current tool requires monolithic analysis.  In this case, given a probabilistic fault hypothesis of $5\times 10^{-7}$, monolithic analysis requires 3 minutes 25 seconds.

During our analysis, we discovered that most properties were verified, but the \textit{Inadvertent braking at the wheel} properties are not resilient to a single fault nor do they meet the desired $10^{-9}$ fault threshold for probabilistic analysis.  In this model, there is a single pedal position sensor for the brake pedal.  If this sensor fails, it can command braking without a pilot request.  Given the {\em counterexample} returned by the tools, it is straightforward to diagnose the fault conditions that lead to property failure.

This counterexample can be used to further analyze the system design.  For our model, there are several possible reasons for failure: it could be that redundant sensors are required on the pedals (here we note that the architecture of the pedal assembly is not discussed in AIR6110) or that the phase of flight is sufficiently short that we need to adjust our pedal failure rate to match this phase of flight, rather than normalizing the failure rate to per-flight-hour.  

It is computationally inexpensive to run the analysis, allowing quick iterations between systems and safety engineers. 
In the case of the WBS, it requires between 2 and 4 minutes to run either compositional (restricting the number of faults) or monolithic (probabilistic) fault analysis on the model.
The sync and update between the safety analysis artifacts and the system architecture/analysis model continues until the system safety property is satisfied with the desired fault tolerance and failure probability achieved.
\end{comment}