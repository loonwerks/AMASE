\subsection{Safety Assessment Process}
\label{subsec:process}
\janet{J's comment: It seems to me that we could reference/mention ARP4761, FHA, PSSA etc. in a figure later in this section and briefly explain them in small-font notes that accompany the figure, but not necessary explaining them at the beginning of this section, so to avoid too much information loaded at the beginning. Therefore I commented out those descriptions for now, and revived some of our old text from the last submission.}
 %According to ARP4754, a safety critical system is a system whose safety cannot be shown solely by test, whose logic is difficult to comprehend without the aid of analytical tools, and whose failure can directly or indirectly cause significant loss of life or property~\cite{SAE:ARP4754A}. Guaranteeing safety and reliability of safety critical systems is mandatory and the process that guides it is highly standardized and controlled~\cite{RTCA:StdC,SAE:ARP4761}. ARP4761, the Guidelines and Methods for Conducting Safety Assessment Process on Civil Airborne Systems and Equipment, identifies systematic means in order to show compliance with certification standards and has been recognized by the Federal Aviation Administration (FAA) as an acceptable method to establish the assurance process.~\cite{SAE:ARP4761}. The safety assessment process is tightly coupled with the system development and verification processes and is used to show compliance with certification requirements as well as meeting a company's internal safety standards.

ARP4754A, the Guidelines for Development of Civil Aircraft and Systems~\cite{SAE:ARP4754A}, provides guidance on applying development assurance at each hierarchical level throughout the development life cycle of highly-integrated/complex aircraft systems, and has been recognized by the Federal Aviation Administration (FAA) as an acceptable method to establish the assurance process. The safety assessment process is a starting point at each hierarchical level of the development life cycle, and is tightly coupled with the system development and verification processes. It is used to show compliance with certification requirements, and for meeting a company's internal safety standards. 

%ARP4761, the Guidelines and Methods for Conducting Safety Assessment Process on Civil Airborne Systems and Equipment~\cite{SAE:ARP4761},  identifies a systematic means to show compliance. The guidelines presented in ARP4761 include industry accepted safety assessment processes (Functional Hazard Assessment (FHA), Preliminary System Safety Assessment (PSSA), and System Safety Assessment (SSA)), and safety analysis methods to conduct the safety assessment, such as Fault Tree Analysis (FTA), Failure Modes and Effect Analysis (FMEA), and Common Cause Analysis (CCA).

%The guidelines presented in ARP4761 include the industry accepted safety assessment processes Functional Hazard Analysis (FHA), Preliminary System Safety Assessment (PSSA), and System Safety Assessment (SSA). An FHA is a comprehensive examination of aircraft functions used to identify and classify failure conditions according to severity. Early in in the development of the aircraft, the aircraft level FHA is a high level assessment of aircraft functions that establishes the safety requirements that the aircraft must meet. The system level FHA considers a failure or combination of system failures that affect an aircraft function. system level FHA is an iterative analysis and becomes more defined and fixed as the system is developed. 

%Preliminary System Safety Assessment (PSSA) is a systematic evaluation of the system architecture to determine safety related design requirements of system components. This assessment is conducted at multiple stages of system, component, and hardware/software design. As the PSSA is developed, the system level FHA is updated accordingly. When the PSSA is completed and the system component safety requirements are determined, the components are implemented and a System Safety Assessment (SSA) can occur. The SSA is a comprehensive evaluation of the implemented system to show that safety requirements defined in the PSSA are in fact satisfied. 


A prerequisite of performing any safety assessment of a system design is to understand how the system is intended to work, primarily focusing on the relationship between component outputs and the overall behavior of the system. The safety engineers then use the acquired understanding to conduct safety analysis, construct the safety analysis artifacts, and compare the analysis results with established safety objectives and safety-related requirements. In practice, prior to performing the safety assessment of a system, the safety engineers are often equipped with the domain knowledge about the system, but do not necessarily have detailed knowledge of how the software functions are designed. Acquiring the required knowledge about the behavior and implementation of each software function in a system can be time-consuming. For example, in a recent project it took one of our safety engineers two days to understand how the software in a Stall Warning System was intended to work. The primary task includes connecting the signal and function flows to relate the input and output signals from end-to-end and understanding the causal effect between them. This is at least as much time as was required to construct the safety analysis artifacts and perform the safety analysis itself.

%In another instance, it took a safety engineer several months to finalize the PSSA document for a Horizontal Stabilizer Control System, because of two major design revisions requiring multiple rounds of reviews with system, hardware, and software engineers to establish complete understanding of the design details.


%For example, recently a Rockwell Collins Safety Engineer performed a series of analyses on a Stall Warning System. The primary task of the safety engineer included connecting the signal and function flows to relate the input and output signals from end-to-end and understanding the causal effect between them. Once the system interactions and behavior is clear, then the safety engineer can begin to construct the safety analysis artifacts and perform the safety analysis. In cases when major design revisions occur in system development, multiple rounds of reviews with system, hardware, and software engineers must take place in order for a safety engineer to establish complete understanding of design details. These tasks are time consuming and due to the complexity of systems can also be error prone.

Industry practitioners have come to realize the benefits and importance of
using models to assist the safety assessment process (either by augmenting the existing system design model, or by building a separate safety model), and a revision of the ARP4761 to include {\em model based safety analysis} is under way.
Capturing failure modes in models and generating safety analysis artifacts directly from models could greatly improve communication and synchronization between system designer and safety engineers, and provide the ability to more accurately analyze complex systems. 

We believe that using a single unified model to conduct both system development and safety analysis can help reduce the gap in comprehending the system behavior and transferring the knowledge between the system designers and the safety analysts. It maintains a living model that captures the current state of the system design as it moves through the system development lifecycle. It also allows all participants of the ARP4754A process to be able to communicate and review the system design using a ``single source of truth.''

A model that supports both system design and safety analysis must describe both the system design information (e.g., system architecture, functional behavior) and safety-relevant information (e.g., failure modes, failure rates).  It must do this in a way that keeps the two types of information distinguishable, yet allows them to interact with each other.

\janet{J's comment: consider clarifying/replacing/complementing the following paragraph with a figure:}
What the Safety Annex helps to provide throughout the assessment process is the ability to examine and define failure modes (the ways in which a component or system can fail) for individual system components (mechanical and digital) and then utilize behavioral propagation to determine whether these failures can cause violations of the safety requirements to occur. This can assist in understanding the system under development as well as how certain component errors contribute to both the aircraft functions and system level hazards defined in the FHA process. The nominal model and fault model are separately defined and yet interact with each  other to provide information about the system under development.

