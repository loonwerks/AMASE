\section{Conclusion}

\janet{J's comments: Not sure if detailed discussion on compositional analysis really belong to this paper, so I comment them out.}\\
\danielle{D's comment: I have uncommented the long version so that  we can see both options and then decide how we want to address this section.}

--------------------------------------------------------------------\\
\danielle{D's comment: Here is the short version.}\\
We have developed an extension to the AADL language with tool support for formal analysis of system safety properties in the presence of faults. Faulty behavior is specified as an extension of the nominal model, allowing safety analysis and system implementation to be driven from a single common model. This new Safety Annex leverages the AADL structural model and nominal behavioral specification (using the AGREE annex) to propagate faulty component behaviors without the need to add separate propagation specifications to the model. Next steps will include adding full support for compositional fault analysis, automatic generation of fault trees, and extensions to automate injection of Byzantine faults. For more details on the tool, models, and approach, see the technical report~\cite{SATechReport}. To access the tool plugin, users manual, or models, see the repository~\cite{SAGithub}. 

--------------------------------------------------------------------\\
\danielle{D's comment: And the long version.}\\
Currently, the Safety Annex only supports monolithic analysis with respect to probabilistic analysis. Compositional fault analysis is of interest for the following reasons. 
\begin{itemize}
\item Compositional analysis is scalable to large system models~\cite{NFM2012:CoGaMiWhLaLu, hilt2013:MuWhRaHe}.
\item Given fault probabilities in a system, what is the top level threshold for the system?
\item Compositional fault tree generation that reflects the fault model, the system architecture, and the system behavior.
\item What is the maximum number of active faults that a system can tolerate and what are those faults?
\end{itemize}

To this end University of Minnesota and Rockwell Collins have been focusing on recent developments in model checking in order to perform compositional fault analysis and generate these artifacts of interest. 

In this research, an extension to the AADL language has been developed with tool support for formal analysis of system safety properties in the presence of faults. Faulty behavior is specified as an extension of the nominal model, allowing safety analysis and system implementation to be driven from a single common model. This new Safety Annex leverages the AADL structural model and nominal behavioral specification (using the AGREE annex) to propagate faulty component behaviors without the need to add separate propagation specifications to the model.   Next steps will include extensions to automate injection of Byzantine faults as well as automatic generation of fault trees.  For more details on the tool, models, and approach, see the technical report~\cite{SATechReport}. To access the tool plugin, users manual, or models, see the repository~\cite{SAGithub}. 


\vspace{2 mm}
\noindent {\bf Acknowledgments.} This research was funded by NASA contract NNL16AB07T and the University of Minnesota College of Science and Engineering Graduate Fellowship.


