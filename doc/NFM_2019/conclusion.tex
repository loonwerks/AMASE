\section{Conclusion}


We have developed an extension to the AADL language with tool support for formal analysis of system safety properties in the presence of faults. Faulty behavior is specified as an extension of the nominal model, allowing safety analysis and system implementation to be driven from a single common model. This new Safety Annex leverages the AADL structural model and nominal behavioral specification (using the AGREE annex) to propagate faulty component behaviors without the need to add separate propagation specifications to the model. The support to the implicit %failure
\janet{error} propagation enables safety engineers to inject failures/faults at component level and assess the effect of behavioral propagation at the system level. It also supports explicit 
%failure
\janet{error} propagation that allows safety engineers to describe dependent faults that are not easily captured using implicit %failure
\janet{error} propagation. Next steps will include adding full support for compositional fault analysis, automatic generation of fault trees, and extensions to automate injection of Byzantine faults. For more details on the tool, models, and approach, see the technical report~\cite{SATechReport}. To access the tool plugin, users manual, or models, see the repository~\cite{SAGithub}. 

%The Safety Annex allows both implicit and explicit failure propagation. This enables safety engineers to inject failures/faults at component level and assess the effect of behavioral propagation at the system level as well as describe dependent faults that are not easily captured using implicit failure propagation. The toolset (AADL, AGREE, and Safety Annex) describes the nominal model (absence of faults) and the faulty model in a cleanly separated and yet analyzable fashion. This serves to preserve the system model for the systems engineering process and simultaneously be able to see their combined effect on the system behavior. 

\vspace{2 mm}
\noindent {\bf Acknowledgments.} This research was funded by NASA contract NNL16AB07T and the University of Minnesota College of Science and Engineering Graduate Fellowship.


