\section{Introduction}
\label{sec:intro}

%This paper describes a new methodology with tool support for model based safety analysis. It is implemented as a {\em Safety Annex} for the Architecture Analysis and Design Language (AADL). The Safety Annex provides the ability to describe faults and faulty component behaviors in AADL models. In contrast to previous AADL-based approaches, the Safety Annex leverages a formal description of the nominal system behavior to propagate faults in the system. This approach ensures consistency with the rest of the system development process and simplifies the work of safety engineers. The language for describing faults is extensible and allows safety engineers to weave various types of faults into the nominal system model. The Safety Annex supports the injection of faults into component level outputs, and the resulting behavior of the system can be analyzed using model checking through the Assume-Guarantee Reasoning Environment (AGREE).

System safety analysis techniques are well-established and are a required activity in the development of safety-critical systems. Model-based systems engineering (MBSE) methods and tools based on formal methods now permit system-level requirements to be specified and analyzed early in the development process~\cite{NFM2012:CoGaMiWhLaLu,CAV2015:BoCiGrMa}. While model-based development methods are widely used in the aerospace industry, they are only recently being applied to system safety analysis.  

%How can we leverage these model-based methods and tools to perform safety analysis based on models of the system architecture and initial functional decomposition? Can these design models be integrated into the safety analysis process to help guarantee accurate and consistent results?
%Seeking solutions to these questions are especially important as the amount of safety-critical hardware and software in various domains has drastically increased due to the demand for greater autonomy, capability, and connectedness.

In this paper, we describe a {\em Safety Annex} for the Architecture Analysis and Design Language (AADL)~\cite{FeilerModelBasedEngineering2012} that provides the ability to reason about faults and faulty component behaviors in AADL models. In the Safety Annex approach, we use formal assume-guarantee contracts to define the nominal behavior of system components. The nominal model is then verified using the Assume Guarantee Reasoning Environment (AGREE)~\cite{NFM2012:CoGaMiWhLaLu}. The Safety Annex  provides a way to weave faults into the nominal system model and analyze the behavior of the system in the presence of faults. The Safety Annex also provides a library of common fault node definitions that is customizable to the needs of system and safety engineers. Our approach adapts the work of Joshi et. al in
~\cite{Joshi05:Dasc} to the AADL modeling language, and provides a domain specific language for the kinds of analysis performed manually in previous work~\cite{Stewart17:IMBSA}.  %More information on the approach is available in~\cite{Stewart17:IMBSA}, and the tool and relevant documentation can be found at: \small \url{https://github.com/loonwerks/AMASE/}. \normalsize

%There are other tools purpose-built for safety analysis, including AltaRica~\cite{PROSVIRNOVA2013127}, smartIFlow~\cite{info8010007} and xSAP~\cite{DBLP:conf/tacas/BittnerBCCGGMMZ16}. These notations are separate from the system development model. Other tools extend existing system models, such as HiP-HOPS~\cite{CHEN201391} and the AADL Error Model Annex, Version 2 (EMV2)~\cite{EMV2}. EMV2 uses enumeration of faults in each component and explicit propagation of faulty behavior to perform safety analysis. The required propagation relationships must be manually added to the system model and can become complex, leading to potential omissions and inconsistencies.

The Safety Annex supports model checking and quantitative reasoning by attaching behavioral faults to components and then using the normal behavioral propagation and proof mechanisms built into the AGREE AADL annex. This allows users to reason about the evolution of faults over time, and produce counterexamples demonstrating how component faults lead to system failures. It can serve as the shared model to capture system design and safety-relevant information, and produce both qualitative and quantitative description of the causal relationship between faults/failures and system safety requirements.
%
Thus, the contributions of the Safety Annex and this paper are:
\begin{itemize}
\item Close integration of behavioral fault analysis into the {\em architectural design language} AADL, which allows close connection between system and safety analysis and system generation from the model,
\item support for {\em behavioral specification of faults} and their {\em implicit propagation} through behavioral relationships in the model, in contrast to existing AADL-based annexes (HiP-HOPS, EMV2) and other related toolsets (COMPASS, Cecilia, etc.),
\item additional support to capture binding relationships between hardware and software and logical and physical communications, and
\item guidance on integration into a traditional safety analysis process.
\end{itemize}
%\mike{What are our contributions?}
