\section{Example}
\label{sec:comparison_with_EMV2}

To illustrate some of the key differences between our approach and approaches that explicitly enumerate and propagate faults such as the AADL Error Model Annex, Version 2 (EMV2)~\cite{EMV2}, we use a simplified example based on the Wheel Brake System (WBS). The WBS model is described in greater detail in ~\cite{Stewart17:IMBSA} and in Section \ref{sec:case_study}. The code pieces extracted from EMV2, AGREE, and Safety Annex do not represent the complete code; they are for illustration purposes only.

As seen in Figure~\ref{fig:comparison_with_EMV2}, a simplified WBS system takes the output signal from the Pedal component, passes through the Sensor and the Braking System Control Unit (BSC) components, and generates a pressure output from the Valve component to apply to the Wheels. To assist explanation in a modeling setting, we use the general term ``fault'' to denote all component errors, hardware failures, and system faults captured by both approaches.

In Figure~\ref{fig:comparison_with_EMV2}, the fault is injected on the output of the sensor. In this case, even when the pedal is mechanically pressed the sensor will output zero to the BSCU. The behavior of the BSCU proceeds assuming that there was no mechanical pedal pressed. This will cause the top level system contract to fail: {\em pedal pressed implies brake pressure output is positive}. No fault propagation is necessary since the faulty value itself propagates through the system just as in the nominal system model. The effects of the active fault are seen through the AGREE contracts. 

\begin{figure}[h!]
	\vspace{-0.19in}
	\centering
	\includegraphics[trim=0 9 0 5,clip,width=\textwidth]{images/Comparison_with_EMV2.pdf}
	%\vspace{0.4in}
	\caption{Differences between Safety Annex and EMV2}
	\label{fig:comparison_with_EMV2}
\end{figure} 
 
In the EMV2 approach, all faults have to be explicitly propagated through each component (e.g., applying fault types on the out ports) in order for a component to have an impact on the rest of the system. The fault types are descriptive in that they cannot be processed directly by any behavioral analysis; the meaning of the fault types have to be explicitly interpreted in the form of data to be analyzed. Consequently, there is no behavioral analysis on the EMV2 model. At the system level, the tool can aggregate the fault flow and propagation information from different components to compose an overall fault flow diagram or fault tree.

In the Safety Annex approach, the faults captured as faulty behaviors (in Safety Annex) that augment the system's behavioral model (in AGREE). No separate or explicit propagation through the components is needed for the faults; all behavioral propagations (e.g., how inputs are passed to the outputs) are modeled in AGREE no matter the inputs are normal or faulty. Users can apply formal behavioral analysis on the model and verify if system level properties hold with or without the presence of faults. 

