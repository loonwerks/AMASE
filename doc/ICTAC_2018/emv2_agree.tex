\section{Example}
\label{sec:comparison_with_EMV2}

To illustrate some of the key differences between our approach and approaches that explicitly enumerate and propagate faults such as the AADL Error Model Annex, Version 2 (EMV2)~\cite{EMV2}, we use a simplified example based on the Wheel Brake System (WBS) adapted from ~\cite{WBS_EMV2_Example}.

As seen in Figure~\ref{fig:comparison_with_EMV2}, a simplified WBS system takes the output signal from the Pedal component, passes through the Sensor and the Braking System Control Unit (BSC) components, and generates a pressure output from the Valve component to apply to the Wheels. To assist explanation in a modeling setting, we use the general term "fault" to denote all component errors, hardware failures, and system faults captured by both approaches.

In an EMV2 approach, all faults have to be explicitly propagated through each component (e.g., applying fault types on the out ports) in order for a component to have an impact on the rest of the system. The fault types are descriptive in that they cannot be processed directly by any behavioral analysis; the meaning of the fault types have to be explicitly interpreted in the form of data to be analyzed. Consequently, there is no behavioral analysis on the EMV2 model. At the system level, the tool can aggregate the fault flow and propagation information from different components to compose an overall fault flow diagram or fault tree.

In the Safety Annex approach, no separate or explicit propagation is done for the faults through the components. The behavioral propagation (e.g., how inputs are passed to the outputs) are modeled in AGREE disregard normal or faulty inputs.

\begin{figure}[h!]
	\vspace{-0.19in}
	\centering
	\includegraphics[trim=0 9 0 5,clip,width=0.85\textwidth]{images/Visio-Comparison_with_EMV2.pdf}
	%\vspace{0.4in}
	\caption{Differences between Safety Annex and EMV2}
	\label{fig:comparison_with_EMV2}
\end{figure}

 


