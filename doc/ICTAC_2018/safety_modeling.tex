\section{Shared System and Safety Modeling}
\label{sec:safety_modeling}

In this section, we describe how different features of the Safety Annex can be used to describe the qualitative and quantitative causal relationship between faults and system safety requirements, and to perform safety analysis on a shared system and safety model.

\subsection{Mapping to Safety Concepts}
The following list describes how the basic safety concepts can be represented using the Safety Annex.

\begin{enumerate}
	\item \textbf{Error} Definition and how we model it
	\item \textbf{Fault} Definition and how we model it
	\item \textbf{Failure} Definition and how we model it. Non deterministic faults
	\item \textbf{Failure Mode} 
\end{enumerate}



\begin{comment}
Errors/Faults/Failures - to a safety engineer, these terms have very specific meanings.  You will see my specific comments on this topic where I located them by your Section 3.3.  If needed, we can talk about this comment after I send you my mark-ups.
In Section 3.1, you introduce the term "non-deterministic".  I am not sure how your new process can be used by the safety engineering discipline unless things are "deterministic" and therefore "repeatable".
\end{comment}



