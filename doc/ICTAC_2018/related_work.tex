\section{Related Work}
\label{sec:related_work}

Formal model based systems engineering (MBSE) methods and tools now permit system level requirements to be specified and analyzed early in the development process~\cite{QFCS15:backes,CIMATTI2015333, NFM2012:CoGaMiWhLaLu, hilt2013:MuWhRaHe}. Design models from which aircraft systems are developed can be integrated into the safety analysis process to help guarantee accurate and consistent results. There are tools that currently support reasoning about faults in architecture description languages such as SysML and AADL. These tools include the AADL Error Model Annex, Version 2 (EMV2)~\cite{EMV2} and HiP-HOPS for EAST-ADL~\cite{CHEN201391}. These approaches primarily utilize \textit{qualitative} reasoning. Faults are enumerated and the propagations through system components are explicitly described. Given many possible faults, these propagation relationships increase in complexity and understandability. Interactions are easily overlooked by analysts and thus not explicitly described. This is also the case with tools like SAML that incorporate both \textit{qualitative} and \textit{quantitative} reasoning~\cite{Gudemann:2010:FQQ:1909626.1909813}.  

Closely related is the model-based safety assessment toolset called COMPASS. COMPASS uses the SLIM language, based on AADL, for its input models. It is possible to describe both the hardware and software components of the system as well as their connections. Separate error models can be defined to describe faults, which can automatically be injected into the model. The error modeling capabilities are based from EMV2 and thus use the same qualitative reasoning for fault behaviors as described above. The COMPASS toolset cannot currently use AADL models as input. 

Similarly, AltaRica has been incorporated into Cecilia OCAS as a model based safety analysis too~\cite{BieberERTS2018}l. Safety assessment, fault tree generation, and functional verification can be performed with the aid of NuSMV model checking~\cite{symbAltaRica}. Failure propagations are specified explicitly throughout the system and failure states must be specified for all components~\cite{Bieber04safetyassessment}. Another limitation of this is that Linear Temporal Logic operators are required in some of the failure definitions. This is a downfall to the safety community/engineers who are not familiar with LTL~\cite{Bieber04safetyassessment}.


In earlier work, an approach to MBSA was demonstrated using the Simulink\textsuperscript{\textregistered} notation~\cite{Joshi05:SafeComp,Joshi05:Dasc}. In this approach, a behavioral model of system dynamics was used to reason about the effects of faults in the system. This approach allows an implicit and natural notion of fault propagation through the system. However, non-functional architectural properties were not captured as Simulink is not designed as an architecture description language. In our approach, we are applying \textit{quantitative} reasoning and implicit fault propagation to a more rich architecture language.  

There are other tools purpose-built for safety analysis, including AltaRica~\cite{PROSVIRNOVA2013127}, smartIFlow~\cite{info8010007} and xSAP~\cite{DBLP:conf/tacas/BittnerBCCGGMMZ16}. These notations are separate from the system development model, but allow similar behavioral analysis to our tools.  Integration of MBSA into safety analysis process is described by Bozzano and Villafiorita~\cite{Bozzano:2010:DSA:1951720}.

Formal verification tools based on model checking have been used to automate the generation of safety artifacts~\cite{symbAltaRica,10.1007/978-3-540-75596-8-13, DBLP:conf/tacas/BittnerBCCGGMMZ16}. This approach has limitations in terms of scalability and readability of the fault trees generated. Work has been done towards mitigating these limitations by the scalable generation of readable fault trees~\cite{10.1007/978-3-319-11936-6-7}.
