\section{Related Work}
\label{sec:related_work}

%Address NFM review comments

Formal model based systems engineering (MBSE) methods and tools now permit system level requirements to be specified and analyzed early in the development process~\cite{QFCS15:backes,CIMATTI2015333, NFM2012:CoGaMiWhLaLu, hilt2013:MuWhRaHe}. Design models from which aircraft systems are developed can be integrated into the safety analysis process to help guarantee accurate and consistent results. There are tools that currently support reasoning about faults in architecture description languages such as SysML and AADL. These tools include the AADL Error Model Annex, Version 2 (EMV2)~\cite{EMV2} and HiP-HOPS for EAST-ADL~\cite{CHEN201391}. These approaches primarily utilize \textit{qualitative} reasoning. Faults are enumerated and the propagations through system components are explicitly described. Given many possible faults, these propagation relationships increase in complexity and understandability. Interactions are easily overlooked by analysts and thus not explicitly described. This is also the case with tools like SAML that incorporate both \textit{qualitative} and \textit{quantitative} reasoning~\cite{Gudemann:2010:FQQ:1909626.1909813}.  


%original version:
%In earlier work, an approach to MBSA was demonstrated using the Simulink notation~\cite{Joshi05:SafeComp,Joshi05:Dasc,NasaRep:MBSA-Aug05} . In this approach, a behavioral model of system dynamics was used to reason about the effects of faults in the system. We believe this approach allows an implicit and natural notion of fault propagation through the system. Since Simulink is not an architecture description language, notions such as hardware devices and non-functional aspects cannot be captured in system models. Using this idea of \textit{quantitative} reasoning and implicit fault propagation, we wish to apply this to a more rich architecture language.

%updated version:
%<<<<<<< HEAD
In earlier work, an approach to MBSA was demonstrated using the Simulink\textsuperscript{\textregistered} notation~\cite{Joshi05:SafeComp,Joshi05:Dasc}. In this approach, a behavioral model of system dynamics was used to reason about the effects of faults in the system. This approach allows an implicit and natural notion of fault propagation through the system. However, non-functional architectural properties were not captured as Simulink is not designed as an architecture description language. In our approach, we are applying \textit{quantitative} reasoning and implicit fault propagation to a more rich architecture language.  

There are other tools purpose-built for safety analysis, including AltaRica~\cite{PROSVIRNOVA2013127}, smartIFlow~\cite{info8010007} and xSAP~\cite{DBLP:conf/tacas/BittnerBCCGGMMZ16}. These notations are separate from the system development model, but allow similar behavioral analysis to our tools.  Integration of MBSA into safety analysis process is described by Bozzano and Villafiorita~\cite{Bozzano:2010:DSA:1951720}.
%=======
%In earlier work, an approach to MBSA was demonstrated using the Simulink\textsuperscript{\textregistered} notation~\cite{Joshi05:SafeComp,Joshi05:Dasc,NasaRep:MBSA-Aug05}. In this approach, a behavioral model of system dynamics was used to reason about the effects of faults in the system. This approach allows an implicit and natural notion of fault propagation through the system. However, non-functional architectural properties were not captured as Simulink is not designed as an architecture description language. In our approach, we are applying \textit{quantitative} reasoning and implicit fault propagation to a more rich architecture language.
%>>>>>>> 80fb5bd8b3b71f60ab4a407a2ee73f016effe2c2

Formal verification tools based on model checking have been used to automate the generation of safety artifacts% and used in safety critical system certification
~\cite{symbAltaRica,10.1007/978-3-540-75596-8-13, DBLP:conf/tacas/BittnerBCCGGMMZ16}. This approach has limitations in terms of scalability and readability of the fault trees generated. Work has been done towards mitigating these limitations by the scalable generation of readable fault trees~\cite{10.1007/978-3-319-11936-6-7}.

%Moved the following to case study
%The Wheel Brake System (WBS) described in ARP4761~\cite{SAE:ARP4761} has been used in the past as a case study for safety analysis, formal verification, and contract based design~\cite{DBLP:conf/cav/BozzanoCPJKPRT15, 10.1007/978-3-319-11936-6-7, CAV2015:BoCiGrMa, Stewart17:IMBSA, propBasedProofSys, Joshi05:SafeComp, NasaRep:MBSA-Aug05} The preliminary work for the safety annex used a simplified model of the WBS~\cite{Stewart17:IMBSA}. In order to show scalability and compare results with other studies, an AADL version of the WBS was designed based off of arch4wbs NuSMV model described in previous work~\cite{DBLP:conf/cav/BozzanoCPJKPRT15}. This model was chosen due to the number of subcomponents in the system and the complexity of behavior captured in the NuSMV model.

