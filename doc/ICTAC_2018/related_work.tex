\section{Related Work}
\label{sec:related_work}

Formal model based systems engineering (MBSE) methods and tools now permit system level requirements to be specified and analyzed early in the development process~\cite{QFCS15:backes,CIMATTI2015333, NFM2012:CoGaMiWhLaLu, hilt2013:MuWhRaHe}. Design models from which aircraft systems are developed can be integrated into the safety analysis process to help guarantee accurate and consistent results. Integration of MBSA into safety analysis process is described by Bozzano and Villafiorita~\cite{Bozzano:2010:DSA:1951720}. There are tools that currently support reasoning about faults in architecture description languages such as SysML and AADL. We provide here a brief overview of the most relevant safety analysis tools. 

Tools such as the AADL Error Model Annex, Version 2 (EMV2)~\cite{EMV2} and HiP-HOPS for EAST-ADL~\cite{CHEN201391} primarily utilize qualitative reasoning. Faults are enumerated and the propagations through system components are explicitly described. Given many possible faults, these propagation relationships increase in complexity and understandability. Interactions are easily overlooked by analysts and thus not explicitly described. In our approach, faults are injected into the system and behaviorally propagated through the use of assume-guarantee statements in AGREE. This avoids the difficulties inherent with explicit fault enumeration and propagation. 

% Say something about behavioral propagation here ---------------------------------------------------

SmartIFlow~\cite{info8010007} is a purpose-built safety analysis tool that describes components and their interactions using finite state machines and events. Verification is done through an explicit state model checker which returns sets of counterexamples for safety requirements in the face of failures. The mechanism to keep the search space size under control during model checking relies on expert knowledge from engineers. This limits the number of failures and removes the possibility of certain failure conditions. Due to this drawback, scalability to industrial sized systems is difficult. As a contrast, the safety annex described in this research is not a standalone model, but is made to be incorporated into the system safety assessment process as described in section 2.1. As shown in section 4, scalability to industrial sized systems is promising and there are no requirements of safety engineers to limit the number of failures or states of a system model. 

Another approach has been introduced by G{\"u}demann et. al.~\cite{Gudemann:2010:FQQ:1909626.1909813}. System models are constructed in SAML (Safety Analysis and Modeling Language) which are then used for both qualitative and quantitative analyses. It allows for the combination of discrete probability distributions and non-determinism. The SAML model can be automatically imported into several analysis tools like NuSMV~\cite{Cimatti2000}, PRISM (Probabilistic Symbolic Model Checker)~\cite{CAV2011:KwNoPa}, or the MRMC probabilistic model checker~\cite{Katoen:2005:MRM:1114692.1115230}. The focus of SAML is to provide modeling support for safety analysis. This is a different focus than what the safety annex provides through AADL. Given that AADL is an SAE International standard modeling language, the goal of AADL is system engineering and the development of performance-critical, embedded, real-time systems. What we accomplish is to provide safety engineers with a way to incorporate safety engineering into already existing model development practices in industry. 

In earlier work, an approach to MBSA was demonstrated using the Simulink\textsuperscript{\textregistered} notation~\cite{Joshi05:SafeComp,Joshi05:Dasc}. In this approach, a behavioral model of system dynamics was used to reason about the effects of faults in the system. This approach allows an implicit and natural notion of fault propagation through the system. However, non-functional architectural properties were not captured as Simulink is not designed as an architecture description language. In our approach, we are applying quantitative reasoning and implicit fault propagation to a more rich architecture language.

Similarly, AltaRica~\cite{PROSVIRNOVA2013127} has been incorporated into Cecilia OCAS as a model based safety analysis tool~\cite{BieberERTS2018}. Safety assessment, fault tree generation, and functional verification can be performed with the aid of NuSMV model checking~\cite{symbAltaRica}. Failure states are defined throughout the system and flow variables are updated through the use of assertions~\cite{Bieber04safetyassessment}. A limitation of this is that Linear Temporal Logic operators are required in some of the failure definitions. This is a downfall to the safety community/engineers who are not familiar with LTL~\cite{Bieber04safetyassessment}. 

Closely related to our work is the model-based safety assessment toolset called COMPASS (Correctness, Modeling project and Performance of Aerospace Systems)~\cite{10.1007/978-3-642-04468-7_15}. COMPASS uses the SLIM language which is based on AADL, for its input models. The SLIM (System Level Integrated Modeling Language) language was developed by the COMPASS project for modeling hardware and software systems for safety-related tasks~\cite{5185388, criticalembeddedsystems}. The nominal system model and the error model are developed separately and then transformed into an extended system model. This extended model is automatically translated into input models for the NuSMV model checker~\cite{Cimatti2000, NuSMV}, MRMC (Markov Reward Model Checker)~\cite{Katoen:2005:MRM:1114692.1115230, MRMC}, and RAT (Requirements Analysis Tool)~\cite{RAT}. The safety analysis tool xSAP~\cite{DBLP:conf/tacas/BittnerBCCGGMMZ16} can be invoked in order to generate safety analysis artifacts such as fault trees and FMEA tables~\cite{compass30toolset}. While it is clear that behavioral contracts can be specified in the SLIM model through the use of assume-guarantee statements, the focus of the tool and examples provided is on explicit fault propagation much like EMV2 AADL error annex~\cite{COMPASSusersguide}. Our approach is different in that the focus is not on explicit fault propagation, but instead leveraging the nominal model behavior in order to view the system behavior in the presence of faults. 

Formal verification tools based on model checking have been used to automate the generation of safety artifacts~\cite{symbAltaRica,10.1007/978-3-540-75596-8-13, DBLP:conf/tacas/BittnerBCCGGMMZ16}. This approach has limitations in terms of scalability and readability of the fault trees generated. Work has been done towards mitigating these limitations by the scalable generation of readable fault trees~\cite{10.1007/978-3-319-11936-6-7}.

