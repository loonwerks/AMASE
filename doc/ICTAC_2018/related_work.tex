\section{Related Work}
\label{sec:related_work}

A model-based approach for safety analysis was proposed by Joshi et. al in \cite{Joshi05:Dasc, Joshi05:SafeComp, Joshi07:Hase}.  In this approach, a safety analysis system model (SASM) is the central artifact in the safety analysis process, and traditional safety analysis artifacts, such as fault trees, are automatically generated by tools that analyze the SASM.

The contents and structure of the SASM differ significantly across different conceptions of MBSA.  We can draw distinctions between approaches along several different axes.  The first is whether they propagate faults explicitly through user-defined propagations, which we call {\em failure logic modeling} (FLM) or through existing behavioral modeling, which we call {\em failure effect modeling} (FEM).  The next is whether models and notations are {\em purpose-built} for safety analysis vs. those that extend {\em existing system models} (ESM).

For FEM approaches, there are several additional dimensions.  One dimension involves whether {\em causal} or {\em non-causal} models are allowed.  Non-causal models allow simultaneous (in time) bi-directional failure propagations, which allow more natural expression of some failure types (e.g. reverse flow within segments of a pipe), but are more difficult to analyze.  A final dimension involves whether analysis is {\em compositional} across layers of hierarchically-composed systems or {\em monolithic}.  Our approach is an extension of AADL (ESM), causal, compositional, mixed FLM/FEM approach.

%We believe this is in a unique area of the trade space compared to other state-of-the-art MBSA approaches.

%Formal model based systems engineering (MBSE) methods and tools now permit system level requirements to be specified and analyzed early in the development process~\cite{QFCS15:backes,CIMATTI2015333, NFM2012:CoGaMiWhLaLu, hilt2013:MuWhRaHe}. Design models from which aircraft systems are developed can be integrated into the safety analysis process to help guarantee accurate and consistent results. Integration of MBSA into safety analysis process is described by Bozzano and Villafiorita~\cite{Bozzano:2010:DSA:1951720}. There are tools that currently support reasoning about faults in architecture description languages such as SysML and AADL. We provide here a brief overview of the most relevant safety analysis tools.

Tools such as the AADL Error Model Annex, Version 2 (EMV2)~\cite{EMV2} and HiP-HOPS for EAST-ADL~\cite{CHEN201391} are {\em FLM}-based {\em ESM} approaches.  As previously discussed, given many possible faults, these propagation relationships require substantial user effort and become more complex.  In addition, it becomes the analyst's responsibility to determine whether faults can propagate; missing propagations lead to unsound analyses.  In our Safety Annex, propagations occur through system behaviors (defined by the nominal contracts) with no additional user effort.

%In analyzing an EMV2 model of the WBS, we found missing feedback loops between the wheels and BSCU
%Interactions are easily overlooked by analysts and thus not explicitly described. In our approach, faults are injected into the system and behaviorally propagated through the use of assume-guarantee statements in AGREE. This avoids the difficulties inherent with explicit fault enumeration and propagation.

Closely related to our work is the model-based safety assessment toolset called COMPASS (Correctness, Modeling project and Performance of Aerospace Systems)~\cite{10.1007/978-3-642-04468-7_15}.  COMPASS is a mixed {\em FLM/FEM}-based, {\em causal} {\em compositional} tool suite that uses the SLIM language, which is based on a subset of AADL, for its input models~\cite{5185388, criticalembeddedsystems}. In SLIM, a nominal system model and the error model are developed separately and then transformed into an extended system model.  This extended model is automatically translated into input models for the NuSMV model checker~\cite{Cimatti2000, NuSMV}, MRMC (Markov Reward Model Checker)~\cite{Katoen:2005:MRM:1114692.1115230, MRMC}, and RAT (Requirements Analysis Tool)~\cite{RAT}. The safety analysis tool xSAP~\cite{DBLP:conf/tacas/BittnerBCCGGMMZ16} can be invoked in order to generate safety analysis artifacts such as fault trees and FMEA tables~\cite{compass30toolset}.  COMPASS is an impressive tool suite, but some of the features that make AADL suitable for SW/HW architecture specification: event and event-data ports, threads, and processes, appear to be missing, which means that the SLIM language may not be suitable as a general system design notation (ESM).

%While it is clear that behavioral contracts can be specified in the SLIM model through the use of assume-guarantee statements, the focus of the tool and examples provided is on explicit fault propagation much like EMV2 AADL error annex~\cite{COMPASSusersguide}. Our approach is different in that the focus is not on explicit fault propagation, but instead leveraging the nominal model behavior in order to view the system behavior in the presence of faults.


% Say something about behavioral propagation here ---------------------------------------------------

SmartIFlow~\cite{info8010007} is a {\em FEM}-based, {\em purpose-built}, {\em monolithic} {\em non-causal} safety analysis tool that describes components and their interactions using finite state machines and events. Verification is done through an explicit state model checker which returns sets of counterexamples for safety requirements in the presence of failures.  SmartIFlow allows {\em non-causal} models containing simultaneous (in time) bi-directional failure propagations.  On the other hand, the tools do not yet appear to scale to industrial-sized problems, as mentioned by the authors~\cite{info8010007}: ``As current experience is based on models with limited size, there is still a long way to go to make this approach ready for application in an industrial context''.


The Safety Analysis and Modeling Language (SAML)~\cite{Gudemann:2010:FQQ:1909626.1909813} is a {\em FEM}-based, {\em purpose-built}, {\em monolithic} {\em causal} safety analysis language.  System models constructed in SAML can be used used for both qualitative and quantitative analyses. It allows for the combination of discrete probability distributions and non-determinism. The SAML model can be automatically imported into several analysis tools like NuSMV~\cite{Cimatti2000}, PRISM (Probabilistic Symbolic Model Checker)~\cite{CAV2011:KwNoPa}, or the MRMC probabilistic model checker~\cite{Katoen:2005:MRM:1114692.1115230}. 
%The focus of SAML is to provide modeling support for safety analysis. This is a different focus than what the safety annex provides through AADL.

%Given that AADL is an SAE International standard modeling language, the goal of AADL is system engineering and the development of performance-critical, embedded, real-time systems. What we accomplish is to provide safety engineers with a way to incorporate safety engineering into already existing model development practices in industry.

%In earlier work, an approach to MBSA was demonstrated using the Simulink\textsuperscript{\textregistered} notation~\cite{Joshi05:SafeComp,Joshi05:Dasc}. In this approach, a behavioral model of system dynamics was used to reason about the effects of faults in the system. This approach allows an implicit and natural notion of fault propagation through the system. However, non-functional architectural properties were not captured as Simulink is not designed as an architecture description language. In our approach, we are applying quantitative reasoning and implicit fault propagation to a more rich architecture language.

AltaRica~\cite{PROSVIRNOVA2013127,BieberERTS2018} is a {\em FEM}-based, {\em purpose-built}, {\em monolithic} safety analysis language with several dialects.  There is one dialect of AltaRica which use dataflow ({\em causal}) semantics, while the most recent language update (AltaRica 3.0) uses non-causal semantics.  The dataflow dialect has substantial tool support, including the commercial Cecilia OCAS tool from Dassault.  For this dialect the Safety assessment, fault tree generation, and functional verification can be performed with the aid of NuSMV model checking~\cite{symbAltaRica}. Failure states are defined throughout the system and flow variables are updated through the use of assertions~\cite{Bieber04safetyassessment}.  AltaRica 3.0 has support for simulation and Markov model generation through the OpenAltaRica (www.openaltarica.fr) tool suite.

Formal verification tools based on model checking have been used to automate the generation of safety artifacts~\cite{symbAltaRica,10.1007/978-3-540-75596-8-13, DBLP:conf/tacas/BittnerBCCGGMMZ16}. This approach has limitations in terms of scalability and readability of the fault trees generated. Work has been done towards mitigating these limitations by the scalable generation of readable fault trees~\cite{10.1007/978-3-319-11936-6-7}.

