\subsection{Modeling Language for System Design}
\label{subsec:aadl-agree}
%Talk about AADL, AGREE, and why safety annex
%Pull AADL/AGREE background from previous papers to support points in the safety process
%Following the motivation/discussion in the process subsection, talk about why we choose to extend AGREE in safety annex, instead of using a separate safety model, or a semi-separate safety model like EMV2.
We are using the Architectural Analysis and Design Language (AADL)~\cite{FeilerModelBasedEngineering2012} to construct system architecture models.  AADL is an SAE International standard~\cite{AADL_Standard} that defines a language and provides a unifying framework for describing the system architecture for ``performance-critical, embedded, real-time systems''~\cite{AADL_Standard}. From its conception, AADL has been designed for the design and construction of avionics systems.  Rather than being merely descriptive, AADL models can be made specific enough to support system-level code generation.  Thus, results from analyses conducted, including the new safety analysis proposed here, correspond to the system that will be built from the model.  

An AADL model describes a system in terms of a hierarchy of components and their interconnections, where each component can either represent a logical entity (e.g., application software functions, data) or a physical entity (e.g., buses, processors). An AADL model can be extended with language annexes to provide a richer set of modeling elements for various system design and analysis needs (e.g., performance-related characteristics, configuration settings, dynamic behaviors). The language definition is sufficiently rigorous to support formal analysis tools that allow for early phase error/fault detection.

The Assume Guarantee Reasoning
Environment (AGREE)~\cite{NFM2012:CoGaMiWhLaLu} is a tool for formal analysis of behaviors in AADL models.  It is implemented as an AADL annex and annotates AADL components with formal behavioral contracts. Each component's contracts can include assumptions and guarantees about the component's inputs and outputs respectively, as well as predicates describing how the state of the component evolves over time.

AGREE translates an AADL model and the behavioral contracts into Lustre~\cite{Halbwachs91:IEEE} and then queries a user-selected
model checker to conduct the back-end analysis. The analysis %is
can be performed compositionally following the architecture hierarchy such that analysis at a higher level is based on the components at the next lower level.  When compared to monolithic analysis (i.e., analysis of the flattened model composed of all components), the compositional approach allows the analysis to scale to much larger systems. 

%In the avionics context, the software functions/applications, the hardware equipment, and the system that is composed of their integration can all be represented as components connected to/composed of/bind to other components in a hierarchical AADL model. AGREE contracts can be used to capture the functional requirements at each level of the hierarchy. Once the model has been reviewed and the requirements captured have been validated, the back-end analysis can be conducted to verify if each level of the model implements its higher level requirements correctly.

%AADL with the AGREE extension serves as a good candidate as the modeling language for describing the system design aspects of a shared system design and safety analysis model. 
In our prior work~\cite{Stewart17:IMBSA}, we added an initial failure effect modeling capability to the AADL/AGREE language and tool set.  We are continuing this work so that our tools and methodology can be used to satisfy system safety objectives of ARP4754A and ARP4761.  

\begin{comment}
In particular, our goals are to:

\begin{itemize}
	\item Provide a comprehensive, qualitative description of the causal relationship between basic failure events and system level safety requirements.
	\item Provide an accurate, quantitative description of the contribution relationship between failure rates of the fault tree basic events and numerical probability requirements at the system level.
\end{itemize}
\end{comment}
%The remainder of the paper describes our approach towards both of the goals.



