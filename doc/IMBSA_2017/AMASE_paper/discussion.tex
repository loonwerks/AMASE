\section{Discussion}
We have used the WBS model as a vehicle to experiment with different modeling and fault representation ideas, and to get a feel for the scalability of our approach.
%
We started from the reference AADL model~\cite{SEI:AADL} to attempt to contrast our FEM approach using AGREE contracts vs. the FLM-based approach that was already part of this model.  Part of this was driven by curiosity as to whether important faults might be caught by one approach and missed by the other, and to contrast the two styles of analysis.

During the process of defining and injecting faults, subtle issues of the system structure and behavioral interactions became much clearer.  The idea that the system must use the green side until a failure occurs was unexpected.  In addition, the extensions to the model were driven by the counterexamples returned by the tools.   The approach quickly and precisely provided feedback towards aspects of the system that were not robust to failure.  The researcher who produced the model (Stewart) was not involved in earlier MBSA work and had no prior exposure to the WBS model and yet was able to relatively quickly construct a fault-tolerant model.  The fact that these holes in the reference model perhaps means that the behavioral approach can be better at drawing attention to certain kinds of failures.

On the other hand, the utility of the safety analysis is driven by the ``goodness'' of the properties.  Our one example property is clearly insufficient: for example, it is not possible to detect faults related to over-pressurization or misapplication of the brakes when no braking is commanded.  Of course, any complete analysis should have properties related to each hazardous condition.  The approach is foundationally a top-down analysis (like fault trees) rather than a bottom up approach (like a FMEA / FMECA).  In addition, if properties are mis-specified, or the system dynamics are incorrectly modeled, then properties may verify even when systems are unsafe.  The explicit propagation approach of the FLM techniques force the analyst to consider each fault interaction.  This too is a double-edged sword: when examining some of the fault propagations in the reference model, we disagreed with some of the choices made, particularly with respect to the selector valve.  For example, if no select alternate commands are received from the BSCU, then both the green and blue lines emit a {\em No\_Service} failure.

In terms of scalability, the analysis time for counterexamples was on the order of 1-2 seconds, and the time for proofs was around 4 seconds, even after annotating the model with several different failures.  From earlier experience applying compositional verification with the AGREE tools (e.g.,~\cite{QFCS15:backes,hilt2013}), we believe that the analysis will scale well to reasonably large models with many component failures, but this will be determined in future work.

The analysis in this paper involved hand-annotating the models with failure nodes.  This process is both schematic and straightforward: we define the AGREE contracts over internal {\em nominal output variables} and then define the actual outputs using the nominal output variables as inputs to the fault nodes like those in Figure~\ref{fig:failureNodes}.   We are currently in the process of defining a fault integration language which will eliminate the need for hand-annotation.  Some aspects of the Error Annex could be directly relevant: the state machines describing leaf-level faults could easily be compiled into behavioral state machines that determine when faults occur.  On the other hand, in a behavioral approach we need to be able to bring in additional quantities (inputs, parameters) to instantiate behavioral faults, and the two approaches have very different notions of propagation.

The xSAP tool~\cite{DBLP:conf/tacas/BittnerBCCGGMMZ16} has an elegant extension language that allows for fault definition, selection between multiple faults for a component, and ``global'' dependent faults that can affect multiple components.  The authors have used this support to construct a sophisticated analysis model for the WBS~\cite{DBLP:conf/cav/BozzanoCPJKPRT15}.  However, some useful aspects of fault modeling, such as global faults that are driven by the state of the model, appear to be hard to construct.  For example, a pipe-burst failure can be seen as a global failure because it may cause unconnected components within the model to fail, so can be represented as having a certain probability.  On the other hand, the likelihood of failure in the real system is driven by the number of currently pressurized pipes in the system, which appears to be hard to define.  We hope to allow for such conditional and model-driven failures in our fault definition language.



