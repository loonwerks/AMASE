\subsection{Example}
Let $P$ be our top level safety property: the ``good'' behavior we want to happen.\\ 

$P =  $ (pressure $>$ threshold) $\implies$ shut down command, i.e. shut down when we should. \\

Our model elements are: $F = \{f1, f2, f3\}$ corresponding to: \\
$f1 =$ sensor 1 fault (stuck at low)\\
$f2 =$ sensor 2 fault (stuck at low)\\
$f3 =$ sensor 3 fault (stuck at low)\\

For the \textit{no voting} implementation, each sensor can have a stuck at low fault and the system shuts down when one of the sensors indicates high pressure. The constraint system for this example corresponds to : $C = \{\neg f1, \neg f2, \neg f3, \neg P\}$. This constraint system is given to the SAT solver. Assuming that the nominal model holds ($P$ and model elements are satisfied), we get an UNSAT result with this constraint system. The SAT solver provides all counterexamples in the form of IVCs which are our Minimal Unsatisfiable Subsets (MUSs). Since these are all of the sets that show $\neg P$ is UNSAT, they also are the sets that prove $P$. For example $IVC_1 = \{\neg f1\}$: if sensor 1 fault is inactive, we can prove P : this is the minimal explanation of infeasibility with respect to $C$.\\

In this case, the IVCs generated are: \\
$IVC_1 = \{\neg f1\}$, sensor 1 fault (stuck at low) = false,\\
$IVC_2 = \{\neg f2\}$, sensor 2 fault (stuck at low) = false,\\
$IVC_3 = \{\neg f3\}$, sensor 3 fault (stuck at low) = false.\\

MUSes are equivalent to the IVCs. \\
$MUS_1 = \{\neg f1\}$,\\
$MUS_2 = \{\neg f2\}$,\\
$MUS_3 = \{\neg f3\}$.\\

Now we look at Maximal Satisfiable Subsets (MSS). MSSs are the sets for which we have the maximal number of elements that will prove our constraint system. If we add anything to these sets, it becomes UNSAT. Since our original constraint system is of the form $C = \{\neg f1, \neg f2, \neg f3, \neg P\}$, we have:\\
MUS: $P$ is SAT $\iff \neg P$ is UNSAT\\
MSS: $P$ is UNSAT $\iff \neg P$ is SAT. \\

The Minimal Correction Sets (MCSs) are the complement of MSS relative to constraint system $C$. The MCSs describes the infeasibility of the system and works as a ``correction'' set to the problem. The MCSs describe the constraints that when removed from the constraint system, provide a satisfiable system. Furthermore, any strict subset of the MCSs, when removed from the constraint system, will provide an unsatisfiable system. It seems that we do not need to actually find the $MSSs$ in order to get their complement because of what is called a \textit{hitting set}. Intuitively, a minimal hitting set has the minimal number of elements in it such that every set in that collection has something in common with the set its ``hitting.'' Every MCS is a minimal hitting set of its MUSes.\\

For us in this example, it is: MCS $ = \{\neg f1, \neg f2, \neg f3\}$. This is the hitting set because if we take the intersection of every MUS with the MCS, it is nonempty and it is the minimal such set for which this is true. When the constraints on these model elements are removed, we get a constraint system that is satisfiable with regard to $\neg P$. Thus, this is the minimal set of elements for which the top level property occurs. What this means from the models perspective is that all three faults together cause the top level property to fail. Thus, when all three sensors have a fault which causes them to report low temp, we do not shut down when we should. Intuitively, it makes sense that this is the Minimal Cut Set because it is the minimal description of why the top level property fails. \\






































