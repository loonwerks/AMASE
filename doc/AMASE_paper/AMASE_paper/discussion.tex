\section{Discussion}
Some of the goals of extending existing modeling languages in this research is to better describe failure conditions and to better understand interactions of system components in the face of faults. In the process of defining and injecting faults, some subtle issues of the system became far more clear. One of these was the necessity of a non-symmetric system in terms of the Alternate and the Normal hydraulic pressure. As the system is described, the Alternate system requires a separate functionality by means of the Selector and Accumulator valve. Without this separation between the Normal and Alternate systems, one permanent fault will cause failure in the system (namely in either of the skid components). Another failure condition that wasn't initially obvious was the initialization of the system to be in Normal mode. If the initial state of the system is Alternate mode, it cannot recover from a failure in the blue\_skid component or the accumulator\_valve component. 

These failure conditions did not appear obvious by studying the organization of components or proving the nominal system requirements, but instead became clear when utilizing some of these compositional analysis techniques. 

Another goal was to better understand the interactions of system components in the face of failures. This understanding was improved during the process of our initial permanent fault introduction to the system. When a fault was injected into the skid components whose output goes directly to the wheel, the system had no time to select otherwise. The interaction between the BSCU and skid components (or wheel) was nonexistent. In order to rectify this lack of communication, further interaction was required between the wheel and BSCU components. By allowing the system one more step of computation, the system is able to respond to these later failures and correct itself accordingly. This realization came while observing this interaction between components.

Once these issues were addressed, we could successfully prove the top level contract in the case of one permanent fault in any of the high level components. When any one of these failures are present, the system can successfully respond to the failure and still provide pressure to the brakes. 




