\section{Introduction}

System safety analysis techniques are well established and are a required activity in the development of commercial aircraft and safety-critical ground systems. However, these techniques are based on informal system descriptions that are separate from the actual system design artifacts, and are highly dependent on the skill and intuition of a safety analyst. The lack of precise models of the system architecture and its failure modes often forces safety analysts to devote significant effort to gathering architectural details about the system behavior from multiple sources and embedding this information in safety artifacts, such as fault trees.

While model-based development (MBD) methods are widely used in the aerospace industry, they are generally disconnected from the safety analysis process itself. Model-based systems engineering (MBSE) methods and tools \mike{Add citations: us, Trento folks, UPenn folks} now permit system-level requirements to be specified and analyzed early in the development process. These tools can also be used to perform safety analysis based on the system architecture and initial functional decomposition. Design models from which aircraft systems are developed can be integrated into the safety analysis process to help guarantee accurate and consistent results. This integration is especially important as the amount of safety-critical hardware and software in domains such as aerospace, automotive, and medical devices has dramatically increased due to desire for greater autonomy, capability, and connectedness.

Architecture description languages, such as SysML~\cite{SysML} and the Architecture Analysis and Design Language (AADL)~\cite{AADL} are appropriate for capturing system safety information.  There are several tools that currently support reasoning about faults in architecture description languages, such as the AADL error annex~\cite{Larson:2013:IAE:2527269.2527271} and HiP-HOPS for EAST-ADL~\cite{CHEN201391}.  However, these approaches primarily use {\em qualitative} reasoning, in which faults are enumerated and their propagations through system components must be explicitly described.  Given many possible faults, these propagation relationships become complex and it is also difficult to describe temporal properties of faults that evolve over time (e.g., leaky valve or slow divergence of sensor values).

In earlier work, Rockwell Collins and the University of Minnesota developed and demonstrated an approach to model-based safety analysis (MBSA)\mike{Add citations} \danielle{A Proposal for Model-Based Safety Analysis? (Joshi05:Dasc)} using the Simulink notation~\mike{Add Simulink citation} \danielle{Do you mean the MBSA of simulink models using SCADE? (Joshi05:SafeComp)}.  In this approach, a behavioral model of (sometimes simplified) system dynamics was used to reason about the effect of faults.  We believe that this approach allows a natural and implicit notion of fault propagation through the changes in pressure, mode, etc. that describe the system's behavior.  Unlike qualitative approaches, this approach allows uniform reasoning about system functionality and failure behavior, and can describe complex temporal fault behaviors.  On the other hand, Simulink is not an architecture description language, and several system engineering aspects, such as hardware devices and non-functional aspects cannot be easily captured in models.

\iffalse
Over the last five years, several research groups have focused on formal reasoning at the system architecture level, resulting in MBSE tools that incorporate assume-guarantee compositional reasoning techniques~\cite{Trento and Rockwell and UMN}.  These tools allow behavioral reasoning about complex system models, but with substantially greater scalability than previous approaches.
\fi 

This paper describes our initial work towards a behavioral approach to MBSA using AADL.  Using assume-guarantee compositional reasoning techniques, we hope to support system safety objectives of ARP4754A and ARP4761.  To make these capabilities accessible to practicing safety engineers, it is necessary to extend modeling notations to better describe failure conditions, interactions, and mitigations, and provide improvements to compositional reasoning approaches focused on the specific needs of system safety analysis.  These extensions involve creating models of fault effects and weaving them into the analysis process.  To a large extent, our work has been an adaptation of the work of Joshi et. al to the AADL modeling language.

To benchmark our work, we have developed architectural models for the Wheel Braking System model in SAE AIR6110 to evaluate the effectiveness and practicality of our approach.  Starting from a reference AADL model constructed by the SEI instrumented with qualitative safety analysis information~\cite{}, we add behavioral contracts to the model.  In so doing, we determine that there are errors related to (manually constructed) propagations across components, and also an architectural decomposition that allows for single points of failure.  We use our analyses to find and fix these errors.

The rest of the paper is organized as follows...



