\section{Introduction}

System safety analysis techniques are well established and are a required activity in the development of commercial aircraft and safety-critical ground systems. However, these techniques are based on informal system descriptions that are separate from the actual system design artifacts, and are highly dependent on the skill and intuition of a safety analyst. The lack of precise models of the system architecture and its failure modes often forces safety analysts to devote significant effort to gathering architectural details about the system behavior from multiple sources and embedding this information in safety artifacts, such as fault trees.

While model-based development (MBD) methods are widely used in the aerospace industry, they are generally disconnected from the safety analysis process itself. Model-based systems engineering (MBSE) methods and tools now permit system-level requirements to be specified and analyzed early in the development process. These tools can also be used to perform safety analysis based on the system architecture and initial functional decomposition. Design models from which aircraft systems are developed can be integrated into the safety analysis process to help guarantee accurate and consistent results. This is especially critical as both airborne and ground-based software for aircraft operating in the National Airspace System (NAS) continues to grow in complexity.

Previously, Rockwell Collins and the University of Minnesota developed and demonstrated an initial approach to model-based safety analysis (MBSA) (CITE THIS). New MBSE tools that incorporate assume-guarantee compositional reasoning techniques provide the basis for greatly improving earlier approaches to safety analysis, and can be used to ensure model consistency, correctness of assumptions, and better scalability.

Using our system architecture modeling and analysis tools based on the Architecture Analysis and Design Language (AADL) as an exemplar, we extend existing analysis methods to support system safety objectives of ARP4754A and ARP4761. This includes extensions to existing modeling languages to better describe failure conditions, interactions, and mitigations, and improvements to compositional reasoning approaches focused on the specific needs of system safety analysis. We develop example systems based on the Wheel Braking System model in SAE AIR6110 to evaluate the effectiveness and practicality of our approach. 

The goal of the AMASE project is to develop Model-Based Safety Analysis (MBSA) methods that will strengthen safety assurance by extending current system architecture modeling techniques and exploiting recent advances in formal compositional reasoning. Our approach will allow system and safety engineers to analyze system architecture models composed from heterogeneous software components that have been annotated with formally proved behavioral contracts. These models include fault behaviors and support assessment of both nominal and faulty system behaviors.

