%\documentclass{sig-alternate-05-2015}
\documentclass{llncs}
\usepackage{makeidx}
\usepackage{tabularx,colortbl}
\usepackage[dvipsnames]{xcolor}
\usepackage{flushend}
\usepackage{cite}
\usepackage{amsmath}
%\usepackage{amsthm}
\usepackage{amssymb}
\usepackage{epsfig}
\usepackage{stmaryrd}
\usepackage{url}
\usepackage{multirow}
\usepackage{latexsym}
\usepackage{graphics}
\usepackage{graphicx}
\usepackage{enumitem}
\usepackage{comment}
\usepackage{longtable}
\usepackage{supertabular}
\usepackage{times}
\usepackage{listings}
\usepackage{subfigure}
\usepackage{color}
\usepackage{balance}
\usepackage{xspace}
\usepackage[ruled, vlined, linesnumbered]{algorithm2e}
\usepackage[autostyle]{csquotes}



%\theoremstyle{Definition}
%\newtheorem{definition}{Definition}
%%%
%\theoremstyle{Theorem}
%\newtheorem{theorem}{Theorem}


%\newcommand{\definition}{\noindent \textbf{Definition} \citation{}}
%\newcommand{\theorem}{\noindent \textbf{Theorem} \citation{}}
%\newcommand{\lemma}{\noindent \textbf{Lemma} \citation{}}

%\newdef{lemma}{Lemma}
%\newdef{definition}{Definition}
%\newdef{theorem}{Theorem}
%\newdef{corollary}{Corollary}
%\newdef{note}{Note}
%\newdef{axiom}{Axiom}
\newcommand{\mkeyword}[1]{\mbox{\texttt{#1}}}
\DeclareMathOperator{\kuop}{uop}
\DeclareMathOperator{\kbop}{bop}
\DeclareMathOperator{\kite}{ite}
\DeclareMathOperator{\kpre}{pre}
\DeclareMathOperator{\dom}{dom}
\DeclareMathOperator{\ktrue}{true}
\DeclareMathOperator{\kfalse}{false}
\DeclareMathOperator{\kselect}{select}
\DeclareMathOperator{\ran}{range}
\newcommand{\lbb}{[\![}
\newcommand{\rbb}{]\!]}
\newcommand{\expr}{\phi}
\newcommand{\exprS}{\Phi}
\newcommand{\mike}[1]{\textcolor{red}{#1}}
\newcommand{\mats}[1]{\textcolor{blue}{#1}}
\newcommand{\darren}[1]{\textcolor{green}{#1}}
\newcommand{\danielle}[1]{\textcolor{orange}{#1}}

\sloppypar



\begin{document}

\definecolor{gold}{rgb}{0.90,.66,0}
\definecolor{dgreen}{rgb}{0,0.6,0}
\newcommand{\stateequiv}{\equiv_{s}}
\newcommand{\traceequiv}{\equiv_{\sigma}}
\newcommand{\ta}{\text{TA}}
\newcommand{\cta}{\text{TA$_{C}$}}
\newcommand{\tta}{\text{TA$_{T}$}}
\newcommand{\ucalg}{\texttt{\small{IVC\_UC}}}
\newcommand{\ucbfalg}{\texttt{\small{IVC\_UCBF}}}


\title{Architectural Modeling and Analysis for Safety Engineering}
%
\author{Danielle Stewart\inst{1}
\and Michael W. Whalen\inst{1}
\and Darren Cofer\inst{2}
\and Mats P.E. Heimdahl\inst{1} }
\institute{University of Minnesota\\Department of Computer
Science and Engineering\\
200 Union Street\\
Minneapolis, MN, 55455, USA\\
\email{whalen, dkstewar, heimdahl@cs.umn.edu}
\and
Rockwell Collins\\
Advanced Technology Center\\400 Collins Rd. NE\\
Cedar Rapids, IA, 52498, USA\\ \email{ darren.cofer@rockwellcollins.com}
}
\maketitle

\begin{abstract}
Architecture description languages such as AADL allow systems engineers to specify the structure of system architectures and perform several analyses over them, including schedulability, resource analysis, and information flow.  In addition, they
permit system-level requirements to be specified and analyzed early in the development process of airborne and ground-based systems. These tools can also be used to perform safety analysis based on the system architecture and initial functional decomposition.

%Previously, Rockwell Collins and the University of Minnesota developed and demonstrated an approach to behavioral model-based safety analysis using Simulink. New MBSE tools that incorporate assume-guarantee compositional analysis techniques provide the basis for greatly improving earlier approaches to safety analysis and can be used to ensure model consistency, correctness of assumptions, and better scalability.

Using AADL-based system architecture modeling and analysis tools as an exemplar, we extend existing analysis methods to support system safety objectives of ARP4754A and ARP4761. This includes extensions to existing modeling languages to better describe failure conditions, interactions, and mitigations, and improvements to compositional reasoning approaches focused on the specific needs of system safety analysis. We develop example systems based on the Wheel Braking System in SAE AIR6110 to evaluate the effectiveness and practicality of our approach.
\end{abstract}

\keywords{Model-based systems engineering, fault analysis, safety engineering}

\section{Introduction}
\label{sec:intro}

%Mats' intro
System safety analysis is crucial in the development life cycle of critical systems to ensure adequate safety as well as demonstrate compliance with applicable standards. A prerequisite for any safety analysis is a thorough understanding of the system architecture and the behavior of its components; safety engineers use this understanding to explore the system behavior to ensure safe operation, assess the effect of failures on the overall safety objectives, and construct the accompanying safety analysis artifacts. Developing adequate understanding, especially for software components, is a difficult and time consuming endeavor. Given the increase in model-based development in critical systems~\cite{Joshi05:Dasc,CAV2015:BoCiGrMa,info17:HaLuHo,5979344,Gudemann:2010:FQQ:1909626.1909813}, leveraging the resultant models in the safety analysis process holds great promise in terms of analysis accuracy as well as efficiency.

In this paper we describe the \emph{Safety Annex} for the system engineering language AADL (Architecture Analysis and Design Language), a SAE Standard modeling language for Model-Based Systems Engineering (MBSE)~\cite{AADL_Standard}. The Safety Annex allows an analyst to model the failure modes of components and then ``weave'' these failure modes together with the original models developed as part of MBSE. The safety analyst can then leverage the merged behavioral models to propagate %failures
errors through the system to investigate their effect on the safety requirements. %(implicit %failure
%error propagation). 
Determining how %faults
errors propagate through software components is currently a costly and time-consuming element of the safety analysis process. 
\begin{comment} 
The use of behavioral contracts to capture the implicit %fault
error propagation characteristics of software component is a significant benefit for safety analysts.  
In addition, the annex allows modeling of explicit %failure 
error propagation that is not captured through the behavioral models, for example, the effect of a single electrical failure on multiple software components or the effect hardware failure (e.g., an explosion) on multiple behaviorally unrelated components. 
\end{comment}
The use of behavioral contracts to capture the %implicit %fault
error propagation characteristics of software component without the need to add separate propagation specifications (\emph{implicit} error propagation) is a significant benefit for safety analysts.  
In addition, the annex allows modeling of %explicit %failure 
dependent faults that are not captured through the behavioral models (\emph{explicit} error propagation)},
%error propagation that is not captured through the behavioral models, 
for example, the effect of a single electrical failure on multiple software components or the effect hardware failure (e.g., an explosion) on multiple behaviorally unrelated components. 
Furthermore, we will describe the tool support enabling engineers to investigate the correctness of the nominal system behavior (where no failures have occurred) as well as the system's resilience to component failures. We illustrate the work with a substantial example drawn from the civil aviation domain.

Our work can be viewed as a continuation of work conducted by Joshi et al.~where they explored model-based safety analysis techniques defined over Simulink/Stateflow~\cite{MathWorks} models~\cite{Joshi05:SafeComp,Joshi07:Hase,Joshi05:Dasc,DBLP:conf/cav/BozzanoCPJKPRT15}. Our current work extends and generalizes this work and provide new modeling and analysis capabilities not previously available.  For example, the Safety Annex allows modeling explicit %fault 
error propagation, supports compositional verification and exploration of the nominal system behavior as well as the system's behavior under failure conditions. Our work is also closely related to the existing safety analysis approaches, in particular, the AADL Error Annex (EMV2)~\cite{EMV2}, COMPASS~\cite{10.1007/978-3-642-04468-7_15}, and AltaRica~\cite{PROSVIRNOVA2013127,BieberERTS2018}. Our approach is significantly different from previous work in that unlike EVM2 we leverage the behavioral modeling for implicit %failure 
error propagation.  We provide compositional analysis capabilities not available in COMPASS.  In addition, the Safety Annex  is fully integrated in a model-based development process and environment unlike a stand alone language such as AltaRica. 

The contributions of the Safety Annex and this paper are:
\begin{itemize}
\renewcommand{\labelitemi}{\textbullet}
		\item close integration of behavioral fault analysis into the {\em Architecture Analysis and Design Language} AADL, which allows close connection between system and safety analysis and system generation from the model,
		\item support for {\em behavioral specification of faults} and their {\em implicit propagation} through behavioral relationships in the model, in contrast to existing AADL-based annexes (HiP-HOPS, EMV2) and other related toolsets (COMPASS, Cecilia, etc.),
		\item additional support to capture binding relationships between hardware and software and logical and physical communications, and
		\item guidance on integration into a traditional safety analysis process.
\end{itemize}

%The remainder of the paper is organized as follows. INSERT WHATEVER IT ENDED UP LOOKING LIKE. 

\begin{comment}
%Danielle's and Janet's intermediate intro

System safety analysis techniques are crucial in the development life cycle of highly integrated/complex aircraft systems and are used to show compliance with certification requirements. A prerequisite of performing any safety assessment of a system design is to understand how the system is intended to work, primarily focusing on the relationship between component outputs and the overall behavior of the system. The safety engineers then use this information to conduct safety analysis, construct the safety analysis artifacts, and compare the analysis results with established safety objectives and safety-related requirements. Acquiring knowledge about the behavior of the software applications hosted in a system and its impact on the overall system behavior is typically a time consuming and involved process.

To help solve this problem, researchers and industry practitioners have turned to the use of models. Models have been shown to be an effective way to help engineers capture, understand and analyze complex systems. Previous work has been done showing the benefits of leveraging the system model in the safety analysis process~\cite{Joshi05:SafeComp,Joshi07:Hase,Joshi05:Dasc,DBLP:conf/cav/BozzanoCPJKPRT15}.

%In order to effectively assist safety engineers to acquire the knowledge on software application behaviors and assess their effects on the overall behavior of the system, the models should allow system designers to capture the expected behaviors of the software application and the expected behavioral propagations among different components/application in the system; and allow safety engineers to leverage the same model provided by system designers, capture failure modes for individual components, and automatically assess the effects to the overall system through the behavioral propagations built in the existing model.

During this safety analysis process, it is important to reason about faults and how faulty component behaviors can impact the overall system behavior. In order to address the problem of understanding the model and complex system, it is advantageous to provide an automated analysis framework that allows for various types of fault definitions, propagations, and modeling options. 

This paper introduces a tool that provides a solution to these problems: the Safety Annex for the system engineering language called Architecture Analysis and Design Language (AADL), a widely used SAE Standard design language for MBSE applications~\cite{AADL_Standard}. Given a system model in AADL and a behavioral model developed in the Assume Guarantee Reasoning Environment (AGREE)~\cite{NFM2012:CoGaMiWhLaLu}, the Safety Annex is a fault modelling tool that utilizes model checking in order to analyze the behavior of a system in the presence of faults. The Safety Annex allows safety engineers to leverage existing models from system development for conducting assessment. 

Throughout this paper we show that the Safety Annex allows both implicit and explicit failure propagation which gives richer fault modeling capabilities than comparable tools. It is also shown how behavioral information regarding the active faults, the component properties and the overall system behavior when faults are active are provided. We demonstrate that the toolset (AADL, AGREE, and Safety Annex) captures behaviors of both the nominal model (absence of faults) and the faulty model in a cleanly separated and yet analyzable fashion. This serves to preserve the system model for the systems engineering process and simultaneously be able to see their combined effect on the system behavior. 


 %Using a Model-Based Safety Analysis (MBSA) approach allows safety engineers to weave a fault model into the entire MBSE process while preserving the separation of a system model and a fault model.


%To help solve this problem, researchers and industry practitioners have turned to Model-based System Engineering (MBSE). Models have been shown to be an effective way to help engineers capture, understand and analyze complex systems. Previous work has been done showing the benefits of leveraging the system model in the safety analysis process~\cite{Joshi05:SafeComp,Joshi07:Hase,Joshi05:Dasc,DBLP:conf/cav/BozzanoCPJKPRT15}. Using a Model-Based Safety Analysis (MBSA) approach allows an analyst \janet{use the term ``safety engineer'' only or both ``safety engineer'' and ``safety analyst''?} to weave a fault model into the entire MBSE process while preserving the separation of a system model and a fault model.
\end{comment}

\iffalse

Throughout the development life cycle of highly-integrated/complex aircraft systems, safety assessment process is a crucial piece in asserting development assurance, and is used to show compliance with certification requirements and meeting a company's internal safety standards. A prerequisite of performing any safety assessment of a system design is to understand how the system is intended to work, primarily focusing on the relationship between component outputs and the overall behavior of the system. The safety engineers then use the acquired understanding to conduct safety analysis, construct the safety analysis artifacts, and compare the analysis results with established safety objectives and safety-related requirements.  

In practice, prior to performing the safety assessment of a system, the safety engineers are often equipped with the domain knowledge about the system, but do not necessarily have detailed knowledge of how the software functions are designed. Acquiring the required knowledge about the behavior and implementation of each software function in a system is typically the most time consuming and involved step in the process.

Industry practitioners have come to realize the benefits and importance of
using models to assist the safety assessment process, such as to better understand system behaviors, communicate with system designers, capture the failure propagations, and manage and analyze more complex systems. And a revision to the Guidelines and Methods for Conducting Safety Assessment Process on Civil Airborne Systems and Equipment~\cite{SAE:ARP4761} to include {\em model based safety analysis} is under way.

%condensed version
%System safety assessment is a crucial process in the development of complex airborne systems to show that the relevant safety requirements are met. Acquiring the required knowledge about how the software functions are intended to work in such systems has shown in practice a very involved and time consuming task. Existing approaches that annotate the system architecture model with failure modes and fault propagations help safety analysts better communicate with system designers and address the complexity of the system. However, knowledge on how the faults propagate through the components still needs to be acquired by safety analysts before such information can be captured in the model. Acquiring the information on fault propagation is still a manual effort.

We think that the following criteria are important for the models to help safety analysts effectively acquire the knowledge about system/software behaviors and capture/analyze failure propagations:

\begin{itemize}
	\item Allow safety engineers to leverage existing models from system development for conducting assessment. %This captures the current state of the system design as it moves through the system development lifecycle, reducing the gap in comprehending the system behavior and transferring the knowledge between the system designers and the safety analysts.
	\item Support capturing behaviors of nominal and faulty behaviors in the system that are cleanly separated, yet analyzable in an integrated fashion to see their combined effect on the system.
	\item Enable safety engineers to inject failures/faults at component level, and assess the effect of behavioral propagation at system level, without needing to acquire the knowledge on the propagation beforehand. 
	\item Allow safety engineers to add failure propagations to the model that may not be behavioral related such as common cause/hardware dependent faults (e.g., common failures such as pipe burst that can propagate through physical systems).
\end{itemize}

\janet{To add: describe our approach that satisfy the criteria and all other work don't}
%The methodology described in this paper enables safety analysts to specify faults and faulty behaviors at individual components (using the Safety Annex for the Architecture Analysis and Design Language (AADL)). 

%The provided tool support auto weaves the faults into the nominal system model provided by system designers. No additional effort is needed to specify fault propagations as the faulty behavior propagates in the nominal system model the same as the normal behavior. The behavior of the system in the presence of faults are verified using model checking through Assume Guarantee Reasoning Environment (AGREE), and the effects of any triggered fault are manifested in the formal analysis results.

%1. Introduce our approach and it addresses all that
%2. behavioral propagation of failures
%3. We could just describe/split in implicit and explicit fault/failure propagations. E.g., for explicit failure propagation, now we can connect behaviorally unrelated components.
%3. You activate a fault/inject a failure to the system, so it's not fault propagation, but failure propagation

%How we evaluate our work in comparison with others'
%1. There are other approaches support some of them. However, they don't support ...
%What are the related work and why they don't solve the problem:
%Researchers like Anjali have explored ...
%EMV2
%xSAP
%2. What we do is different from XXX because ...
%3. The EMV2 is really talking about failure propagation. They view errors as corrupted states. That lead to certain level of confusion. 
%4. In ARP4754, an error is treated as a source of fault, but a fault can happen without error. An error might lead to a fault. A SW can have errors as software doesn't fail on its own - someone has to put it there

\begin{comment}
%This paper describes a new methodology with tool support for model based safety analysis. It is implemented as a {\em Safety Annex} for the Architecture Analysis and Design Language (AADL). The Safety Annex provides the ability to describe faults and faulty component behaviors in AADL models. In contrast to previous AADL-based approaches, the Safety Annex leverages a formal description of the nominal system behavior to propagate faults in the system. This approach ensures consistency with the rest of the system development process and simplifies the work of safety engineers. The language for describing faults is extensible and allows safety engineers to weave various types of faults into the nominal system model. The Safety Annex supports the injection of faults into component level outputs, and the resulting behavior of the system can be analyzed using model checking through the Assume-Guarantee Reasoning Environment (AGREE).

System safety analysis techniques are well-established and are a required activity in the development of safety-critical systems. Model-based systems engineering (MBSE) methods and tools based on formal methods now permit system-level requirements to be specified and analyzed early in the development process~\cite{NFM2012:CoGaMiWhLaLu,CAV2015:BoCiGrMa}. While model-based development methods are widely used in the aerospace industry, they are only recently being applied to system safety analysis.  

%How can we leverage these model-based methods and tools to perform safety analysis based on models of the system architecture and initial functional decomposition? Can these design models be integrated into the safety analysis process to help guarantee accurate and consistent results?
%Seeking solutions to these questions are especially important as the amount of safety-critical hardware and software in various domains has drastically increased due to the demand for greater autonomy, capability, and connectedness.

In this paper, we describe a {\em Safety Annex} for the Architecture Analysis and Design Language (AADL)~\cite{FeilerModelBasedEngineering2012} that provides the ability to reason about faults and faulty component behaviors in AADL models. In the Safety Annex approach, we use formal assume-guarantee contracts to define the nominal behavior of system components. The nominal model is then verified using the Assume Guarantee Reasoning Environment (AGREE)~\cite{NFM2012:CoGaMiWhLaLu}. The Safety Annex  provides a way to weave faults into the nominal system model and analyze the behavior of the system in the presence of faults. The Safety Annex also provides a library of common fault node definitions that is customizable to the needs of system and safety engineers. Our approach adapts the work of Joshi et. al in
~\cite{Joshi05:Dasc} to the AADL modeling language, and provides a domain specific language for the kinds of analysis performed manually in previous work~\cite{Stewart17:IMBSA}.  %More information on the approach is available in~\cite{Stewart17:IMBSA}, and the tool and relevant documentation can be found at: \small \url{https://github.com/loonwerks/AMASE/}. \normalsize

%There are other tools purpose-built for safety analysis, including AltaRica~\cite{PROSVIRNOVA2013127}, smartIFlow~\cite{info8010007} and xSAP~\cite{DBLP:conf/tacas/BittnerBCCGGMMZ16}. These notations are separate from the system development model. Other tools extend existing system models, such as HiP-HOPS~\cite{CHEN201391} and the AADL Error Model Annex, Version 2 (EMV2)~\cite{EMV2}. EMV2 uses enumeration of faults in each component and explicit propagation of faulty behavior to perform safety analysis. The required propagation relationships must be manually added to the system model and can become complex, leading to potential omissions and inconsistencies.

The Safety Annex supports model checking and quantitative reasoning by attaching behavioral faults to components and then using the normal behavioral propagation and proof mechanisms built into the AGREE AADL annex. This allows users to reason about the evolution of faults over time, and produce counterexamples demonstrating how component faults lead to system failures. It can serve as the shared model to capture system design and safety-relevant information, and produce both qualitative and quantitative description of the causal relationship between faults/failures and system safety requirements.
%
Thus, the contributions of the Safety Annex and this paper are:
\begin{itemize}
\item Close integration of behavioral fault analysis into the {\em architectural design language} AADL, which allows close connection between system and safety analysis and system generation from the model,
\item support for {\em behavioral specification of faults} and their {\em implicit propagation} through behavioral relationships in the model, in contrast to existing AADL-based annexes (HiP-HOPS, EMV2) and other related toolsets (COMPASS, Cecilia, etc.),
\item additional support to capture binding relationships between hardware and software and logical and physical communications, and
\item guidance on integration into a traditional safety analysis process.
\end{itemize}
%\mike{What are our contributions?}
\end{comment}

\fi

\section{Safety Assessment Process}
\label{sec:process}
\mike{COMPLETELY STOLEN FROM THE SAFECOMP-05 PAPER!  Either note it or modify it}
\danielle{I added a citation. I think this late in the game we might as well just cite it.}

The overall safety assessment process that is followed in practice
in the avionics industry is described in the SAE standard ARP
4761~\cite{SAE:ARP4761}. Our summary in this section is largely
adopted from ARP 4761.

\iffalse

This section describes the overall safety assessment process that is followed in
practice in the avionics industry along the lines of the SAE standard ARP 47-61 \cite{SAE:ARP4761}. The descriptions of the various phases of the safety assessment process
covered in this section are essentially based on the ARP 47-61 document.

\fi


\begin{figure}
\includegraphics[trim=25 375 0 125, clip, scale=.60]{V}
\caption{Traditional ``V'' Safety Assessment Process} \label{fig:V}
\end{figure}

%The safety assessment process is an integral part of the development process.

Figure~\ref{fig:V} from \cite{Joshi05:SafeComp} shows an overview of the safety assessment
process as recommended in ARP 4761. The process includes safety
requirements identification (the left side of the ``V'' diagram)
and verification (the right side of the ``V'' diagram), that
support the aircraft development activities. An aircraft level
Functional Hazard Analysis (FHA) is conducted at the beginning of
the aircraft development cycle, which is then followed by system
level FHA for individual sub-systems. The FHA is followed by
Preliminary System Safety Assessment (PSSA), which derives safety
requirements for the subsystems, primarily using Fault Tree
Analysis (FTA). The PSSA process iterates with the design
evolution, with design changes necessitating changes to the
derived system requirements (and also to the fault trees) and
potential safety problems identified through the PSSA leading to
design changes. Once design and implementation are completed, the
System Safety Assessment (SSA) process verifies whether the safety
requirements are met in the implemented design. The system Failure
Modes and Effects Analysis (FMEA) is performed to compute the
actual failure probabilities on the items. The verification is
then completed through quantitative and qualitative analysis of
the fault trees created for the implemented design, first for the
subsystems and then for the integrated aircraft.

%\medskip

We propose to modify this traditional ``V'' process so that the
lower level PSSA and SSA activities are performed based on a
formal model of the system under consideration.
Figure~\ref{fig:Vmod} shows the modified ``V'' diagram for
model-based safety analysis. The shaded blocks are those
activities that will be modified or added.

\begin{figure}
\includegraphics[trim=15 350 0 125, clip, scale=.60]{Mod_V_Process_FaultModel}
\caption{Modified ``V'' Safety Assessment Process} \label{fig:Vmod}
\end{figure}


As we can observe from Figure~\ref{fig:Vmod}, the parts of the analysis that are
primarily affected are at the bottom of the ``V''. The biggest difference is that
the safety analysis activities at this level are now focused around a formal
model of the system behavior, and that many of the artifacts of the safety
analysis can be derived from this model. The idea is to try to pose the right
verification questions to formal tools (such as model checkers and theorem
provers) so that it is possible to derive the necessary safety analysis
information. We then wish to turn the results of these analyses back into
artifacts that can be easily understood and used by safety engineers.


\section{Example: Wheel Brake System}

%\mike{Move the generic description to a section right after the ``safety'' section along with the ARP figure.  Leave the modeling that we did in AADL here.  That way we can reference it in the MBSA sections.}
%\danielle{The whole nominal model description is here. The modeling we did is in the faults section.}

As a preliminary case study, we utilized the Wheel Brake System (WBS) described in \cite{AIR6110} (previously found in ARP4761 Appendix L). This ficticious aircraft system was developed to illustrate the design and safety analysis principles of ARP4754A and ARP4761.  The WBS is installed on the two main aircraft landing gears and is used during taxi, landing, and rejected take off. Braking is either commanded manually using brake pedals or automatically by a digital control system with no need for the pedals (autobrake). When the wheels have traction, the autobrake function will provide a constant smooth deceleration.

\begin{figure}
\begin{center}
\includegraphics[width=\textwidth]{images/wbsfederated3.jpg}
\caption{AADL Simple Model of the Wheel Brake System }
\label{fig:wbs_ima}
\end{center}
\end{figure}

Each wheel has a brake assembly that can be operated by two independent hydraulic systems (designated green and blue). In normal braking mode, the green hydraulic system operates the brake assembly.  If there is a failure in the green hydraulics, the system switches to alternate mode which uses the blue hydraulic system.  The blue system is also supplied by an accumulator which is a device that stores hydraulic pressure that can be released if both of the primary hydraulic pumps (blue and green) fail. The accumulator supplies hydraulic pressure in Emergency braking mode.

Switching between the hydraulic pistons and pressure sources can be commanded automatically or manually. If the hydraulic pressure in the green supply is below a certain threshold, there is an automatic switchover to the blue hydraulic supply. If the blue hydraulic pump fails, then the accumulator is used to supply hydraulic pressure.

In both normal and alternate modes, an anti-skid capability is available. In the normal mode, the brake pedal position is electronically fed to a computer called the Braking System Control Unit (BSCU). The BSCU monitors signals that denote critical aircraft and system states to provide correct braking function, detect anomalies, broadcast warnings, and sent maintenance information to other systems.

\subsection{Nominal System Model}
\label{sec:nominal}
The WBS AADL model of the nominal system behavior consists of mechanical and digital components and their interconnections, as shown in Figure~\ref{fig:wbs_ima}. The following section describes this nominal model from which the fault model was generated.

\paragraph{Wheel Braking System (WBS)}
The highest level model component is the WBS. It consists of the BSCU, green and blue hydraulic pressure lines (supplied by the green pump  and blue pump/accumulator respectively), a Selector which selects between normal and alternate modes of hydraulic pressure, and the wheel system. The WBS takes inputs from the environment including PedalPos1, AutoBrake, DecRate, AC\_Speed, and Skid. All of these inputs are forwarded to the BSCU to compute the brake commands.

\paragraph{Braking System Control Unit (BSCU)}
The BSCU is the digital component in the system that receives inputs from the WBS. It also receives feedback from the green and blue hydraulic lines and two power inputs from two separate power sources. The BSCU is composed of two command and monitor subsystems each powered independently from separate power sources. The pedal position is provided to these units and when skidding occurs, the command and monitor units will decrease the pressure to the brakes.
The command unit regulates the pressure to the brakes in the green hydraulic line through the command cmd\_nor. Computing this command requires both the brake requested power and the skid information. The command unit also regulates the pressure in the blue hydraulic line in order to prevent skidding which it does through the cmd\_alt command. The monitor unit checks the validity of the command unit output.

The BSCU switches from normal to alternate mode (blue hydraulic system) when the output from either one of its command units is not valid or the green hydraulic pump is below its pressure threshold.  Once the system has switched into alternate mode, it will not switch back into normal mode again.

\noindent Once the system has switched into alternate mode, it will not switch into normal mode again.

\paragraph{Hydraulic Pumps}
There are three hydraulic pumps in the system, green pump (normal mode), blue pump (alternate mode), and accumulator pump (emergency mode). Each pump provides pressure to the system and is modeled in AADL as a floating point value.

\paragraph{Shutoff Valve}

The shutoff valve is situated between the green pump and the selector. It receives an input from the BSCU regarding valve position and regulates the pressure coming through the green pipe accordingly.

\paragraph{Selector Valve}
The selector receives inputs from the pumps regarding pressure output and the BSCU regarding which mode the system is in. It will output the appropriate pressure from green, blue, or accumulator pump. An added requirement of the selector system is that it will only output pressure from one of these sources. Thus, the case of having pressure supplied to the wheels from more than one pump is avoided. The Selector takes the two pipe pressures (green and blue) as input, selects the system with adequate pressure and blocks the system with inadequate pressure. If both systems have pressure greater than the threshold, the AADL selects normal mode as the default.

\paragraph{Skid Valves}
The blue\_skid and green\_skid valves receive input from the selector as pressure coming through the respective pipes as well as input from the BSCU that commands normal or alternate mode. The skid valves will use these inputs to choose between the green or the blue pressure to send to the wheel.

\subsection{Modeling Nominal System Behavior}
In order to reason about behaviors of complex system architectures, we have developed a compositional verification tool for AADL models.
Our tool, the {\em Assume-Guarantee Reasoning Environment} (AGREE) \cite{NFM2012:CoGaMiWhLaLu}  is based on {\em assume-guarantee} contracts that can be added to AADL components.  The language used for contract specification is based on the LUSTRE dataflow language~\cite{Halbwachs91:IEEE}. The tool allows scaling of formal verification to large systems by splitting the analysis of a complex system architecture into a collection of verification tasks that correspond to the structure of the architecture.

We use AGREE to specify behavioral contracts corresponding to the behaviors expected of each of the WBS components. An example of a contract is shown in Figure~\ref{fig:agreeContract}.
%

\begin{figure}
\begin{center}
\includegraphics[trim=60 480 50 60,clip,width=\textwidth]{images/bscu.pdf}
\caption{AGREE Contract for BSCU }
\label{fig:agreeContract}
\end{center}
\end{figure}

\iffalse

\subsection{Nominal System Modeling}
\mike{KEEP HERE!}
A formal specification of the nominal system model consists of mechanical and digital components and their interconnections.

The highest level component is the Wheel Braking System (WBS). It consists of a digital control unit, the BSCU, and normal and alternate hydraulic pressure lines (supplied by green pump and blue pump/accumulator respectively). The system takes inputs from the environment including PedalPos1, AutoBrake, DecRate, AC\_Speed, and Skid. All of these inputs are forwarded to the BSCU to compute the brake commands. The outputs of the WBS are normal\_pressure, alternate\_pressure, and System\_Mode (normal, alternate, EMERGENCY).

\subsection{Braking System Control Unit (BSCU)}
The BSCU is the digital component in the system that receives inputs from the WBS. It also receives feedback from the normal and alternate lines and two power inputs from two separate power sources.

\fi



\section{Background}

Background information on fault trees, IVCs, and Soteria.




\section{New MBSA Capabilities}

\mike{REWRITE!}

Previous research has been based on MBD tools (such as Simulink and SCADE) that were available at the time \cite{Joshi05:Dasc}. However, these tools are really targeted at the design and implementation of software components, rather than at the system architecture level where most safety concerns arise.

Within the past five years there have been great advances in the capabilities of tools for modeling and analysis of at the system level, based on languages such as SysML \cite{SysML} and AADL \cite{AADL}. We use these new MBSE capabilities and extend them to implement the safety analysis methods needed for the design and certification of commercial aircraft systems. The system modeling tools that we plan to use are based on AADL, but they can import and export models from SysML.

\begin{itemize}
\item We will improve the efficiency of MBSA methods by using MBSE tools that can perform compositional reasoning over complex system models. Using the assume- guarantee contract mechanism, these tools provide support for heterogeneous component models implemented in different languages (such as Simulink or C/C++).

\item Our new analysis methods move away from traditional static safety analysis methods focused on probabilistic models (e.g., Fault Tree Analysis), to the direct modeling of potential failure mechanisms and the analysis of dynamic fault-mitigation strategies.

\item Formal verification of system models provides increased assurance that these models are accurate and will produce correct results. The hierarchical structure of system architecture models supports analysis at varying levels of abstraction. Compositional analysis explicitly checks assumptions captured in component and subsystem contracts. Consistency and realizability checks [23] provide the ability to detect conflicting requirements between component and subsystem models.
\end{itemize}

%In this section we describe some of the new capabilities that we will develop, including improvements related to the AADL Error Model Annex, the use of model-based assurance cases, and evaluation of system-level certification objectives related to system safety that can be satisfied using the proposed methods.

We bridge the descriptions of errors in the error model annex with behavioral descriptions of components. We start from the error model notions of error types and state machines that describe transitions from nominal to error states. However, we then tie these nominal and error states to behavioral models of the components in question that describe how the faults manifest themselves in terms of the signals or quantities produced by the components. Now the behavioral models can provide implicit propagation of the faulty behaviors and the natural consequences of failures on component behavior will be manifested in the propagation of other component faults through the behavioral model.

To accomplish this, we use AADL and the error model annex to describe faults, and to use the AGREE contract specification language to describe behavioral models. This requires extensions to AGREE to define fault models that describe how different faults manifest themselves in changes to output signals. It also requires changes to the error annex. The conditions under which faults occur will become richer such that they describe not just propagation of enumerations from other components, but also valuations of input signals.

An example is shown in Figure~\ref{fig:error_annex}. Errors are described by the red rectangles labeled with error types: 1-BadData, 2-NoData, 3-NoSvc. Error events that can cause a component to fail are labeled with the corresponding error number. The error behavior of components is described by their state machines. Note that while all state machines in Figure~\ref{fig:error_annex} have two states, they can be much more complex. The red dashed arrows indicate propagations describing how failures in one component can cause other components to fail. For example, failures in the physical layer propagate to failures in the associated logical components.




%\subsection{Fault and Behavioral Modeling}



\section{Fault Modeling}

In order to start from the error model notions of error types and look more closely at the state machines that describe transitions from the nominal state to the error state, we focus on the types of errors that can be present in the components of the system. 

To begin with, we proved a top level property of the WBS using AGREE. This property states: \\

\begin{tt}
If pedals are pressed and no skid occurs, then the brakes will receive pressure. \\
\end{tt}

Using the AADL nominal model, this property proves. From this point, we focus our attention on component failures and how this will affect the top level property of the system. 

We would like to specify different component failure modes. These failure modes can be triggered by some internal error or propagated fault. In order to trigger these faults, additional input was added to the AADL model for each fault that can occur within a nominal model component. Our simple fault model contains two types of faults:

\begin{itemize}
\item \textit{fail\_to} fault: This type of failure accounts for both nondeterministic failures and stuck-at failures. The components that are affected by this fault include meter valves and pumps. This fault can be modeled in such a way as to describe both digital and mechanical errors. Examples of digital errors include a \textit{stuck\_at} failure mode for the Command subsystems in the BSCU component. This causes the Command units to become stuck at a previous value. An example of a mechanical error that is captured by this fault is a valve stuck open (or closed). The definition in AGREE of this fault is shown in Figure~\ref{fig:fail_to_node}.\\

\begin{figure}[h!]
  \centering
 \includegraphics[width=1\textwidth]{images/fail_to.png}
  \vspace{-0.1in}
  \caption{AGREE Definition of a \textit{fail\_to} Fault}
  \label{fig:fail_to_node}
\end{figure}

%%%%%%% DANIELLE
%                     CHECK INTO MECHANICAL FAILURES

\item \textit{inverted\_fail} fault: This type of fault will be used on components which contain boolean output. It will simply take boolean input, negate it, and output the negated value. An examples of this is the Selector. See Figure~\ref{fig:inverted_fail_node}

\end{itemize}

\begin{figure}[h!]
  \centering
 \includegraphics[width=1\textwidth]{images/inverted_fail.png}
  \vspace{-0.1in}
  \caption{AGREE Definition of a \textit{inverted\_fail} Fault}
  \label{fig:inverted_fail_node}
\end{figure}

While modeling these errors, the duration of the fault must also be taken into account. There are both permanent faults and transient faults. We defined an AGREE node called 
\begin{tt}
historically
\end{tt}
which will model the permanent faults based on whether transient faults are historically true. 

The following is a short summary of the failures defined in the fault model. 

\subsection{AADL Faults}
\begin{itemize}

\item Valves and Pumps: All valves and pumps have the possibility of a \textit{fail\_to} fault. This includes Green pump, Blue pump, Accumulator, and the Shutoff valves. 

\item  The Selector can also have a digital \textit{fail\_to} fault regarding the inputs from BSCU commanding to use Normal or Alternate means of pressure along with an \textit{inverted\_fail} fault which would change the boolean value that commands antiskid to activate. 

\end{itemize}

Given our understanding of the AADL system, the assumption was that permanent faults could be introduced into the system and we would still be able to prove our top level property. Upon connecting the faults into the WBS, it was shown that this top level property could not be proven using these permanent faults. Upon further reflection, it was clear why that was the case. 

In the AADL model, the Selector is situated between the pumps and the Blue\_Skid and Green\_Skid components. It's function is as follows: inputs are pump pressures (for Blue, Green, and Accumulator) and information from the BSCU regarding Normal or Alternate pressure. It selects between the Blue and the Green pump in this way. The output of the Selector is pressure of the Blue and Green pumps to the Blue\_Skid and Green\_Skid components respectively. The Blue\_Skid and Green\_Skid components also get a flag from BSCU. Blue\_Skid receives a flag with cmd\_alt while Green\_Skid gets the command cmd\_nor. This will cause those components to either output pressure (if their respective BSCU command is true) or output zero pressure (if the BSCU command is false). The pressure output from here goes directly to the wheel. For a clear picture of this organization, see Figure~\ref{fig:wbs_ima}.

Here is a case when the fault is \textit{fail\_to} and it is a failure of the Green\_Skid component. Previously in the system, the Selector received a command from BSCU to output Normal pressure (Green pump). This command was passed on to the Green\_Skid component along with positive green pressure. If the Green\_Skid component gets stuck at a value of zero for pressure, the output to the wheel is zero. The Blue\_Skid component gets no pressure from the Selector (it has already selected green pressure) and thus outputs no pressure to the wheel. 

Despite one transient failure, the tail end of the system is not designed to recover from this failure. A similar single point of failure is the Blue\_Skid component and the Selector. These faults cause system failure for the same reason. 

\subsection{Strengthening the Nominal Model}

To solve this issue, we added a component based on the Simulink model in \cite{Joshi05:Dasc}. This component, the Accumulator\_Valve, takes in the Blue pressure from the Selector and the Accumulator pressure. It also takes in a \textit{select\_alternate} flag from the BSCU. The output of the Accumulator\_Valve goes directly to the Blue\_Skid component and is either the blue or the accumulator pressure. 

An additional piece we added was output to the BSCU from the wheel. The pressure at the wheel is sent to the BSCU to readjust system mode if needed. For instance, if the pressure at the wheel is zero from both blue and green skid components, then the BSCU has a chance to send a \textit{select\_alternate} command over to the Accumulator\_Valve and turn on the accumulator pressure. We assume at this point that we will only use accumulator pressure and not switch back over to Normal mode. 

In order for this to solve the problem, the original top level contract needed to be revisited: 

\begin{tt}
If pedals are pressed and no skid occurs, then the brakes will receive pressure. 
\end{tt}

As it is stated, the added components will not solve the problem. There must be stateful information encoded into this contract. Thus it is changed to: 

\begin{tt}
If pedals are pressed in the previous state and pressed in the current state and no skid occurs, then the brakes will receive pressure. 
\end{tt}

It was also necessary to guarantee that \textit{select\_alternate} is false until an error occurs in the system. This was written as an AGREE guarantee stating that the initialization of \textit{select\_alternate} is false. It remains false until the system commands brake pressure (i.e. brake pedals are pressed), there is no skid occurring, and there is no realized pressure at the wheel. At this point, the system will enter Alternate mode and command pressure from the blue (or accumulator) pump. It will not return to Normal mode. 



















\section{Discussion}
Some of the goals of extending existing modeling languages in this research is to better describe failure conditions and to better understand interactions of system components in the face of faults. In the process of defining and injecting faults, some subtle issues of the system became far more clear. One of these was the necessity of a non-symmetric system in terms of the Alternate and the Normal hydraulic pressure. As the system is described, the Alternate system requires a separate functionality by means of the Selector and Accumulator valve. Without this separation between the Normal and Alternate systems, one permanent fault will cause failure in the system (namely in either of the skid components). Another failure condition that wasn't initially obvious was the initialization of the system to be in Normal mode. If the initial state of the system is Alternate mode, it cannot recover from a failure in the blue\_skid component or the accumulator\_valve component. 

These failure conditions did not appear obvious by studying the organization of components or proving the nominal system requirements, but instead became clear when utilizing some of these compositional analysis techniques. 

Another goal was to better understand the interactions of system components in the face of failures. This understanding was improved during the process of our initial permanent fault introduction to the system. When a fault was injected into the skid components whose output goes directly to the wheel, the system had no time to select otherwise. The interaction between the BSCU and skid components (or wheel) was nonexistent. In order to rectify this lack of communication, further interaction was required between the wheel and BSCU components. By allowing the system one more step of computation, the system is able to respond to these later failures and correct itself accordingly. This realization came while observing this interaction between components.

Once these issues were addressed, we could successfully prove the top level contract in the case of one permanent fault in any of the high level components. When any one of these failures are present, the system can successfully respond to the failure and still provide pressure to the brakes. 






\section{Conclusions \& Future Work}
In this paper, we describe our initial work towards performing MBSA using the AADL architecture description language using a failure effect modeling approach.  Our goal is to be able to perform safety analysis on common models used by systems and safety engineers for functional and non-functional analyses, schedulability, and perhaps system image generation.  To perform this analysis, we use existing capabilities within AADL to describe the structure of the system, and build on the existing AGREE framework for compositional analysis of components.  

As part of our exploration, we are interested in examining the strengths and weaknesses of our FEM and the AADL Error Annex FLM-based approach.  We believe that the FEM approach has advantages both in terms of brevity of specifications and accuracy of results, and can build on existing analyses performed for systems engineering.  However, there are also risks in the FEM approach involving incomplete or mis-specified properties.  

We illustrated the ideas using architecture models based on the Wheel Braking System model in SAE AIR 6110 \cite{AIR6110} and use this in the evaluation of our approach. Using assume-guarantee compositional reasoning techniques, we prove a top level property of the wheel brake system that states when the brake pedals are pressed in the absence of skidding, there will be hydraulic pressure supplied to the brakes.  

Starting from the error model notions of error types, two main faults were defined: \textit{fail\_to} which will describe failures of valves and pressure regulators and \textit{inverted\_fail} which describes the failures occurring to components that output boolean values. Using the AADL behavioral model of the WBS, these permanent faults were tied into the nominal model in order to reason about how this model behaves in the presence of specific kinds of faults.

In order to demonstrate that the system was resilient to single faults, we modified the model to allow feedback from the wheel pressure to the BSCU.   This changed the way the system responded to faults that were further downstream of the BSCU or Selector and created a chance for the system to switch to alternate forms of hydraulic pressure. We also reasoned about the initialization values of the system in regards to which mode is the starting mode. It is crucial for the system to begin in Normal mode in order to function successfully in the presence of faults.  After model modification and a small weakening of our original property to account for feedback delay, the model does fulfill the top level contract even when a permanent fault of one of the high level components is introduced.

The current capabilities of AGREE are well-suited to specifying faults.  Our approach allows for scalar types of unbounded integers and reals, as well as composite types such as tuples and structures.  It is possible to model systems and reason about them in either discrete time or real-time.  However, adding faults to existing components is cumbersome and can obscure the nominal behaviors of the model.  We are currently examining several fault specification languages, giving special consideration to the xSAP modeling language.

Future research work will involve the continuation of development of the methods and tools needed to perform model-based safety analysis at the system architecture level. By introducing a common set of models for both nominal system design and safety analysis, we hope to reduce the cost of development and improve safety. Our hope is to demonstrate the practicality of formal analysis for early detection of safety issues that would be prohibitively expensive to find through testing and inspection. We will base this research on industry standard notations that are being used in airborne and ground-based avionics in order to ensure transition of this technology.

\subsection*{Acknowledgements} This research was funded by NASA AMASE NNL16AB07T and University of Minnesota College of Science and Engineering Graduate Fellowship.



%ACKNOWLEDGMENTS are optional
%TODO: Fill in for final version
\vspace{0.08in}


%\textbf{Acknowledgments:}
%This work was supported by

%We thank XXXX

\bibliographystyle{abbrv}
\bibliography{biblio}

% This ~ seems to fix an odd bibliography alignment issue
~

%\ifdefined\TECHREPORT
%\appendix
%
%\section{Appendix: Proof of Equivalence}
%\input{appendix}
%\fi

%\section{Appendix: GPCA CENTA Model}
%\label{appendix:gpcacenta}
%\begin{figure}[!ht]
%\begin{center}
%\includegraphics[scale=0.6]{images/sampled_pca.PNG} %[trim = 0 2 0 0, clip=true]{Comp}
%\caption{GPCA AGREE Properties modeled as a Timed Automata} \label{fig:samplepca}
%\end{center}
%\end{figure}

%\balancecolumns

\end{document}
