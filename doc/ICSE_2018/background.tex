\section{Background}

We provide here a brief description of AADL, AGREE, and the SOTERIA tools and languages as background for the reader.

\subsection{Architecture Analysis and Design Language}
We are using the Architectural Analysis and Design Language (AADL)~\cite{FeilerModelBasedEngineering2012} to construct system architecture models.  AADL is an SAE International standard~\cite{AADL_Standard} that defines a language and provides a unifying framework for describing the system architecture for ``performance-critical, embedded, real-time systems''~\cite{AADL_Standard}. From its conception, AADL has been designed for the design and construction of avionics systems.  Rather than being merely descriptive, AADL models can be made specific enough to support system-level code generation.  Thus, results from analyses conducted, including the new safety analysis proposed here, correspond to the system that will be built from the model.  

An AADL model describes a system in terms of a hierarchy of components and their interconnections, where each component can either represent a logical entity (e.g., application software functions, data) or a physical entity (e.g., buses, processors). An AADL model can be extended with language annexes to provide a richer set of modeling elements for various system design and analysis needs (e.g., performance-related characteristics, configuration settings, dynamic behaviors). The language definition is sufficiently rigorous to support formal analysis tools that allow for early phase error/fault detection.

\subsection{Assume Guarantee Reasoning Environment}
The Assume Guarantee Reasoning Environment (AGREE)~\cite{NFM2012:CoGaMiWhLaLu} is a tool for formal analysis of behaviors in AADL models.  It is implemented as an AADL annex and annotates AADL components with formal behavioral contracts. Each component's contracts can include assumptions and guarantees about the component's inputs and outputs respectively, as well as predicates describing how the state of the component evolves over time.

AGREE translates an AADL model and the behavioral contracts into Lustre~\cite{Halbwachs91:IEEE} and then queries a user-selected
model checker to conduct the back-end analysis. The analysis can be performed compositionally following the architecture hierarchy such that analysis at a higher level is based on the components at the next lower level.  When compared to monolithic analysis (i.e., analysis of the flattened model composed of all components), the compositional approach allows the analysis to scale to much larger systems. 

\subsection{Safety Annex for AADL}
The Safety Annex for AADL \cite{Stewart17:IMBSA,SATechReport} is a tool that provides the ability to reason about faults and faulty component behaviors in AADL models. In the Safety Annex approach, formal assume-guarantee contracts are used to define the nominal behavior of system components. The nominal model is verified using AGREE. The Safety Annex weaves faults into the nominal model and analyzes the behavior of the system in the presence of faults. The tool supports behavioral specification of faults and their implicit propagation through behavioral relationships in the model as well as provides support to capture binding relationships between hardware and software compönents of the system. 

\subsection{SOTERIA}
The Safe and Optimal Techniques Enabling Recovery, Integrity, and Assurance (SOTERIA) tool \cite{SOTERIAproject} to perform safety analysis of Integrated Modular Avionics (IMA) systems. In the SOTERIA project, a compositional modeling language was developed and this language is used as input in order to automatically synthesize the qualitative and quantitative safety analyses. The tool is compositional in that it requires safety aspects at each component level which enables the generation of compositional fault trees. 
