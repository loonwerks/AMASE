\section{Introduction}
\label{sec:intro}

System safety analysis techniques are well-established and are a required activity in the development of safety-critical systems. Model-based systems engineering (MBSE) methods and tools based on formal methods now permit system-level requirements to be specified and analyzed early in the development process~\cite{NFM2012:CoGaMiWhLaLu,CAV2015:BoCiGrMa}. While model-based development methods are widely used in the aerospace industry, they are only recently being applied to system safety analysis.  

In this paper, we describe a {\em Safety Annex} for the Architecture Analysis and Design Language (AADL)~\cite{FeilerModelBasedEngineering2012} that provides the ability to reason about faults and faulty component behaviors in AADL models. In the Safety Annex approach, we use formal assume-guarantee contracts to define the nominal behavior of system components. The nominal model is then verified using the Assume Guarantee Reasoning Environment (AGREE)~\cite{NFM2012:CoGaMiWhLaLu}. The Safety Annex  provides a way to weave faults into the nominal system model and analyze the behavior of the system in the presence of faults. The Safety Annex also provides a library of common fault node definitions that is customizable to the needs of system and safety engineers. Our approach adapts the work of Joshi et. al in
~\cite{Joshi05:Dasc} to the AADL modeling language, and provides a domain specific language for the kinds of analysis performed manually in previous work~\cite{Stewart17:IMBSA}.  

The Safety Annex supports model checking and quantitative reasoning by attaching behavioral faults to components and then using the normal behavioral propagation and proof mechanisms built into the AGREE AADL annex. This allows users to reason about the evolution of faults over time, and produce counterexamples demonstrating how component faults lead to system failures. It can serve as the shared model to capture system design and safety-relevant information, and produce both qualitative and quantitative description of the causal relationship between faults/failures and system safety requirements.
%
Thus, the contributions of the Safety Annex and this paper are:
\begin{itemize}
\item Close integration of behavioral fault analysis into the {\em architectural design language} AADL, which allows close connection between system and safety analysis and system generation from the model,
\item support for {\em behavioral specification of faults} and their {\em implicit propagation} through behavioral relationships in the model, in contrast to existing AADL-based annexes (HiP-HOPS, EMV2) and other related toolsets (COMPASS, Cecilia, etc.),
\item additional support to capture binding relationships between hardware and software and logical and physical communications, and
\item guidance on integration into a traditional safety analysis process.
\end{itemize}
