\section{Introduction}
\label{sec:intro}

Risk and safety analyses are an important activity used to ensure that critical systems operate in an expected way. From nuclear power plants and airplanes to heart monitors and automobiles, critical systems are vitally important in our society. The systems are required to not only operate safely under nominal (normal) conditions, but also under conditions when faults are present in the system. Guaranteeing these properties when in the presence of faults is an active area of research and various safety artifacts are often required during the development process of critical systems. In these systems, software is an important component to assess in terms of safety. An area of interest in safety-critical system engineering and development is how failures in the system are propagated through software designs. It is a difficult problem to reason about failure propagations through software in a large scale critical system and automating the process can provide much needed insight into the system and potential problems that could be costly in terms of resources or even life. 

%Formal method techniques are often utilized in the development of safety-critical systems. These mathematical techniques are used to determine correct specification and development of both hardware and software systems. 

 For complex systems, safety analysis techniques can produce thousands of combinations of events that can cause system failure. Many of these events are propagated through software components. Determination of these events can be a very time consuming and error prone process. SAE Standard Aerospace Recommended Practice (ARP) 4761 provides general guidance on evaluating the safety aspects of a design~\cite{SAE:ARP4761} and from the design phase through to the detailed design phase, safety artifacts regarding fault propagation and effects are an important part of the assessment process. 

In this paper, we describe a new technique for determining these events utilizing model checking techniques on a model of the system written in an architecture design language, AADL~\cite{AADL_Standard}. The model is annotated with assume-guarantee contracts for component level requirements on both hardware and software components using AGREE~\cite{QFCS15:backes} and extended into a fault model using the Safety Annex \cite{Stewart17:IMBSA,SATechReport}. A benefit of using this approach for model development is that the faults are specified on a component level, e.g. hardware component, and the behavior of the system as a whole can be observed in the presence of this fault. In other words, it can be seen how the fault manifests itself and is propagated through the software components of the system. 

The artifact produced utilizing this approach is a compositional fault tree in which probabilistic analysis is performed. At each layer of the system, a proof is produced showing the necessary model elements (including faults) that are required in order for the negation of the component requirement to be met. This is often referred to as a Top Level Event (TLE). In the end, the reasons for the fault propagation can be seen as well as the affects of such a propagation. 

%For compositional probabilistic fault analysis, we are going to employ the online enumeration of Minimal Inductive Validity Cores (MIVCs) \cite{GhassabaniGW16,Ghassabani2017EfficientGO} to build a fault dependency graph that shows the causal relationship between leaf level faults and the violation of requirements or safety properties at each architectural level.  We will then feed that fault dependency graph to the Safe and Optimal Techniques Enabling Recovery, Integrity, and Assurance (SOTERIA) tool \cite{SOTERIAproject} to synthesize a fault tree with top level failure probabilities and minimal cutsets computed. 

The rest of the paper is organized as follows. We first describe the architecture language and tool suites used for this approach in section II. We then describe the formalisms and methodology of the approach in section III followed by case studies and experimentation in section IV. Finally we discuss related work and conclusion in sections V and VI.  
