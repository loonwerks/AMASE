%\documentclass{sig-alternate-05-2015}
\documentclass[10pt,conference]{IEEEtran}
%\documentclass{llncs}
\usepackage{makeidx}
\usepackage{tabularx,colortbl}
\usepackage[dvipsnames]{xcolor}
\usepackage{flushend}
\usepackage{cite}
\usepackage{amsmath}
%\usepackage{amsthm}
\usepackage{amssymb}
\usepackage{epsfig}
\usepackage{stmaryrd}
\usepackage{url}
\usepackage{multirow}
\usepackage{latexsym}
\usepackage{graphics}
\usepackage{graphicx}
\usepackage{enumitem}
\usepackage{comment}
\usepackage{longtable}
\usepackage{supertabular}
\usepackage{times}
\usepackage{listings}
\usepackage{subfigure}
\usepackage{color}
\usepackage{balance}
\usepackage{xspace}
\usepackage[ruled, vlined, linesnumbered]{algorithm2e}
\usepackage[autostyle]{csquotes}



%\theoremstyle{Definition}
%\newtheorem{definition}{Definition}
%%%
%\theoremstyle{Theorem}
%\newtheorem{theorem}{Theorem}


%\newcommand{\definition}{\noindent \textbf{Definition} \citation{}}
%\newcommand{\theorem}{\noindent \textbf{Theorem} \citation{}}
%\newcommand{\lemma}{\noindent \textbf{Lemma} \citation{}}

%\newdef{lemma}{Lemma}
%\newdef{definition}{Definition}
%\newdef{theorem}{Theorem}
%\newdef{corollary}{Corollary}
%\newdef{note}{Note}
%\newdef{axiom}{Axiom}
\newcommand{\mkeyword}[1]{\mbox{\texttt{#1}}}
\DeclareMathOperator{\kuop}{uop}
\DeclareMathOperator{\kbop}{bop}
\DeclareMathOperator{\kite}{ite}
\DeclareMathOperator{\kpre}{pre}
\DeclareMathOperator{\dom}{dom}
\DeclareMathOperator{\ktrue}{true}
\DeclareMathOperator{\kfalse}{false}
\DeclareMathOperator{\kselect}{select}
\DeclareMathOperator{\ran}{range}
\newcommand{\lbb}{[\![}
\newcommand{\rbb}{]\!]}
\newcommand{\expr}{\phi}
\newcommand{\exprS}{\Phi}
\newcommand{\mike}[1]{\textcolor{red}{#1}}
\newcommand{\janet}[1]{\textcolor{blue}{#1}}
\newcommand{\darren}[1]{\textcolor{green}{#1}}
\newcommand{\danielle}[1]{\textcolor{orange}{#1}}

\sloppypar



\begin{document}

\definecolor{gold}{rgb}{0.90,.66,0}
\definecolor{dgreen}{rgb}{0,0.6,0}
\newcommand{\stateequiv}{\equiv_{s}}
\newcommand{\traceequiv}{\equiv_{\sigma}}
\newcommand{\ta}{\text{TA}}
\newcommand{\cta}{\text{TA$_{C}$}}
\newcommand{\tta}{\text{TA$_{T}$}}
\newcommand{\ucalg}{\texttt{\small{IVC\_UC}}}
\newcommand{\ucbfalg}{\texttt{\small{IVC\_UCBF}}}


\title{ICSE paper outline}
%


\author{\IEEEauthorblockN{Danielle Stewart\IEEEauthorrefmark{1}, Jing (Janet) Liu\IEEEauthorrefmark{2}, Michael W. Whalen\IEEEauthorrefmark{1}, Darren Cofer\IEEEauthorrefmark{2}, Mats Heimdahl\IEEEauthorrefmark{1}, Michael Peterson\IEEEauthorrefmark{3}}
\IEEEauthorblockA{\IEEEauthorrefmark{1}University of Minnesota\\Department of Computer
	Science and Engineering\\Email: \{dkstewar, whalen, heimdahl\}@umn.edu}
\IEEEauthorblockA{\IEEEauthorrefmark{2}Rockwell Collins\\
	Advanced Technology Center\\Email: \{Jing.Liu, Darren.Cofer\}@rockwellcollins.com}
\IEEEauthorblockA{\IEEEauthorrefmark{3}Rockwell Collins\\
	Commercial Systems Flight Controls Safety Engineering\\Email: Michael.Peterson@rockwellcollins.com}}


\maketitle

\begin{abstract}
Risk and fault analysis are important activities that help to ensure that critical systems operate in an expected way. As critical systems become larger, the methods used to perform both formal verification and fault analysis require scalable algorithms. Compositional verification uses the idea that the correctness of a system may be determined from the correctness of its components. This has shown to be an efficient way of verifying system safety properties. A prominant artifact of fault analysis is a fault tree: a graphical representation of the failure modes of a critical system. In this paper, we describe a new approach in which fault trees can be automatically generated for a system model using compositional reasoning and formal verification techniques. We describe the technique of generating these safety artifacts, prove that our approach is sound, and describe its implementation in the Osate tool suite for AADL. We then present experiments in which we benchmark our approach in terms of scalability and the artifacts produced. 
\end{abstract}

\section{Introduction}
\label{sec:intro}

%Mats' intro
System safety analysis is crucial in the development life cycle of critical systems to ensure adequate safety as well as demonstrate compliance with applicable standards. A prerequisite for any safety analysis is a thorough understanding of the system architecture and the behavior of its components; safety engineers use this understanding to explore the system behavior to ensure safe operation, assess the effect of failures on the overall safety objectives, and construct the accompanying safety analysis artifacts. Developing adequate understanding, especially for software components, is a difficult and time consuming endeavor. Given the increase in model-based development in critical systems~\cite{Joshi05:Dasc,CAV2015:BoCiGrMa,info17:HaLuHo,5979344,Gudemann:2010:FQQ:1909626.1909813}, leveraging the resultant models in the safety analysis process holds great promise in terms of analysis accuracy as well as efficiency.

In this paper we describe the \emph{Safety Annex} for the system engineering language AADL (Architecture Analysis and Design Language), a SAE Standard modeling language for Model-Based Systems Engineering (MBSE)~\cite{AADL_Standard}. The Safety Annex allows an analyst to model the failure modes of components and then ``weave'' these failure modes together with the original models developed as part of MBSE. The safety analyst can then leverage the merged behavioral models to propagate %failures
errors through the system to investigate their effect on the safety requirements. %(implicit %failure
%error propagation). 
Determining how %faults
errors propagate through software components is currently a costly and time-consuming element of the safety analysis process. 
\begin{comment} 
The use of behavioral contracts to capture the implicit %fault
error propagation characteristics of software component is a significant benefit for safety analysts.  
In addition, the annex allows modeling of explicit %failure 
error propagation that is not captured through the behavioral models, for example, the effect of a single electrical failure on multiple software components or the effect hardware failure (e.g., an explosion) on multiple behaviorally unrelated components. 
\end{comment}
The use of behavioral contracts to capture the %implicit %fault
error propagation characteristics of software component without the need to add separate propagation specifications (\emph{implicit} error propagation) is a significant benefit for safety analysts.  
In addition, the annex allows modeling of %explicit %failure 
dependent faults that are not captured through the behavioral models (\emph{explicit} error propagation)},
%error propagation that is not captured through the behavioral models, 
for example, the effect of a single electrical failure on multiple software components or the effect hardware failure (e.g., an explosion) on multiple behaviorally unrelated components. 
Furthermore, we will describe the tool support enabling engineers to investigate the correctness of the nominal system behavior (where no failures have occurred) as well as the system's resilience to component failures. We illustrate the work with a substantial example drawn from the civil aviation domain.

Our work can be viewed as a continuation of work conducted by Joshi et al.~where they explored model-based safety analysis techniques defined over Simulink/Stateflow~\cite{MathWorks} models~\cite{Joshi05:SafeComp,Joshi07:Hase,Joshi05:Dasc,DBLP:conf/cav/BozzanoCPJKPRT15}. Our current work extends and generalizes this work and provide new modeling and analysis capabilities not previously available.  For example, the Safety Annex allows modeling explicit %fault 
error propagation, supports compositional verification and exploration of the nominal system behavior as well as the system's behavior under failure conditions. Our work is also closely related to the existing safety analysis approaches, in particular, the AADL Error Annex (EMV2)~\cite{EMV2}, COMPASS~\cite{10.1007/978-3-642-04468-7_15}, and AltaRica~\cite{PROSVIRNOVA2013127,BieberERTS2018}. Our approach is significantly different from previous work in that unlike EVM2 we leverage the behavioral modeling for implicit %failure 
error propagation.  We provide compositional analysis capabilities not available in COMPASS.  In addition, the Safety Annex  is fully integrated in a model-based development process and environment unlike a stand alone language such as AltaRica. 

The contributions of the Safety Annex and this paper are:
\begin{itemize}
\renewcommand{\labelitemi}{\textbullet}
		\item close integration of behavioral fault analysis into the {\em Architecture Analysis and Design Language} AADL, which allows close connection between system and safety analysis and system generation from the model,
		\item support for {\em behavioral specification of faults} and their {\em implicit propagation} through behavioral relationships in the model, in contrast to existing AADL-based annexes (HiP-HOPS, EMV2) and other related toolsets (COMPASS, Cecilia, etc.),
		\item additional support to capture binding relationships between hardware and software and logical and physical communications, and
		\item guidance on integration into a traditional safety analysis process.
\end{itemize}

%The remainder of the paper is organized as follows. INSERT WHATEVER IT ENDED UP LOOKING LIKE. 

\begin{comment}
%Danielle's and Janet's intermediate intro

System safety analysis techniques are crucial in the development life cycle of highly integrated/complex aircraft systems and are used to show compliance with certification requirements. A prerequisite of performing any safety assessment of a system design is to understand how the system is intended to work, primarily focusing on the relationship between component outputs and the overall behavior of the system. The safety engineers then use this information to conduct safety analysis, construct the safety analysis artifacts, and compare the analysis results with established safety objectives and safety-related requirements. Acquiring knowledge about the behavior of the software applications hosted in a system and its impact on the overall system behavior is typically a time consuming and involved process.

To help solve this problem, researchers and industry practitioners have turned to the use of models. Models have been shown to be an effective way to help engineers capture, understand and analyze complex systems. Previous work has been done showing the benefits of leveraging the system model in the safety analysis process~\cite{Joshi05:SafeComp,Joshi07:Hase,Joshi05:Dasc,DBLP:conf/cav/BozzanoCPJKPRT15}.

%In order to effectively assist safety engineers to acquire the knowledge on software application behaviors and assess their effects on the overall behavior of the system, the models should allow system designers to capture the expected behaviors of the software application and the expected behavioral propagations among different components/application in the system; and allow safety engineers to leverage the same model provided by system designers, capture failure modes for individual components, and automatically assess the effects to the overall system through the behavioral propagations built in the existing model.

During this safety analysis process, it is important to reason about faults and how faulty component behaviors can impact the overall system behavior. In order to address the problem of understanding the model and complex system, it is advantageous to provide an automated analysis framework that allows for various types of fault definitions, propagations, and modeling options. 

This paper introduces a tool that provides a solution to these problems: the Safety Annex for the system engineering language called Architecture Analysis and Design Language (AADL), a widely used SAE Standard design language for MBSE applications~\cite{AADL_Standard}. Given a system model in AADL and a behavioral model developed in the Assume Guarantee Reasoning Environment (AGREE)~\cite{NFM2012:CoGaMiWhLaLu}, the Safety Annex is a fault modelling tool that utilizes model checking in order to analyze the behavior of a system in the presence of faults. The Safety Annex allows safety engineers to leverage existing models from system development for conducting assessment. 

Throughout this paper we show that the Safety Annex allows both implicit and explicit failure propagation which gives richer fault modeling capabilities than comparable tools. It is also shown how behavioral information regarding the active faults, the component properties and the overall system behavior when faults are active are provided. We demonstrate that the toolset (AADL, AGREE, and Safety Annex) captures behaviors of both the nominal model (absence of faults) and the faulty model in a cleanly separated and yet analyzable fashion. This serves to preserve the system model for the systems engineering process and simultaneously be able to see their combined effect on the system behavior. 


 %Using a Model-Based Safety Analysis (MBSA) approach allows safety engineers to weave a fault model into the entire MBSE process while preserving the separation of a system model and a fault model.


%To help solve this problem, researchers and industry practitioners have turned to Model-based System Engineering (MBSE). Models have been shown to be an effective way to help engineers capture, understand and analyze complex systems. Previous work has been done showing the benefits of leveraging the system model in the safety analysis process~\cite{Joshi05:SafeComp,Joshi07:Hase,Joshi05:Dasc,DBLP:conf/cav/BozzanoCPJKPRT15}. Using a Model-Based Safety Analysis (MBSA) approach allows an analyst \janet{use the term ``safety engineer'' only or both ``safety engineer'' and ``safety analyst''?} to weave a fault model into the entire MBSE process while preserving the separation of a system model and a fault model.
\end{comment}

\iffalse

Throughout the development life cycle of highly-integrated/complex aircraft systems, safety assessment process is a crucial piece in asserting development assurance, and is used to show compliance with certification requirements and meeting a company's internal safety standards. A prerequisite of performing any safety assessment of a system design is to understand how the system is intended to work, primarily focusing on the relationship between component outputs and the overall behavior of the system. The safety engineers then use the acquired understanding to conduct safety analysis, construct the safety analysis artifacts, and compare the analysis results with established safety objectives and safety-related requirements.  

In practice, prior to performing the safety assessment of a system, the safety engineers are often equipped with the domain knowledge about the system, but do not necessarily have detailed knowledge of how the software functions are designed. Acquiring the required knowledge about the behavior and implementation of each software function in a system is typically the most time consuming and involved step in the process.

Industry practitioners have come to realize the benefits and importance of
using models to assist the safety assessment process, such as to better understand system behaviors, communicate with system designers, capture the failure propagations, and manage and analyze more complex systems. And a revision to the Guidelines and Methods for Conducting Safety Assessment Process on Civil Airborne Systems and Equipment~\cite{SAE:ARP4761} to include {\em model based safety analysis} is under way.

%condensed version
%System safety assessment is a crucial process in the development of complex airborne systems to show that the relevant safety requirements are met. Acquiring the required knowledge about how the software functions are intended to work in such systems has shown in practice a very involved and time consuming task. Existing approaches that annotate the system architecture model with failure modes and fault propagations help safety analysts better communicate with system designers and address the complexity of the system. However, knowledge on how the faults propagate through the components still needs to be acquired by safety analysts before such information can be captured in the model. Acquiring the information on fault propagation is still a manual effort.

We think that the following criteria are important for the models to help safety analysts effectively acquire the knowledge about system/software behaviors and capture/analyze failure propagations:

\begin{itemize}
	\item Allow safety engineers to leverage existing models from system development for conducting assessment. %This captures the current state of the system design as it moves through the system development lifecycle, reducing the gap in comprehending the system behavior and transferring the knowledge between the system designers and the safety analysts.
	\item Support capturing behaviors of nominal and faulty behaviors in the system that are cleanly separated, yet analyzable in an integrated fashion to see their combined effect on the system.
	\item Enable safety engineers to inject failures/faults at component level, and assess the effect of behavioral propagation at system level, without needing to acquire the knowledge on the propagation beforehand. 
	\item Allow safety engineers to add failure propagations to the model that may not be behavioral related such as common cause/hardware dependent faults (e.g., common failures such as pipe burst that can propagate through physical systems).
\end{itemize}

\janet{To add: describe our approach that satisfy the criteria and all other work don't}
%The methodology described in this paper enables safety analysts to specify faults and faulty behaviors at individual components (using the Safety Annex for the Architecture Analysis and Design Language (AADL)). 

%The provided tool support auto weaves the faults into the nominal system model provided by system designers. No additional effort is needed to specify fault propagations as the faulty behavior propagates in the nominal system model the same as the normal behavior. The behavior of the system in the presence of faults are verified using model checking through Assume Guarantee Reasoning Environment (AGREE), and the effects of any triggered fault are manifested in the formal analysis results.

%1. Introduce our approach and it addresses all that
%2. behavioral propagation of failures
%3. We could just describe/split in implicit and explicit fault/failure propagations. E.g., for explicit failure propagation, now we can connect behaviorally unrelated components.
%3. You activate a fault/inject a failure to the system, so it's not fault propagation, but failure propagation

%How we evaluate our work in comparison with others'
%1. There are other approaches support some of them. However, they don't support ...
%What are the related work and why they don't solve the problem:
%Researchers like Anjali have explored ...
%EMV2
%xSAP
%2. What we do is different from XXX because ...
%3. The EMV2 is really talking about failure propagation. They view errors as corrupted states. That lead to certain level of confusion. 
%4. In ARP4754, an error is treated as a source of fault, but a fault can happen without error. An error might lead to a fault. A SW can have errors as software doesn't fail on its own - someone has to put it there

\begin{comment}
%This paper describes a new methodology with tool support for model based safety analysis. It is implemented as a {\em Safety Annex} for the Architecture Analysis and Design Language (AADL). The Safety Annex provides the ability to describe faults and faulty component behaviors in AADL models. In contrast to previous AADL-based approaches, the Safety Annex leverages a formal description of the nominal system behavior to propagate faults in the system. This approach ensures consistency with the rest of the system development process and simplifies the work of safety engineers. The language for describing faults is extensible and allows safety engineers to weave various types of faults into the nominal system model. The Safety Annex supports the injection of faults into component level outputs, and the resulting behavior of the system can be analyzed using model checking through the Assume-Guarantee Reasoning Environment (AGREE).

System safety analysis techniques are well-established and are a required activity in the development of safety-critical systems. Model-based systems engineering (MBSE) methods and tools based on formal methods now permit system-level requirements to be specified and analyzed early in the development process~\cite{NFM2012:CoGaMiWhLaLu,CAV2015:BoCiGrMa}. While model-based development methods are widely used in the aerospace industry, they are only recently being applied to system safety analysis.  

%How can we leverage these model-based methods and tools to perform safety analysis based on models of the system architecture and initial functional decomposition? Can these design models be integrated into the safety analysis process to help guarantee accurate and consistent results?
%Seeking solutions to these questions are especially important as the amount of safety-critical hardware and software in various domains has drastically increased due to the demand for greater autonomy, capability, and connectedness.

In this paper, we describe a {\em Safety Annex} for the Architecture Analysis and Design Language (AADL)~\cite{FeilerModelBasedEngineering2012} that provides the ability to reason about faults and faulty component behaviors in AADL models. In the Safety Annex approach, we use formal assume-guarantee contracts to define the nominal behavior of system components. The nominal model is then verified using the Assume Guarantee Reasoning Environment (AGREE)~\cite{NFM2012:CoGaMiWhLaLu}. The Safety Annex  provides a way to weave faults into the nominal system model and analyze the behavior of the system in the presence of faults. The Safety Annex also provides a library of common fault node definitions that is customizable to the needs of system and safety engineers. Our approach adapts the work of Joshi et. al in
~\cite{Joshi05:Dasc} to the AADL modeling language, and provides a domain specific language for the kinds of analysis performed manually in previous work~\cite{Stewart17:IMBSA}.  %More information on the approach is available in~\cite{Stewart17:IMBSA}, and the tool and relevant documentation can be found at: \small \url{https://github.com/loonwerks/AMASE/}. \normalsize

%There are other tools purpose-built for safety analysis, including AltaRica~\cite{PROSVIRNOVA2013127}, smartIFlow~\cite{info8010007} and xSAP~\cite{DBLP:conf/tacas/BittnerBCCGGMMZ16}. These notations are separate from the system development model. Other tools extend existing system models, such as HiP-HOPS~\cite{CHEN201391} and the AADL Error Model Annex, Version 2 (EMV2)~\cite{EMV2}. EMV2 uses enumeration of faults in each component and explicit propagation of faulty behavior to perform safety analysis. The required propagation relationships must be manually added to the system model and can become complex, leading to potential omissions and inconsistencies.

The Safety Annex supports model checking and quantitative reasoning by attaching behavioral faults to components and then using the normal behavioral propagation and proof mechanisms built into the AGREE AADL annex. This allows users to reason about the evolution of faults over time, and produce counterexamples demonstrating how component faults lead to system failures. It can serve as the shared model to capture system design and safety-relevant information, and produce both qualitative and quantitative description of the causal relationship between faults/failures and system safety requirements.
%
Thus, the contributions of the Safety Annex and this paper are:
\begin{itemize}
\item Close integration of behavioral fault analysis into the {\em architectural design language} AADL, which allows close connection between system and safety analysis and system generation from the model,
\item support for {\em behavioral specification of faults} and their {\em implicit propagation} through behavioral relationships in the model, in contrast to existing AADL-based annexes (HiP-HOPS, EMV2) and other related toolsets (COMPASS, Cecilia, etc.),
\item additional support to capture binding relationships between hardware and software and logical and physical communications, and
\item guidance on integration into a traditional safety analysis process.
\end{itemize}
%\mike{What are our contributions?}
\end{comment}

\fi

\section{Background}

Background information on fault trees, IVCs, and Soteria.




\section{Methodology}
Given a complex model, it is often useful to extract traceability information related to the proof, in other words, which portions of the model were necessary to construct the proof. To this end, an algorithm was developed that efficiently computes the \textit{inductive validity cores} (IVC) within a model necessary for the proofs of safety properties for sequential systems \cite{DBLP:journals/corr/GhassabaniGW16}. 

\subsection{Preliminaries}
Given a state space $S$, a transition system $(I,T)$ consists of the initial state predicate $I : S \rightarrow \{0,1\}$ and a transition step predicate $T : S \times S \rightarrow \{0,1\}$. Reachability for $(I,T)$ is defined as the smallest predicate $R : S \rightarrow \{0,1\}$ which satisfies the following formulas:
\begin{center}
$\forall s. I(s) \Rightarrow R(s)$\\
$\forall s, s' .  R \land T(s,s') \Rightarrow R(s')$\\
\end{center}
A safety property $\mathcal{P} : S \to \{0,1\}$ is a state predicate. A safety property $\mathcal{P}$ holds on a transition system $(I,T)$ if it holds on all reachable states. More formally, $\forall s . R(s) \Rightarrow \mathcal{P}(s)$. When this is the case, we write $(I,T) \vdash\mathcal{P}$. Following Ghassabani, et. al. \cite{DBLP:journals/corr/GhassabaniGW16}, we formalize IVCs as follows.\\

\begin{definition}Inductive Validity Core\\
 Let $(I,T)$ be a transition system and let $\mathcal{P}$ be a safety property with $(I,T) \vdash \mathcal{P}$. Then $S \subseteq T$ is an \textit{inductive validity core} for $(I,T) \vdash \mathcal{P}$ iff $(I,S) \vdash\mathcal{P}$.  \\
\end{definition}

\begin{definition}Minimal Inductive Validity Core\\
An inductive validity core $S$ for $(I,T) \vdash \mathcal{P}$ is minimal iff $! \exists S' . S' \subset S \ni (I,S') \vdash \mathcal{P}$. \\
\end{definition}

A fault tree is a directed acyclic graph (DAG) consisting of the node types \textit{events} and \textit{gates}. An event is an occurance within the system, typically the failure of a subsystem down to an individual component. Events can be grouped into \textit{basic events} (BEs), which occur independently, and \textit{intermediate events} which occur dependently and are caused by one or more other events. The event at the top of the tree, the \textit{top level event} (TLE), is the event being analyzed. This event models the failure of the system (or subsystem) under consideration. The gates represent how failures propagate through the system and how failures in subsystems can cause system wide failures. The two main logic symbols used are the Boolean logic AND-gates and OR-gates.
An AND-gate is used when the undesired top level event can only occur when all the lower conditions are true. The OR-gate is used when the undesired event can occur if any one or more of the next lower conditions is true. This is not a comprehensive list of gate types, but we focus our attention on these two common gate types.\\

To formalize a fault tree (FT), we use $GateTypes = \{And, Or\}$. Following Ruijters, et. al. \cite{RuijtersSurvey}, we formalize FT as follows. \\

\begin{definition}Fault Tree\\ 
A FT is a 4-tuple $F = \langle BE, G, T, I \rangle$ consisting of the following components. 
\begin{itemize}
\item BE is the set of basic events
\item G is the set of gates with $BE \cap G = \emptyset$. We write $E = BE \cup G$ for the set of elements.
\item $T: G \to GateTypes$ is a function that describes the type of each gate.
\item $I: G \to P(E)$ describes the inputs of each gate. We require that $I(G) \neq \emptyset$.
\end{itemize}
\end{definition}

The graph formed by $\langle E, I \rangle$ is a directed acyclic graph with a unique root $TLE$ which is reachable from all nodes. \\

\begin{definition}Semantics of a Fault Tree\\  
The semantics of FT F is a function $\pi_F : \mathcal{P}(BE) \times E \vdash \{0,1\}$ where $\pi_F(S, e)$ indicates whether $e$ fails given the set $S$ of failed BEs. It is defined as follows. 
\begin{itemize}
\item For $e \in BE$, $\pi_F(S,e) = e \in S$.
\item For $g \in G$ and $T(g) = And$, let\\ $\pi_F(S,g) = \land_{x \in I(g)} \pi_F(S, x)$
\item For $g \in G$ and $T(g) = Or$, let\\ $\pi_F(S,g) = \lor_{x \in I(g)} \pi_F(S, x)$ \\
\end{itemize}
\end{definition}

The interpretation of the TLE $t$ is written as $\pi_F(S,t) = \pi_F(S)$. If the failure of $S$ causes the TLE to occur, we write $\pi_F(S) = 1$. 

 In qualitative analysis, cut sets and minimal cut sets provide information about the vulnerabilities of a system in terms of its basic events. A \textit{cut set} is a set of components that together can cause a system to fail. A \textit{minimal cut set} is a cut set which contains the minimum number of basic events required in order to cause the TLE to occur. More formally, these are defined as follows. \\

\begin{definition}Cut Set\\   
 $C \subseteq BE$ is a cut set of FT F if $\pi_F(C) = 1$. \\
\end{definition}

\begin{definition}Minimal Cut Set\\    
$C \subseteq BE$ is a MCS if $\pi_F(C) = 1 \land \forall C' \subset C. \pi_F(C') = 0$. In other words, a minimal cut set (MCS) is a cut set of which no subset is a cut set. \\
\end{definition}

Given the formalisms defined previously, we show that finding the minimal IVCs is equivalent to finding the MCS. \\

\begin{theorem} Finding all minimal IVCs is equivalent to finding the set of all MCSs\\

\begin{proof} 
Let $T = \{f_1, f_2, ..., f_n\}$ be the set of model elements corresponding to faults for the components and let $\mathcal{P}_{tle}$ be the top level event (TLE). By the definition for IVC, we know that $S \subseteq T$ is an IVC for $(I, T) \vdash \mathcal{P}_{tle}$ iff $(I, S) \vdash \mathcal{P}_{tle}$. 
In this case, $\pi_F(S) = 1$, i.e. given the set $S \subseteq T$ of failed model elements, the top level event occurs. 
Since $S$ is minimal IVC, for any $M \subset S$, $(I, M) \doesnotentail \mathcal{P}_{tle}$. Hence it follows that $\pi_F(M) = 0$ for any $M \subset S$.
\end{proof}
\end{theorem}

\subsection{Quantitative analysis}
Quantitative analysis methods derive numerical values for fault trees. One of these calculations that is of particular interest to the safety engineering community is the probability of the occurance of the top level event (TLE). This value is of interest because in certain critical systems, the top level properties must be proved safe within a certain probabilistic threshold~\cite{SAE:ARP4761}. The next section describes probability theory and provides a description of how they are used to calculate the probability of the TLE. These definitions are based on the ones given in~\cite{RuijtersSurvey}.\\

\danielle{I will add the probability formalisms here later. For now, I just want to clarify my findings regarding the proof that Mike was talking about yesterday.}\\
Assume events occur independently. \\

Or gate: \\
Probability is defined as follows: \\
$P(A \lor B) = P(A) + P(B) - P(A \land B)$. \\
Assume that the probability of events A, B is quite small (this is called the Rare Event Approximation). Then the term $P(A \land B)$ will also be quite small and hence is an ``error term". If we drop this term from the calculations, we end up with the approximated probability being: $P(A) + P(B) \geq P(A) + P(B) - P(A \land B)$. \\ This is clearly a conservative estimate. \danielle{It is also shown in the Fault Tree Handbook that this error is less than 0.1 I believe. I will have to review that again, but there is a bound on this error that has been previously proven.}

And gate: \\
With the calculations of the and gate, this is where independence is really required. Using Bayes Rule, we have $P(A \land B) = P(B)P(A|B) = P(A)P(B|A)$ for conditionally dependent events and $P(A \land B) = P(B)P(A)$ for independent events. If we assume independence when the events are NOT independent, in the worst case scenario, we get something like $B$ is completely dependent on $A$. Therefore $P(B|A) = 1$ and $P(A \land B) = P(B) \geq P(B)P(A)$. Not a conservative estimation. In the case of dependence, joint probability values would be required (acommon cause analysis).  \\

\danielle{Any other kind of gate uses combinations of these rules. These are not proofs that I made, I just looked at the logic of the operations. I am not sure how much we want to include in the paper. I assume a description is necessary, but a proof? No, this isn't a proof... This is just following definitions down a short little jaunt in the woods.}








\section{Case Studies}

Case studies or tool comparisons.





\section{Related Work}
\label{sec:related_work}

Formal model based systems engineering (MBSE) methods and tools now permit system level requirements to be specified and analyzed early in the development process~\cite{QFCS15:backes,CIMATTI2015333, NFM2012:CoGaMiWhLaLu, hilt2013:MuWhRaHe}. Design models from which aircraft systems are developed can be integrated into the safety analysis process to help guarantee accurate and consistent results. Integration of MBSA into safety analysis process is described by Bozzano and Villafiorita~\cite{Bozzano:2010:DSA:1951720}. There are tools that currently support reasoning about faults in architecture description languages such as SysML and AADL. We provide here a brief overview of the most relevant safety analysis tools. 

Tools such as the AADL Error Model Annex, Version 2 (EMV2)~\cite{EMV2} and HiP-HOPS for EAST-ADL~\cite{CHEN201391} primarily utilize \textit{qualitative} reasoning. Faults are enumerated and the propagations through system components are explicitly described. Given many possible faults, these propagation relationships increase in complexity and understandability. Interactions are easily overlooked by analysts and thus not explicitly described. In our approach, faults are injected into the system and behaviorally propagated through the use of assume-guarantee statements in AGREE. This avoids the difficulties inherent with explicit fault enumeration and propagation. 

% Say something about behavioral propagation here ---------------------------------------------------

SmartIFlow~\cite{info8010007} is a purpose-built safety analysis tool that describes components and their interactions using finite state machines and events. Verification is done through an explicit state model checker which returns sets of counterexamples for safety requirements in the face of failures. The mechanism to keep the search space size under control during model checking relies on expert knowledge from engineers. This limits the number of failures and removes the possibility of certain failure conditions. Due to this drawback, scalability to industrial sized systems is difficult. The safety annex described in this research is not a standalone model, but is made to be incorporated into the system safety assessment process as described in section 2.1. \danielle{Scalability: in terms of the safety annex, just reference the case study described in this work?}

Another approach has been introduced by G{\"u}demann et. al.~\cite{Gudemann:2010:FQQ:1909626.1909813}. System models are constructed in SAML (Safety Analysis and Modeling Language) which can be imported into several analysis tools like NuSMV~\cite{Cimatti2000} or PRISM~\cite{CAV2011:KwNoPa} (Probabilistic Symbolic Model Checker) or MRMC probabilistic model checker~\cite{Katoen:2005:MRM:1114692.1115230}. SAML is used for both qualitative and quantitative analyses and allows for the combination of discrete probability distributions and non-determinism. The framework is used to create a system model comprised of software control, hardware components, environment and failure mode modeling. 

In earlier work, an approach to MBSA was demonstrated using the Simulink\textsuperscript{\textregistered} notation~\cite{Joshi05:SafeComp,Joshi05:Dasc}. In this approach, a behavioral model of system dynamics was used to reason about the effects of faults in the system. This approach allows an implicit and natural notion of fault propagation through the system. However, non-functional architectural properties were not captured as Simulink is not designed as an architecture description language. In our approach, we are applying quantitative reasoning and implicit fault propagation to a more rich architecture language.

Similarly, AltaRica~\cite{PROSVIRNOVA2013127} has been incorporated into Cecilia OCAS as a model based safety analysis tool~\cite{BieberERTS2018}. Safety assessment, fault tree generation, and functional verification can be performed with the aid of NuSMV model checking~\cite{symbAltaRica}. Failure states are defined throughout the system and flow variables are updated through the use of assertions~\cite{Bieber04safetyassessment}. A limitation of this is that Linear Temporal Logic operators are required in some of the failure definitions. This is a downfall to the safety community/engineers who are not familiar with LTL~\cite{Bieber04safetyassessment}. 

Closely related to our work is the model-based safety assessment toolset called COMPASS (Correctness, Modeling project and Performance of Aerospace Systems)~\cite{10.1007/978-3-642-04468-7_15}. COMPASS uses the SLIM language which is based on AADL, for its input models. The SLIM (System Level Integrated Modeling Language) language was developed by the COMPASS project for modeling hardware and software systems for safety-related tasks~\cite{5185388, criticalembeddedsystems}. System models in SLIM are translated into the input language of NuSMV and xSAP~\cite{DBLP:conf/tacas/BittnerBCCGGMMZ16} is invoked to generate safety analysis artifacts (e.g. Fault Trees, FMEA tables, etc.)~\cite{compass30toolset}.

Formal verification tools based on model checking have been used to automate the generation of safety artifacts~\cite{symbAltaRica,10.1007/978-3-540-75596-8-13, DBLP:conf/tacas/BittnerBCCGGMMZ16}. This approach has limitations in terms of scalability and readability of the fault trees generated. Work has been done towards mitigating these limitations by the scalable generation of readable fault trees~\cite{10.1007/978-3-319-11936-6-7}.



\section{Conclusions \& Future Work}
In this paper, we describe our initial work towards performing MBSA using the AADL architecture description language using a failure effect modeling approach.  Our goal is to be able to perform safety analysis on common models used by systems and safety engineers for functional and non-functional analyses, schedulability, and perhaps system image generation.  To perform this analysis, we use existing capabilities within AADL to describe the structure of the system, and build on the existing AGREE framework for compositional analysis of components.  

As part of our exploration, we are interested in examining the strengths and weaknesses of our FEM and the AADL Error Annex FLM-based approach.  We believe that the FEM approach has advantages both in terms of brevity of specifications and accuracy of results, and can build on existing analyses performed for systems engineering.  However, there are also risks in the FEM approach involving incomplete or mis-specified properties.  

We illustrated the ideas using architecture models based on the Wheel Braking System model in SAE AIR 6110 \cite{AIR6110} and use this in the evaluation of our approach. Using assume-guarantee compositional reasoning techniques, we prove a top level property of the wheel brake system that states when the brake pedals are pressed in the absence of skidding, there will be hydraulic pressure supplied to the brakes.  

Starting from the error model notions of error types, two main faults were defined: \textit{fail\_to} which will describe failures of valves and pressure regulators and \textit{inverted\_fail} which describes the failures occurring to components that output boolean values. Using the AADL behavioral model of the WBS, these permanent faults were tied into the nominal model in order to reason about how this model behaves in the presence of specific kinds of faults.

In order to demonstrate that the system was resilient to single faults, we modified the model to allow feedback from the wheel pressure to the BSCU.   This changed the way the system responded to faults that were further downstream of the BSCU or Selector and created a chance for the system to switch to alternate forms of hydraulic pressure. We also reasoned about the initialization values of the system in regards to which mode is the starting mode. It is crucial for the system to begin in Normal mode in order to function successfully in the presence of faults.  After model modification and a small weakening of our original property to account for feedback delay, the model does fulfill the top level contract even when a permanent fault of one of the high level components is introduced.

The current capabilities of AGREE are well-suited to specifying faults.  Our approach allows for scalar types of unbounded integers and reals, as well as composite types such as tuples and structures.  It is possible to model systems and reason about them in either discrete time or real-time.  However, adding faults to existing components is cumbersome and can obscure the nominal behaviors of the model.  We are currently examining several fault specification languages, giving special consideration to the xSAP modeling language.

Future research work will involve the continuation of development of the methods and tools needed to perform model-based safety analysis at the system architecture level. By introducing a common set of models for both nominal system design and safety analysis, we hope to reduce the cost of development and improve safety. Our hope is to demonstrate the practicality of formal analysis for early detection of safety issues that would be prohibitively expensive to find through testing and inspection. We will base this research on industry standard notations that are being used in airborne and ground-based avionics in order to ensure transition of this technology.

\subsection*{Acknowledgements} This research was funded by NASA AMASE NNL16AB07T and University of Minnesota College of Science and Engineering Graduate Fellowship.





%\vspace{-0.40cm}
\bibliographystyle{abbrv}

\bibliography{biblio}
%\vspace{-7.25cm}
% This ~ seems to fix an odd bibliography alignment issue


%\ifdefined\TECHREPORT
%\appendix
%
%\section{Appendix: Proof of Equivalence}
%\input{appendix}
%\fi

%\section{Appendix: GPCA CENTA Model}
%\label{appendix:gpcacenta}
%\begin{figure}[!ht]
%\begin{center}
%\includegraphics[scale=0.6]{images/sampled_pca.PNG} %[trim = 0 2 0 0, clip=true]{Comp}
%\caption{GPCA AGREE Properties modeled as a Timed Automata} \label{fig:samplepca}
%\end{center}
%\end{figure}

%\balancecolumns

\end{document} 