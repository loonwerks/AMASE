\section{Related Work}
\label{sec:related_work}

Formal verification tools based on model checking have been used to automate the generation of safety artifacts~\cite{symbAltaRica,10.1007/978-3-540-75596-8-13, DBLP:conf/tacas/BittnerBCCGGMMZ16}. The techniques and tools developed early on suffered from a limitation brought on by difficulty in generating MCSs and readable fault trees or other safety artifacts.

 Work has been done towards mitigating these limitations by the scalable generation of readable fault trees. Othello Contracts Refinement Analysis (OCRA) is a tool that checks refinement of contracts specified in linear temporal logic~\cite{6693137}. This tool was extended in order to provide hierarchically organized and automatically generated fault trees~\cite{10.1007/978-3-319-11936-6-7}. This process entails extending the model to include failure ports on both the input (failure of environment) and output (failure of component). The fault tree is generated starting at the TLE and linking it to the intermediate events that could cause this event to occur. The tree is defined recursively in this fashion. 

\danielle{Adding these paragraphs to incorporate together later.}\\

\danielle{Efficient Anytime Techniques of prob analysis}
Given a set of MCSs and probabilities for leaf level faults (BEs), the computation of this TLE probability is possible using BDD techniques. The difficulty in this calculation is that often with large systems it is not easy (or possible) to get MCSs. This problem was addressed in previous work using an `'anytime'' approach in which the TLE probability is approximated while utilizing model checking to compute MCSs of an xSAP model~\cite{CAV2015:BoCiGrMa,mattareiThesis}. This provides a way to estimate the probability using lower and upper bounds and even if the model checker cannot complete its task of finding all MCSs, the probability range is known. 


\danielle{COMPASS}
Closely related to our work is the model-based safety assessment toolset called COMPASS (Correctness, Modeling project and Performance of Aerospace Systems)~\cite{10.1007/978-3-642-04468-7_15}.  COMPASS is a tool suite that uses the SLIM language, which is based on a subset of AADL, for its input models~\cite{5185388, criticalembeddedsystems}. In SLIM, a nominal system model and the error model are developed separately and then transformed into an extended system model.  This extended model is automatically translated into input models for the NuSMV model checker~\cite{Cimatti2000, NuSMV}, MRMC (Markov Reward Model Checker)~\cite{Katoen:2005:MRM:1114692.1115230, MRMC}, and RAT (Requirements Analysis Tool)~\cite{RAT}. The safety analysis tool xSAP~\cite{DBLP:conf/tacas/BittnerBCCGGMMZ16} can be invoked in order to generate safety analysis artifacts such as fault trees and FMEA tables~\cite{compass30toolset}. The fault tree generation provided utilizes the ``anytime' techniques for probabilistic analysis and the contract based approach as well as the xSAP approach for the safety artifact generation'~\cite{10.1007/978-3-642-04468-7_15}.







