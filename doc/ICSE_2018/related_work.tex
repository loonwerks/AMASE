\section{Related Work}
\label{sec:related_work}








\danielle{Who makes fault trees currently in our list of tools here? Who makes compositional fault trees? Why is our way better/different? Why is what we do important for AADL? AADL currently has a method to develop fault trees using error annex. Why is this better/different?}

A model-based approach for safety analysis was proposed by Joshi et. al in \cite{Joshi05:Dasc, Joshi05:SafeComp, Joshi07:Hase}.  In this approach, a safety analysis system model (SASM) is the central artifact in the safety analysis process, and traditional safety analysis artifacts, such as fault trees, are automatically generated by tools that analyze the SASM.

The contents and structure of the SASM differ significantly across different conceptions of MBSA.  We can draw distinctions between approaches along several different axes.  The first is whether they propagate faults explicitly through user-defined propagations, which we call {\em failure logic modeling} (FLM) or through existing behavioral modeling, which we call {\em failure effect modeling} (FEM).  The next is whether models and notations are {\em purpose-built} for safety analysis vs. those that extend {\em existing system models} (ESM).

For FEM approaches, there are several additional dimensions.  One dimension involves whether {\em causal} or {\em non-causal} models are allowed.  Non-causal models allow simultaneous (in time) bi-directional failure propagations, which allow more natural expression of some failure types (e.g. reverse flow within segments of a pipe), but are more difficult to analyze.  A final dimension involves whether analysis is {\em compositional} across layers of hierarchically-composed systems or {\em monolithic}.  Our approach is an extension of AADL (ESM), causal, compositional, mixed FLM/FEM approach.

Tools such as the AADL Error Model Annex, Version 2 (EMV2)~\cite{EMV2} and HiP-HOPS for EAST-ADL~\cite{CHEN201391} are {\em FLM}-based {\em ESM} approaches.  As previously discussed, given many possible faults, these propagation relationships require substantial user effort and become more complex.  In addition, it becomes the analyst's responsibility to determine whether faults can propagate; missing propagations lead to unsound analyses.  In our Safety Annex, propagations occur through system behaviors (defined by the nominal contracts) with no additional user effort.

Closely related to our work is the model-based safety assessment toolset called COMPASS (Correctness, Modeling project and Performance of Aerospace Systems)~\cite{10.1007/978-3-642-04468-7_15}.  COMPASS is a mixed {\em FLM/FEM}-based, {\em causal} {\em compositional} tool suite that uses the SLIM language, which is based on a subset of AADL, for its input models~\cite{5185388, criticalembeddedsystems}. In SLIM, a nominal system model and the error model are developed separately and then transformed into an extended system model.  This extended model is automatically translated into input models for the NuSMV model checker~\cite{Cimatti2000, NuSMV}, MRMC (Markov Reward Model Checker)~\cite{Katoen:2005:MRM:1114692.1115230, MRMC}, and RAT (Requirements Analysis Tool)~\cite{RAT}. The safety analysis tool xSAP~\cite{DBLP:conf/tacas/BittnerBCCGGMMZ16} can be invoked in order to generate safety analysis artifacts such as fault trees and FMEA tables~\cite{compass30toolset}.  COMPASS is an impressive tool suite, but some of the features that make AADL suitable for SW/HW architecture specification: event and event-data ports, threads, and processes, appear to be missing, which means that the SLIM language may not be suitable as a general system design notation (ESM).

SmartIFlow~\cite{info8010007} is a {\em FEM}-based, {\em purpose-built}, {\em monolithic} {\em non-causal} safety analysis tool that describes components and their interactions using finite state machines and events. Verification is done through an explicit state model checker which returns sets of counterexamples for safety requirements in the presence of failures.  SmartIFlow allows {\em non-causal} models containing simultaneous (in time) bi-directional failure propagations.  On the other hand, the tools do not yet appear to scale to industrial-sized problems, as mentioned by the authors~\cite{info8010007}: ``As current experience is based on models with limited size, there is still a long way to go to make this approach ready for application in an industrial context''.


The Safety Analysis and Modeling Language (SAML)~\cite{Gudemann:2010:FQQ:1909626.1909813} is a {\em FEM}-based, {\em purpose-built}, {\em monolithic} {\em causal} safety analysis language.  System models constructed in SAML can be used used for both qualitative and quantitative analyses. It allows for the combination of discrete probability distributions and non-determinism. The SAML model can be automatically imported into several analysis tools like NuSMV~\cite{Cimatti2000}, PRISM (Probabilistic Symbolic Model Checker)~\cite{CAV2011:KwNoPa}, or the MRMC probabilistic model checker~\cite{Katoen:2005:MRM:1114692.1115230}. 

AltaRica~\cite{PROSVIRNOVA2013127,BieberERTS2018} is a {\em FEM}-based, {\em purpose-built}, {\em monolithic} safety analysis language with several dialects.  There is one dialect of AltaRica which use dataflow ({\em causal}) semantics, while the most recent language update (AltaRica 3.0) uses non-causal semantics.  The dataflow dialect has substantial tool support, including the commercial Cecilia OCAS tool from Dassault.  For this dialect the Safety assessment, fault tree generation, and functional verification can be performed with the aid of NuSMV model checking~\cite{symbAltaRica}. Failure states are defined throughout the system and flow variables are updated through the use of assertions~\cite{Bieber04safetyassessment}.  AltaRica 3.0 has support for simulation and Markov model generation through the OpenAltaRica (www.openaltarica.fr) tool suite.

Formal verification tools based on model checking have been used to automate the generation of safety artifacts~\cite{symbAltaRica,10.1007/978-3-540-75596-8-13, DBLP:conf/tacas/BittnerBCCGGMMZ16}. This approach has limitations in terms of scalability and readability of the fault trees generated. Work has been done towards mitigating these limitations by the scalable generation of readable fault trees~\cite{10.1007/978-3-319-11936-6-7}.

\danielle{Adding this to incorporate later.}
Given a set of MCSs and probabilities for leaf level faults (BEs), the computation of this TLE probability is possible using the techniques described in section III. The difficulty in this calculation is that often with large systems it is not easy (or possible) to get MCSs. This problem was addressed in previous work using an `'anytime'' approach in which a number of algorithms are tested in order to approximate the TLE probability while utilizing model checking to compute MCSs of an xSAP model~\cite{CAV2015:BoCiGrMa}. Nine different algorithms were used in this approximation technique on the WBS TLEs. This provided an upper and lower bound on the TLE probability even when the model checking procedure timed out and not all MCSs were found. The TLE probability found by our MCS calculations for a subset of the TLEs for the WBS model and the convergence bounds for these properties provided in~\cite{CAV2015:BoCiGrMa,mattareiThesis}.

